% $Id: testing.tex 5678 2017-10-13 15:20:20Z mskala $

%
% MSK 010 testing instructions
% Copyright (C) 2017  Matthew Skala
%
% This program is free software: you can redistribute it and/or modify
% it under the terms of the GNU General Public License as published by
% the Free Software Foundation, version 3.
%
% This program is distributed in the hope that it will be useful,
% but WITHOUT ANY WARRANTY; without even the implied warranty of
% MERCHANTABILITY or FITNESS FOR A PARTICULAR PURPOSE.  See the
% GNU General Public License for more details.
%
% You should have received a copy of the GNU General Public License
% along with this program.  If not, see <http://www.gnu.org/licenses/>.
%
% Matthew Skala
% https://northcoastsynthesis.com/
% mskala@northcoastsynthesis.com
%
\chapter{Testing}

The MSK~010 requires no adjustment or trimming.  All the components are
fixed-value.  It should simply work as soon as it's built.  Nonetheless, to
guard against mistakes in construction, it's a good idea to do some simple
tests before using it in an artistically critical situation.

You will need at least your assembled module, a power supply for it, and a
multimeter.  An oscilloscope would also be useful for troubleshooting, but
is not really necessary.

\section{Short-circuit test}

With no power applied to the module, check for short circuits between the
three power connections on the Board~2 Eurorack power connector.  The two
pins at the bottom, marked with white on the circuit board, are for -12V. 
The two at the other end are for +12V; and the remaining six pins in the
middle are all ground pins.  Check between each pairing of these three
voltages, in both directions (six tests in all).  Ideally, you should use a
multimeter's ``diode test'' range for this; if yours has no such range, use
a low resistance-measuring setting.  It should read infinite in the reverse
direction (positive lead to $-$12V and negative lead to each of the other
two, as well as positive lead to ground and negative to $+$12V) and greater
than 1V or 1k$\Omega$ in the forward direction (reverse those three
tests).  If any of these six measurements is less than 1k$\Omega$ or 1V,
then something is wrong with the build, most likely a blob of solder
shorting between two connections, and you should troubleshoot that before
applying power.

Plug the module into a Eurorack power supply and make sure
neither it nor the power supply emits smoke, overheats, makes any unusual
noises, or smells bad.  If any of those things happen, turn off the power
immediately, and troubleshoot the problem before proceeding.

\emph{Optional}: Plug the module into a Eurorack power supply
\emph{backwards}, see that nothing bad happens, and congratulate yourself on
having assembled the reverse-connection protective circuit properly. 
Reconnect it right way round before proceeding to the next step.

\section{Blinking lights}

Plug the module into a Eurorack power supply and look at the lights on the
front.  They should all fade from green to red and back, at different
speeds, with the slowest one at lower left and the fastest at upper right. 
The very fastest may change faster than your eye can follow, and appear
solid yellow.  The very slowest may take over a minute to complete the
cycle.

When the module is first plugged in after having been turned off for a few
seconds or more, all lights will probably glow \emph{bright} green, brighter
than the green that appears in the usual cycle, for a short time as the
oscillators start up.  This might last up to half a cycle time (about half a
minute for the slowest oscillator).

If any of the LEDs fail to light up at all, glow solid red or green (after
the startup time) instead of cycling, or cycle at a rate much different from
what you expected, then something may be wrong.  Disconnect the power and
investigate what is wrong.

\section{Oscilloscope test}

Power up the module and connect an oscilloscope to each output in turn. 
Each signal should look like a sine wave, with the frequency and period
documented in the relevant ``Board~2'' chapter $\pm$15\%; amplitude
10$\pm$2V peak to peak; and little enough harmonic
distortion and noise that it looks ``clean'' on the scope.

\section{Troubleshooting}

It would require several books to convey all the skills and knowledge useful
in troubleshooting even a simple electronic circuit like this one, but here
are some possible symptoms and some suggestions on diagnosis and treatment.

No response from the module at all; \emph{none} of the lights light up, no
signal on the output.  Most likely a power problem, such as a power cable
plugged in wrong or a short circuit.  It might even be a problem in the
power supply and not the module itself.

No oscillation in one or more oscillators; a light remains solid red or
green for more than a minute, especially if it is brighter than normal.
This could be caused by components being switched around; for instance, not
the same value for the two time-determining resistors in an oscillator.  It
also could be caused by bad soldering; check all your solder connections
visually.  Note that ``solid yellow'' is normal for the
highest-frequency oscillators; that is simply what a very rapid flash between
red and green looks like to human eyes.

During prototyping I also observed a problem of this kind that seemed to
originate with solder flux contamination washed into the board-to-board
connector during cleaning.  It appeared that solvent with dissolved flux in
it had dried inside the connector and formed a varnish-like insulting film. 
In one case all the lights went red on power-up and stayed that way; in
another, just a couple of oscillators seemed to get stuck.  Both were
resolved by unscrewing the standoffs and carefully unplugging and plugging
back in the board-to-board connector a couple of times.  If you see a
similar problem in your build, especially if you have solvent-cleaned your
boards, it would be worth trying this as a first step before doing anything
more drastic.

An oscillator starts, runs for a few cycles, then dies out, with the light
going dark.  This suggests the loop gain is too low.  It is designed with a
moderate safety margin, so something has to be wrong for this to happen. 
Components out of their tolerance range, with values close to the right ones
but not as close as the specifications require, is one possibility.  Make
sure any component substitutions you made were really appropriate. 
Especially for the lowest-frequency oscillators (which use very high
impedance), it's possible that solder-flux residue could also cause this
kind of problem.  Try cleaning the boards with isopropyl alcohol and a
brush.  If you used water-soluble flux for your soldering (not recommended),
then you should clean with water first to remove any remaining flux, and
then isopropyl alcohol to remove the water and any other contaminants.

Four oscillators misbehave, in a checkerboard pattern.  That is, the
oscillators connected to either J1, J3, J6, and J8; or J2, J4, J5, and J7. 
Each of these sets of four is driven by one of the two TL074 chips, so if
they are all bad in a set, it suggests something wrong with the
corresponding TL074.  The \{J1,J3,J6,J8\} set is driven by U2 (on the bottom
when the module is mounted vertically) and the \{J2,J4,J5,J7\} set is
driven by U1 (on the top).  Check, for the relevant chip, that:
\begin{itemize}
  \item it really is a TL074, not some other kind of chip;
  \item it is plugged in snugly;
  \item all the legs of the chip go nicely into the corresponding holes in
    the socket, with none bent outside or folded up under the chip;
  \item it is plugged in \emph{at all} (forgetting to do so is a surprisingly
    common mistake!);
  \item it is plugged in the right way around, with the Pin 1
    indentation or notch at the top and matching the other clues on the
    board (if this is wrong, the chip is probably destroyed and will need to
    be replaced);
  \item there are no solder bridges on the chip socket, unsoldered pins,
    debris clogging the socket holes, or similar;
  \item its decoupling capacitors (the small ceramic ones) are installed and
    there is nothing wrong with their solder joints; and
  \item if all that turns up nothing, try swapping the two TL074s and see if
    the set of failing oscillators also changes.
\end{itemize}

An oscillator's waveform as seen on an oscilloscope has excessive
distortion, perhaps appearing more like a square wave or trapezoid, with or
without rounded corners, instead of a proper sine.  This suggests the loop
gain is too high; it's most likely caused by an incorrect resistor value. 
Also check the amplitude-limiting components (Zener diodes and 27k$\Omega$
resistors); if they are not properly connected, the result might be
flattened waves combined with a much higher than normal amplitude.

An oscillator's waveform as seen on an oscilloscope contains a significant
amount of higher-frequency content, in the kHz or MHz range.  This might be
just a ripple added to the desired low-frequency sine, or it might be a
rail-to-rail square wave overwhelming everything else, at a much higher
frequency than the module is intended to produce.  Either way, it suggests
that the op amp is unstable, which should not be possible in this circuit
with the recommended parts.  Check all the op amp troubleshooting points
mentioned above, and try cleaning to remove solder residue.  See if a
0.01$\mu$F capacitor connected temporarily in parallel with the op amp's
22k$\Omega$ feedback resistor helps, but even if it does, it would be
preferable to figure out what's really wrong instead of just soldering that
in and hoping for the best.

Regular high-frequency spikes superimposed on the oscillator output, with a
repetition rate of twice the power-line frequency (120Hz in North America,
100Hz in many other places): there may be some piece of electrical
equipment, such as a lighting dimmer switch, putting interference into your
mains grounding system.  I observed this while testing an MSK~010 prototype
and it took me a while to figure out it was not something wrong with the
prototype.  Make sure your oscilloscope is plugged into the same mains power
circuit as your bench power supply, and try turning off any dimmer switches.

The module seems to work for a while, but then suddenly fails with at least
four LEDs going solid-colour or dark; it works again when you turn the
power off and on again.  This suggests either or both of the
chips have gone into ``latch-up,'' which could be triggered by spikes on
the power supply, static electricity, or bad voltages applied to the output
jacks; but also check the solder joints for any that may be loose.

Apparent crosstalk between oscillators:  do not trust the LEDs as a way of
detecting this.  The LEDs are driven by simple resistors from voltage
sources; that means when the output voltage of a slow oscillator is close to
the LED's minimum forward voltage for turning on, even the tiniest variation
in voltage will push it back and forth across the boundary, resulting in a
disproportionate fluctuation in light output.  Combining that effect with
optical illusions and the power of suggestion, it is easy to think that you
\emph{see} the slow oscillator LEDs wavering in time with the relatively
fast ones, when you would not be able to \emph{hear} it.  The tiny amount of
visible wavering or instability I've observed in my prototypes is best
thought of as just part of the charm of analog, and not a problem.  However,
a really large amount of crosstalk verifiable by objective measurement of
the output waveforms may indicate issues with the ground connection from
the module to the power supply.
