% $Id: general.tex 5678 2017-10-13 15:20:20Z mskala $

%
% MSK 010 general notes
% Copyright (C) 2017  Matthew Skala
%
% This program is free software: you can redistribute it and/or modify
% it under the terms of the GNU General Public License as published by
% the Free Software Foundation, version 3.
%
% This program is distributed in the hope that it will be useful,
% but WITHOUT ANY WARRANTY; without even the implied warranty of
% MERCHANTABILITY or FITNESS FOR A PARTICULAR PURPOSE.  See the
% GNU General Public License for more details.
%
% You should have received a copy of the GNU General Public License
% along with this program.  If not, see <http://www.gnu.org/licenses/>.
%
% Matthew Skala
% https://northcoastsynthesis.com/
% mskala@northcoastsynthesis.com
%

\chapter{General notes}

This manual documents the MSK~010 Fixed Sine Bank, which is a
control voltage generator module for use in a Eurorack modular synthesizer. 
Each module contains eight analog sine oscillators, and the same parts can
be assembled into three different variants of the module with different
selections of output frequencies, so that with three modules, up to 24
different nominal output frequencies are available.

\section{Specifications}

This module has no inputs.  The impedance of each output is approximately
1k$\Omega$.  The open-circuit output voltage is approximately 10V peak to
peak ($\pm$5V), but this will vary because of component tolerances and is
not a guaranteed specification.  There may also be a transient outside the
normal output voltage range for about half a cycle time during power-up. 
Shorting any output to any fixed voltage at or between the power rails, or
shorting two outputs to each other, is not recommended but should be
harmless to the module; patching the MSK~010's output into some other
module's output should be harmless to the MSK~010, though that is also not
recommended because it is possible the non-MSK~010 module may be harmed.

Nominal output frequencies range from 16mHz to 16Hz, depending on the build
variant selected.  The frequencies will vary a few percent due to component
tolerances, but should have good to very good temperature stability.

This module (assuming a correct build using the recommended components) is
protected against reverse power connection.  It will not function with the
power reversed, but will not cause or suffer any damage.  Some other kinds
of power misconnection may possibly be dangerous to the module or the power
supply.

Most of the current drawn from the power supply by this module goes to
lighting up the LEDs, so the current demand fluctuates with the oscillator
outputs.  In normal operation, the peak current demand is 40mA from each of
the +12V and -12V supplies; the average is a fair bit less, more like 25mA. 
For a few seconds during power-up it may require a few extra milliamps from
the negative supply as the larger timing capacitors charge.  Placing a heavy
load on the outputs (for instance, with so-called passive modules) will
increase the power supply current.

The circuit design can operate on $\pm$15V supplies, for use in non-Eurorack
synthesizer formats, assuming all the components used are rated for the
increased voltage.  There should be no need to change any resistor values or
similar, as long as the nominal $\pm$5V output level is acceptable.  To
achieve a different output level may require some trial and error with the
Zener diode voltages; no other components need change.

\section{Source package}

A ZIP archive containing source code for this document and for the module
itself, including things like machine-readable CAD files, is available from 
the Web site at 
\url{https://northcoastsynthesis.com/}.  Be aware that actually building
from source requires some manual steps; Makefiles for GNU Make are provided,
but you may need to manually generate PDFs from the CAD files for inclusion
in the document, make Gerbers from the PCB design, manually edit the .csv
bill of materials files if you change the bill of materials, and so on.

Recommended software for use with the source code includes:
\begin{itemize}
  \item GNU Make;
  \item \LaTeX\ for document compilation;
  \item LaTeX.mk (Danjean and Legrand, not to be confused with other
    similarly-named \LaTeX-automation tools);
  \item Circuit\_macros (for in-document schematic diagrams);
  \item Kicad (electronic design automation);
  \item Qcad (2D drafting); and
  \item Perl (for the BOM-generating script).
\end{itemize}

The kicad-symbols/ subdirectory contains my customised schematic symbol and PCB
footprint libraries for Kicad.  Kicad doesn't normally keep dependencies
like symbols inside a project directory, so on my system, these files
actually live in a central directory shared by many projects.  As a result,
upon unpacking the ZIP file you may need to do some reconfiguration of the
library paths stored inside the project files, in order to allow the symbols
and footprints to be found.  Also, this directory will probably contain some
extra bonus symbols and footprints not actually used by this project,
because it's a copy of the directory shared with other projects.

The package is covered by the GNU GPL, version 3, a copy of which is
included in the file COPYING.

\section{PCBs and physical design}

The enclosed PCB design is for two boards, each
3.90$''$\linebreak[0]$\times$\linebreak[0]1.50$''$, or
99.06mm\linebreak[0]$\times$\linebreak[0]38.10mm.  They are intended to
mount in a stack parallel to the Eurorack panel, held together with M3
machine screws and male-female hex standoff hardware.  See
Figure~\ref{fig:panel-stack}.  Including 18mm of clearance for the mated
power connector, the module should fit in 43mm of depth measured from the
back of the front panel.

\begin{figure}
{\centering
\begin{tikzpicture}[scale=0.1]
  \path[draw=black,dashed,thick] (24.2,-46.95) rectangle (42.2,-26.95);
  \path[draw=black,fill=black!30!white] (-2.0,-64.25) rectangle (0,64.25);
  \path[draw=black,fill=white] (0,-47.45) rectangle (10,-41.45);
  \path[draw=black,fill=white] (0,47.45) rectangle (10,41.45);
  \path[draw=black,fill=blue!50!white] (10,-49.53) rectangle (11.6,49.53);
  \path[draw=black,fill=white] (11.6,-47.45) rectangle (22.6,-41.45);
  \path[draw=black,fill=white] (11.6,47.45) rectangle (22.6,41.45);
  \path[draw=black,fill=blue!50!white] (22.6,-49.53) rectangle (24.2,49.53);
  \path[draw=black,fill=white] (24.2,-47.45) rectangle (26.2,-41.45);
  \path[draw=black,fill=white] (24.2,47.45) rectangle (26.2,41.45);
  \path[draw=black,fill=black!10!white] (26.2,-45.95) rectangle (28.2,-42.95);
  \path[draw=black,fill=black!10!white] (26.2,45.95) rectangle (28.2,42.95);
  \node at (5.0,53) {\parbox{10mm}{\centering \small 10mm standoff}};
  \node at (17.1,53) {\parbox{11mm}{\centering \small 11mm standoff}};
  \node at (13,64) {\small 2mm front panel};
  \node at (18.4,60.5) {\small 2$\times$ 1.6mm PCBs};
  \draw[>=latex',->,very thick] (10.8,58.7) -- (10.8,50.53);
  \draw[>=latex',->,very thick] (23.4,58.7) -- (23.4,50.53);
  \node at (33.2,-36.95)
    {\parbox{17mm}{\centering \small 18mm clearance for mated power connector}};
  \draw (42.2,-48) -- (42.2,-64);
  \draw[>=latex',<->,thick] (0,-56) -- (42.2,-56);
  \node[fill=white] at (21.5,-56) {\small $\approx$43mm depth};
\end{tikzpicture}\par}
\caption{Assembled module, side view.}\label{fig:panel-stack}
\end{figure}

\section{Component substitutions}

You can change the frequencies of the oscillators by substituting different
resistance and capacitance values, but there are a few things to be aware
of.  First, the two frequency-determining resistors and the two
frequency-determining capacitors in an oscillator unit, must match.  I
specified 5\% capacitors and 1\% resistors.  Tolerances worse than these are
unlikely to work, and 3\% capacitors (or hand-matching wider-tolerance caps
to that spec) would be better.  I specified plastic film capacitors. 
Tolerance and leakage issues pretty much rule out electrolytic capacitors;
and issues of bad DC behaviour rule out silver mica and ferroelectric (X7R
and similar) ceramic capacitors.  Using NP0 ceramic capacitors should be
fine if you can find them in the proper tolerance, but that will usually
mean settling for surface-mount.

If you want very long cycle times, you'll probably be pushing both the
capacitors and the resistors to the limit, and then it's a question of how
much capacitance you can fit into the physical space (maybe you could go as
high as 4.7$\mu$F and still fit the caps on the board) and how high an
impedance you can use with the op amps.  The TL074 chips I use seem to
perform very well at 10M$\Omega$ without the input bias
current throwing off the high-impedance tuned circuit; I don't know how much
higher in impedance they can go, but I wouldn't be surprised by at least
another factor of ten, with careful attention to board cleaning and so on. 
There may be sourcing problems when you try to obtain resistors bigger than
10M$\Omega$ in 1\% tolerance.

Be aware of the start-up issue: it takes at least a half cycle for each
oscillator to settle down into its stable voltage level after power-up, and
if you build one with a cycle time of many minutes, that startup time will
be noticeable.  This is part of the price you pay for having a fully analog
implementation; but compare to other analog sine oscillators, such as
typical resonating synth filters, which typically have \emph{much longer}
startup-time requirements in relation to their cycle times.

Making the frequency adjustable with this circuit would not be easy because
of the need to make two variable components vary at once and track each
other.  The necessary level of matching is at the limits of what an
inexpensive dual-gang potentiometer might be able to achieve, and trying to
make it voltage controlled too would be even worse.  You're probably better
off switching to some completely different design if you must have variable
frequency.

Swapping out the TL074 op amps for some other model is not recommended.  No
relevant specification of this circuit would be improved by a ``better'' op
amp, and op amps designed for audio use may lack the very high input
impedance of the TL074, which is critically necessary for this circuit
(especially in the longer cycle time oscillators).

This circuit's output level is determined by 1N5229B Zener diodes running at
significantly less than the 20mA nominal reverse current for which their
voltage is officially specified, and as a result, their behaviour may be
hard to predict, inconsistent from manufacturer to manufacturer, and so on. 
I specified these, and the 27k$\Omega$ ballast resistors to go with them,
after extensive testing with physical samples.  If you substitute these
diodes, then test the diodes you plan to use carefully to make sure they
give the output level you want.

\section{Use and contact information}

This module design is released under the GNU GPL, version 3, a copy of which
is in the source code package in the file named \texttt{COPYING}.  One
important consequence of the license is that if you distribute the design to
others---for instance, as a built hardware device---then you are obligated
to make the source code available to them at no additional charge, including
any modifications you may have made to the original design.  Source code for
a hardware device includes without limitation such things as the
machine-readable, human-editable CAD files for the circuit boards and
panels.  You also are not permitted to limit others' freedoms to
redistribute the design and make further modifications of their own.

I sell this and other modules, both as fully assembled products and
do-it-yourself kits, from my Web storefront at
\url{http://northcoastsynthesis.com/}.  Your support of my business is what
makes it possible for me to continue releasing module designs for free. 
The latest version of this document and the associated source files can be
found at that Web site.

Email should be sent to\\ \url{mskala@northcoastsynthesis.com}.
