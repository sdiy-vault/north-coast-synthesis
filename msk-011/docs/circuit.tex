% $Id: circuit.tex 5953 2018-03-09 02:14:49Z mskala $

%
% MSK 011 circuit explanation
% Copyright (C) 2018  Matthew Skala
%
% This program is free software: you can redistribute it and/or modify
% it under the terms of the GNU General Public License as published by
% the Free Software Foundation, version 3.
%
% This program is distributed in the hope that it will be useful,
% but WITHOUT ANY WARRANTY; without even the implied warranty of
% MERCHANTABILITY or FITNESS FOR A PARTICULAR PURPOSE.  See the
% GNU General Public License for more details.
%
% You should have received a copy of the GNU General Public License
% along with this program.  If not, see <http://www.gnu.org/licenses/>.
%
% Matthew Skala
% https://northcoastsynthesis.com/
% mskala@northcoastsynthesis.com
%

\chapter{Circuit explanation}

The MSK~011 is an attempt at minimalist design; the simplest
transistor-based mixer circuit that will still be usable.  It is designed
with modern components, which makes some issues easier to address, but the
basic operating principles go all the way back to the earliest transistor
(and, to some extent, vacuum tube) amplifying circuits.

\section{Transistor rules}

Consider a simple NPN bipolar transistor.

{\centering\input{transistor.tex}\par}

The behaviour of all the transistors in the MSK~011 can be understood by two
simple rules.  These are only approximations, and they are specific to NPN
transistors operating in the mode used here.

\emph{Current rule:}  The current flowing into the collector (C) is
approximately equal to the current flowing out of the emitter (E), and there
is approximately no current through the base (B).

\emph{Voltage rule:}  The base is approximately 0.7V more positive than the
emitter, and the collector is whatever voltage it needs to be to obey the
current rule, provided that is above the emitter voltage.

These rules cut in all directions.  A transistor will, to the extent it can,
change the voltages and currents at all of its terminals in order to obey
the rules---changing the voltage at the emitter to be 0.7V less than the
base when the base is driven to a specific voltage by external circuity, but
also changing the voltage at the base to 0.7V above the emitter when the
emitter is driven to a specific voltage, and so on.  If the transistor
cannot obey the rules, then they are not sufficient to understand the
circuit in question and we need to apply other, more detailed, theory.  But
in the MSK~011 during normal operation, these simple rules are a good
description of what is going~on.

\pagebreak

\section{Input buffers}

The MSK~011 contains four input buffers.  Here is the schematic for one of
them; the others are identical.

{\centering\input{buffer.tex}\par}

The signal enters from the right and is attenuated by the potentiometer R2. 
Depending on the position of the knob, somewhere from 0 to 100\%\ of the
input voltage is applied to the base of Q1.  Then, by the voltage rule, the
transistor will drive its emitter to 0.7V less than the voltage applied to
the base.  That will draw some amount of current through the resistor R8,
and the same amount of current is drawn from the power supply at the
transistor collector.

The load at the input jack looks like a fairly constant 100k$\Omega$
resistance; the transistor only places negligible load on whatever is
driving the input.  Meanwhile, we can push or pull a relatively large amount
of current through the output and the transistor will adjust its current
draw from the positive power supply to keep the output at the right voltage,
always 0.7V less than the attenuated input.  If something else tries to pull
the output up, then up to the limit of the roughly 500$\mu$A of current that
is flowing through R8, the transistor will just provide a smaller share of
that current and allow the external circuit to provide the rest, keeping the
voltage across the resistor unchanged.  Similarly, if the external circuit
tries to pull the output down, the transistor (up to its limits) will just
draw more current from the positive supply to satisfy the external circuit
while keeping its emitter 0.7V below its base.

Therefore, what happens at the output of this circuit in terms of current
draw is not seen at the input.  The input is said to be ``buffered'' against
fluctuations in the output.  Such a buffer is useful in this mixer module
because it helps prevent crosstalk between the input channels.

This kind of transistor circuit is called an \emph{emitter follower},
because the voltage at the emitter follows or copies (except for the 0.7V
offset) the voltage applied to the base.  It is also called a
\emph{common collector amplifier}; the collector of the transistor is
connected to the ``common'' connection of the positive power supply. 
Although this circuit has no voltage gain (it adds an offset, but does not
multiply voltage changes in the input by anything other than 1), it serves
as a current amplifier because the current available at the output is
potentially much more than that consumed at the input.

One detail not shown on the above diagram, but visible in the full schematic
on page~\pageref{fig:schematic}, is the input normalling:  channels~1 and~4
have the switching contacts on their jack sockets connected to the power
supplies through 100k$\Omega$ resistors, so that if there is no signal
plugged into these channels, you can turn up their knobs to create a DC
offset (positive for channel~1, negative for channel~4).  Because of the
100k$\Omega$ resistors in series with the 100k$\Omega$ resistance of the
attenuator pot, the maximum offset in either direction is about 6V.

\section{Mixing network}

After buffering, the four input signals go through a passive mixing network
which combines them in equal proportion, attenuating them and shifting them
into a range of voltages near the negative power supply, in order to hit the
desired input range for the output amplifier.  This network is adjustable
with a trimmer pot on the circuit board, to compensate for variations in the
offsets of the transistors throughout the circuit, which are only (under the
voltage rule above) \emph{approximately} 0.7V each.

{\centering\input{mixer.tex}\par}

Note the low-valued resistor R17, just 2.7k$\Omega$ to the negative supply. 
The other resistances are much larger, making this network function as a
fairly strong fixed attenuator; the large signal range at the input
translates into a small range at the input of the next stage.

\section{Output amplifier}

The signal from the mixing network needs to be amplified back up to modular
level, and shifted to be centred on zero.

{\centering\input{outamp.tex}\par}

This is a two-stage \emph{common emitter} amplifier; the emitter of each
transistor is connected (through the low-value resistors R19 and R23) to the
common $-$12V power supply.

As the input signal runs over its small range near the lower power supply
voltage, the transistor Q5 (under the voltage rule) forces its emitter to
follow that voltage, approximately 0.7V down.  This voltage across the
510$\Omega$ resistor R19 induces a proportional current.  Then, under the
current rule, the transistor draws the same amount of current through the
5.9k$\Omega$ resistor R18, which causes the voltage across that resistor to
also vary in proportion.  The collector voltage of Q5 floats to take up the
slack voltage, in accordance with the transistor voltage rule.  But because
R18 is \emph{a bigger resistor}, the same current variation causes a larger
voltage variation across R18.  The circuit exhibits voltage gain:  a small
voltage change on the input results in a larger voltage change on the
output.

As a first approximation we can say that the voltage gain of the first stage
(Q5) is equal to R18 divided by R19; a change of 1V on the transistor base
and therefore on R19 results in 1$/$R19 current change which results in
R18/R19 voltage change on R19.  Really, the gain is a little less than that
because of the current consumed by the resistive coupling network between Q5
and Q6.  A more accurate estimate comes from measuring the ratio between R19
and the parallel combination of R18 with the R20$+$R21 network, which gives
a voltage gain of $-9.86$.  The voltage gain is \emph{negative}.  Lower
voltages on the input give higher voltages on the output (at the collector
of Q5), because lower input voltages mean less current through R19, less
current through R18, and therefore less voltage drop across R18, bringing
the collector of Q5 and low end of R18 closer to the positive supply.

The inverted signal at the collector of Q5 goes through the resistor network
(R20 and R21) to again bring it into the negative-voltage range needed for
input to one of these common-emitter stages, and then Q6, R22, and R23
amplify and invert it again.  The Q6 amplifier stage has a little bit less
voltage gain because of the smaller resistor R22, but otherwise is identical
in operation to the Q5 stage.  And the result is used directly as the
module's DC-coupled output.

The main reason for using \emph{two} amplifier stages here is so that the
module's overall response will be non-inverting.  Another reason is that,
despite the loss in the coupling network between the two stages, doing it
this way allows each single amplifier to work at a lower gain, which makes
it easier to control their gains precisely and reduces risks from things
like feedback.  The 2N5088 transistors recommended here can run comfortably
at much higher gain per stage, but using them that way would make the
response more dependent on variation among individual transistors,
temperature, and so on.

\section{AC coupling network}

The output from Q6 is used directly as the DC-coupled output of the module. 
Having a DC-coupled output is more or less necessary in a Eurorack mixer
because people want to use them for DC control voltages.  However, it is an
unavoidable fact of using such a minimal circuit design that there is some
dependence on transistor behaviour, and in particular, on temperature.  That
offset between base and emitter that I called ``approximately 0.7V''
actually depends a lot on temperature, and a little on the individual
transistors.  One consequence is that the exact ranges of signal voltages
will shift with temperature; zero voltage on all inputs to the mixer may not
give zero volts at the DC-coupled output.  The offset trimmer can be
adjusted to eliminate the offset at some typical temperature, but as the
temperature changes, it will fall out of trim again.

So for use with audio signals, the MSK~011 also provides an AC-coupled
output, using a simple RC high-pass network.

{\centering\input{coupler.tex}\par}

Applying the formula for RC cutoff frequency with a 4.7$\mu$F capacitor and
330k$\Omega$ resistor suggests that this will roll off signals below about
0.10Hz.  Really, the input impedance of whatever it's plugged into,
appearing in parallel with R24, will have the effect of reducing the
resistance and increasing the corner frequency; but even driving a
low-impedance ``passive'' module it should pass all reasonable audio
frequencies.  The real point of R24 is not so much to create a controlled
roll-off frequency as to drain away any static charges that might from time
to time appear on the output pin; and the point of the network as a whole is
to block DC.
