% $Id: general.tex 5963 2018-03-22 17:52:01Z mskala $

%
% MSK 011 manual introduction
% Copyright (C) 2018  Matthew Skala
%
% This program is free software: you can redistribute it and/or modify
% it under the terms of the GNU General Public License as published by
% the Free Software Foundation, version 3.
%
% This program is distributed in the hope that it will be useful,
% but WITHOUT ANY WARRANTY; without even the implied warranty of
% MERCHANTABILITY or FITNESS FOR A PARTICULAR PURPOSE.  See the
% GNU General Public License for more details.
%
% You should have received a copy of the GNU General Public License
% along with this program.  If not, see <http://www.gnu.org/licenses/>.
%
% Matthew Skala
% https://northcoastsynthesis.com/
% mskala@northcoastsynthesis.com
%

\chapter{General notes}

This manual documents the MSK~011 Transistor Mixer, which is a
utility mixer module for use with both audio and control voltages
in a Eurorack modular synthesizer. 
It is a minimalist design using only six transistors and no integrated
circuits.

\section{Specifications}

This module is designed to mount in 6HP of rack space in a standard 3U
Eurorack case.

The module's maximum current requirement in ordinary use is 9mA on the
$+$12V supply and 14mA on the $-$12V supply.
Unusual loads on the outputs, including directly-connected headphones or
speakers and so-called ``passive'' modules, may cause the MSK~011 to draw
more than this amount of current.
It does not require $+$5V power.

The input impedance is nominally 100k$\Omega$ for all inputs, varying a
little with control knob positions.  The output impedance is nominally
5k$\Omega$.  Separate AC and DC coupled outputs are provided.  Shorting any
input or output to any fixed voltage at or between the power rails should be
harmless to the module; patching the MSK~011's output into the output of
some other module on the same power system should be harmless to the
MSK~011, though doing that is not recommended because it is possible the
other module may be harmed.

Voltage gain for each channel can be adjusted between zero and a little more
than unity.  The mixer is designed to handle input and output signals over a
range of at least $\pm$8V; it is harmless to the module to drive the inputs
all the way to the power rails, but such signals will clip.

The very simple Class~A transistor amplifiers used in this module are
subject to temperature-dependent DC offset.  There is a trimmer provided on
the circuit board for overall adjustment, and the DC level can be fine-tuned
in many patches with the normalled offset controls on the front panel;
however, in patches where it is important to avoid any DC offset at all, it
may be preferable to use the AC-coupled output.  Since it also may be
difficult to adjust the gain controls to exactly unity, this mixer may not
be the best choice for sensitive pitch-CV applications if one wants to keep
the pitches exactly in tune; the North Coast Synthesis Ltd.\ MSK~008 Dual VC
Octave Switch, with its precision adders based on op amp chips, may
be a better choice for exact pitch control.  The Transistor Mixer is
intended, instead, for users who want to embrace the organic variability of
classic discrete-transistor circuits.

This module (assuming a correct build using the recommended components) is
protected against reverse power connection.  It will not function with the
power reversed, but will not suffer or cause any damage.  Some other kinds
of misconnection may possibly be dangerous to the module or the power
supply.

\section{Controls and connections}

Here's a summary of the items on the front panel of the module.

\begin{description}
  \item[inputs] one for each of the four channels.  Note the
  top one is normalled to a positive voltage (equivalent to roughly $+$6V)
  and the bottom to a negative voltage (about $-$6V).

  \item[gain controls] one knob for each channel.  These are non-inverting,
  logarithmic controls.  For the top and bottom channels, if there is nothing
  plugged into the corresponding input jack, the knob controls the amount
  of DC offset.  \emph{You should set these knobs to zero, that is, fully
  counterclockwise, if you are not using them as inputs and don't want a DC
  offset.}

  \item[DC output] the result of mixing the four inputs.
  
  \item[AC output] same as the DC output, but with a simple filter to block
  DC.  The cutoff frequency depends on the load you apply, but would
  typically be less than 1Hz, so this output is suitable even for control
  voltages if they are reasonably fast-moving.
\end{description}

Also note, on the circuit board behind the panel, the offset trimmer,
labelled ``R7 100k offset null.'' This will have been pre-set when the
module was built, but it can be adjusted again later if necessary. 
Adjustment range for this trimmer is approximately $\pm$2V.

\section{Source package}

A ZIP archive containing source code for this document and for the module
itself, including things like machine-readable CAD files, is available from 
the Web site at 
\url{https://northcoastsynthesis.com/}.  Be aware that actually building
from source requires some manual steps; Makefiles for GNU Make are provided,
but you may need to manually generate PDFs from the CAD files for inclusion
in the document, make Gerbers from the PCB design, manually edit the .csv
bill of materials files if you change the bill of materials, and so on.

Recommended software for use with the source code includes:
\begin{itemize}
  \item GNU Make;
  \item \LaTeX\ for document compilation;
  \item LaTeX.mk (Danjean and Legrand, not to be confused with other
    similarly-named \LaTeX-automation tools);
  \item Circuit\_macros (for in-document schematic diagrams);
  \item Kicad (electronic design automation);
  \item Qcad (2D drafting); and
  \item Perl (for the BOM-generating script).
\end{itemize}

The kicad-symbols/ subdirectory contains my customised schematic symbol and
PCB footprint libraries for Kicad.  Kicad doesn't consistently keep
dependencies like symbols inside a project directory, so on my system, these
files actually live in a central directory shared by many projects.  As a
result, upon unpacking the ZIP file you may need to do some reconfiguration
of the library paths stored inside the project files, in order to allow the
symbols and footprints to be found.  Also, this directory will probably
contain some extra bonus symbols and footprints not actually used by this
project, because it's a copy of the directory shared with other projects.

The package is covered by the GNU GPL, version 3, a copy of which is
included in the file COPYING.

\section{PCBs and physical design}

This module is built on a single PCB
4.20$''$\linebreak[0]$\times$\linebreak[0]1.30$''$, or
106.68mm\linebreak[0]$\times$\linebreak[0]33.02mm, which mounts
perpendicular to the front panel.  With about another 1mm
of gap between the PCB and panel, the total depth requirement is 34mm.

\section{Component substitutions}

This circuit should work with almost any NPN transistors.  We ship
2N5088 transistors in our kits and assembled modules, and these are
high-quality silicon amplifier transistors with gain in the range of a few
hundred; to be honest, they are higher specification than this circuit
needs.  Any transistors used should be able to handle the full power supply
voltage (24V) between collector and emitter, and at least about 10mA of
current.  Low-gain transistors (less than about 30 or 40) may have some
impact on the accuracy and available gain of the mixing, but if you are
using such transistors, you probably intend for them to have an effect.  Any
other effects of transistor substitution are likely to be very subtle.  It
is not necessary to match transistors, though if two channel buffers have
very different transistors in them, then those channels may end up sounding
different from each other.

When substituting NPN transistors, connect the collector, base, and emitter
to the pads marked ``C,'' ``B,'' and ``E'' on the board according to the
pinout of the specific transistors you are using, even if that does not
match the flat side of the TO-92 outline in the silkscreen (which refers to
the 2N5088 pinout).

Fairchild/ON Semi, formerly the main manufacturer of 2N5088 transistors,
announced their discontinuation while this product was in development. 
North Coast made a lifetime buy and we should be able to continue supplying
kits and modules with 2N5088 transistors for the foreseeable future.  As of
this writing, Central Semiconductor still makes 2N5088 transistors.  But
most other small NPN amplifier transistors should also work well in this
circuit and could be substituted; the transistor selection is not critical.

On substituting other components:  we suggest 1\%\ metal film fixed
resistors throughout, because these days, they are cheap enough to use
indiscriminately; however, it would probably still work fine with 5\%\ 
resistors in most if not all locations.  None of the resistors are
particularly critical with respect to power handling and we supply different
power ratings of resistors for different values according to what we can
source easily.  The output AC-coupling capacitor is designed as 4.7$\mu$F
film for best performance, but it should be safe to substitute one a little
smaller, or a non-polar electrolytic.

\section{PNP transistor modification}

\emph{This modification is only appropriate for advanced builders.}

Because the circuit is so simple, it is possible to reverse the polarities
of the components and build it with PNP instead of NPN transistors,
without any changes to the board layout!  This might be useful if one wants
to use, for instance, vintage germanium transistors, which tend to be PNP
type.  However, it means disobeying the instructions on the board
silkscreen, and attaching the power backwards; so it is not for the faint of
heart.

To build using PNP transistors:

\begin{itemize}
  \item Install PNP transistors on the board according to the silkscreen
  markings:  collectors to ``C,'' bases to ``B,'' and emitters to ``E.''
  Disregard the TO-92 outlines on the board, which may or may not match your
  transistors depending on their pinout.
  \item Install the diodes D1 and D2 \emph{against} the silkscreen markings,
  with their cathodes in the circular pads pointing away from the silkscreen
  symbol's stripe.
  \item Install the electrolytic capacitors C2 and C3 \emph{against} the
  silkscreen markings, with their positive terminals in the circular pads
  marked ``$-$.''
  \item Attach the power cable \emph{upside down} (!), with its striped edge
  denoting the negative rail at the top of the connector on the module PCB,
  against the ``$-$12V'' marking on the silkscreen.
\end{itemize}

\section{Modification for $\pm$15V power}

To change this circuit for $\pm$15V power:

\begin{itemize}
  \item Make sure all components (taking particular note of transistors
  and electrolytic capacitors) are rated for 30V.
  \item Change R17 to 2.4k$\Omega$ (was 2.7k$\Omega$).
  \item Change R18 to 6.2k$\Omega$ (was 5.9k$\Omega$).
  \item Substitute an appropriate power connection for the Eurorack power
  header.
\end{itemize}

The resistor changes were determined by simulation; I have not built a
$\pm$15V prototype for testing.

\section{Use and contact information}

This module design is released under the GNU GPL, version 3, a copy of which
is in the source code package in the file named \texttt{COPYING}.  One
important consequence of the license is that if you distribute the design to
others---for instance, as a built hardware device---then you are obligated
to make the source code available to them at no additional charge, including
any modifications you may have made to the original design.  Source code for
a hardware device includes without limitation such things as the
machine-readable, human-editable CAD files for the circuit boards and
panels.  You also are not permitted to limit others' freedoms to
redistribute the design and make further modifications of their own.

I sell this and other modules, both as fully assembled products and
do-it-yourself kits, from my Web storefront at
\url{http://northcoastsynthesis.com/}.  Your support of my business is what
makes it possible for me to continue releasing module designs for free. 
Even if you only use the free plans and cannot buy the commercial products I
sell, any assistance you can offer to increasing the profile of North Coast
would be much appreciated.  For instance, you might post photos of your
completed DIY build on your social media.  The latest version of this
document and the associated source files can be found at the North Coast Web
site.

Email should be sent to\\ \url{mskala@northcoastsynthesis.com}.
