% $Id: adjustment.tex 8331 2020-11-19 05:17:40Z mskala $

%
% MSK 011 testing and adjustment instructions
% Copyright (C) 2018  Matthew Skala
%
% This program is free software: you can redistribute it and/or modify
% it under the terms of the GNU General Public License as published by
% the Free Software Foundation, version 3.
%
% This program is distributed in the hope that it will be useful,
% but WITHOUT ANY WARRANTY; without even the implied warranty of
% MERCHANTABILITY or FITNESS FOR A PARTICULAR PURPOSE.  See the
% GNU General Public License for more details.
%
% You should have received a copy of the GNU General Public License
% along with this program.  If not, see <http://www.gnu.org/licenses/>.
%
% Matthew Skala
% https://northcoastsynthesis.com/
% mskala@northcoastsynthesis.com
%
\chapter{Adjustment and testing}

The MSK~011 is designed to work with very little adjustment.  As a result of
its minimalist transistor-based design, there is unavoidably some
temperature-sensitive offset in the DC-coupled output.  A trimmer is
provided to help null this out, but because of the temperature sensitivity
the offset will probably never remain at zero in practical use.  For audio
applications it is preferable to use the AC-coupled output.

This adjustment procedure requires the finished module, a smallish cross-tip
screwdriver, a suitable power supply, and a multimeter.

\section{Short-circuit test}

With no power applied to the module, check for short circuits between the
three power connections on the Eurorack power connector.  The two
pins at the bottom, marked with an arrow on the circuit board, are for -12V. 
The two at the other end are for +12V; and the remaining six pins in the
middle are all ground pins.  Check between each pairing of these three
voltages, in both directions (six tests in all).  Ideally, you should use a
multimeter's ``diode test'' range for this; if yours has no such range, use
a low resistance-measuring setting. It should read infinite in the reverse
direction (positive lead to $-$12V and negative lead to each of the other
two, as well as positive lead to ground and negative to $+$12V) and greater
than 1V or 1k$\Omega$ in the forward direction (reverse those three
tests).  If any of these six measurements is less than 1k$\Omega$ or 1V,
then something is wrong with the build, most likely a blob of solder
shorting between two connections, and you should troubleshoot that before
applying power.

\emph{Optional}:  Although we test all cables before we sell them, bad
cables have been known to exist, so it might be worth plugging the Eurorack
power cable into the module and repeating these continuity tests across the
cable's corresponding contacts (using bits of narrow-guage wire to get into
the contacts on the cable if necessary, or probing the pins of the power
connector on the back side of the circuit board) to make sure there are no
shorts in the cable crimping.  Doing this test \emph{with the cable
connected to the module} makes it easier to avoid mistakes, because the
module itself will short together all wires that carry equal potential,
making it easier to be sure of testing the relevant adjacent-wire pairs in
the cable.

Plug the module into a Eurorack power supply and make sure
neither it nor the power supply emits smoke, overheats, makes any unusual
noises, or smells bad.  If any of those things happen, turn off the power
immediately, and troubleshoot the problem before proceeding.

\emph{Optional}: Plug the module into a Eurorack power supply
\emph{backwards}, see that nothing bad happens, and congratulate yourself on
having assembled the reverse-connection protective circuit properly. 
Reconnect it right way round before proceeding to the next step.

\section{Output offset adjustment}

Turn all knobs fully counterclockwise.  Apply power to the module with
nothing plugged into the input, and measure the DC voltage of the DC-coupled
output.  Adjust R7 at the top of the board, labelled ``offset null,'' to
bring the output as close to 0V as is reasonably possible.  It is unlikely
that you will be able to get it to stay at exactly zero.
