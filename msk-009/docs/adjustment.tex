% $Id: adjustment.tex 8046 2020-08-15 16:08:52Z mskala $

%
% MSK 009 testing and adjustment instructions
% Copyright (C) 2018  Matthew Skala
%
% This program is free software: you can redistribute it and/or modify
% it under the terms of the GNU General Public License as published by
% the Free Software Foundation, version 3.
%
% This program is distributed in the hope that it will be useful,
% but WITHOUT ANY WARRANTY; without even the implied warranty of
% MERCHANTABILITY or FITNESS FOR A PARTICULAR PURPOSE.  See the
% GNU General Public License for more details.
%
% You should have received a copy of the GNU General Public License
% along with this program.  If not, see <http://www.gnu.org/licenses/>.
%
% Matthew Skala
% https://northcoastsynthesis.com/
% mskala@northcoastsynthesis.com
%

\chapter{Adjustment and testing}

It is inevitable that a circuit of this kind will have a certain amount of
DC offset and control feedthrough, and since the MSK~009 was never meant to
be a ``well-behaved'' filter anyway, it doesn't seem appropriate to attempt
to eliminate it entirely.  Should you wish a completely DC-free signal, I
recommend running the filter output through a North Coast Synthesis Ltd.\
MSK~011 Transistor Mixer with its AC-coupled output.  However, the MSK~009
does have two trimmers for minimizing offset, and this chapter describes how
to properly adjust them, along with some tips on troubleshooting should
everything not work perfectly at initial power-up.

You will need a multimeter and a Eurorack power supply.  An oscilloscope may
be useful for troubleshooting but should not be needed for basic adjustment.

\section{Short-circuit test}

With no power applied to the module, check for short circuits between the
three power connections on the Board~2 Eurorack power connector.  The two
pins at the bottom, marked with white on the circuit board, are for -12V. 
The two at the other end are for +12V; and the remaining six pins in the
middle are all ground pins.  Check between each pairing of these three
voltages, in both directions (six tests in all).  Ideally, you should use a
multimeter's ``diode test'' range for this; if yours has no such range, use
a low resistance-measuring setting. It should read infinite in the reverse
direction (positive lead to $-$12V and negative lead to each of the other
two, as well as positive lead to ground and negative to $+$12V) and greater
than 1V or 1k$\Omega$ in the forward direction (reverse those three
tests).  If any of these six measurements is less than 1k$\Omega$ or 1V,
then something is wrong with the build, most likely a blob of solder
shorting between two connections, and you should troubleshoot that before
applying power.

\emph{Optional}:  Although we test all cables before we sell them, bad
cables have been known to exist, so it might be worth plugging the Eurorack
power cable into the module and repeating these continuity tests across the
cable's corresponding contacts (using bits of narrow-gauge wire to get into
the contacts on the cable if necessary) to make sure there are no shorts in
the cable crimping.  Doing this \emph{with the cable connected to the
module} makes it easier to avoid mistakes, because the module itself will
short together all wires that carry equal potential, making it easier to be
sure of testing the relevant adjacent-wire pairs in the cable.

Plug the module into a Eurorack power supply and make sure
neither it nor the power supply emits smoke, overheats, makes any unusual
noises, or smells bad.  If any of those things happen, turn off the power
immediately, and troubleshoot the problem before proceeding.

\emph{Optional}: Plug the module into a Eurorack power supply
\emph{backwards}, see that nothing bad happens, and congratulate yourself on
having assembled the reverse-connection protective circuit properly. 
Reconnect it right way round before proceeding to the next step.

\section{Output offset adjustment}

Identify the two trimmers on the back of the module (Board~2).  When you are
looking at the back of the module, R20 is at upper left and R30 is at upper
right.  There is some interaction between them, but R20 primarily affects
the offsets on the HP and LP outputs, and R30 the offsets on the BP and LP
outputs.  The labels on early versions of the circuit board do not mention
R20's effect on HP.  Bearing in mind that with two trimmers we cannot
perfectly control all three offsets, and offset on LP is more important than
on BP because of the intended application of the module as an envelope
follower, the recommended procedure is to use R20 to minimize the offset on
the HP output, then R30 to minimize the offset on the LP output, then repeat
to deal with their interaction.  The offset on the BP output is not directly
controlled but should be small once the other two are minimized.

Disconnect any signal cables from the filter.  Turn the resonance and CV
attenuator controls fully counterclockwise and the tuning control not quite
fully counterclockwise (knob pointer around eight o'clock).  Apply
power.

Measure the DC voltage on the HP output and adjust R20 to bring this voltage
as near zero as you reasonably can.  Within $\pm$0.10V is good.

Measure the DC voltage on the LP output and adjust R30 to bring this voltage
near zero also.  Again, within $\pm$0.10V is good.

Repeat the last two steps a second time:  measure the HP voltage and adjust
R20 to null it, and then measure the LP voltage and adjust R30 to null it.
This completes the adjustment.

If you find that you cannot zero the voltages because the trimmers have
insufficient adjustment range, try repeating the procedure with the tuning
knob turned a little further counterclockwise.  In general, the module will
be easier to adjust the lower the tuning setting, but the behaviour of the
module will be less predictable at high frequencies if adjusted at lower
frequencies.  On the other hand, if the knob is turned too far clockwise
during the adjustment, then the module may tend to couple noise from the
power supply into its outputs when set to very low frequencies.  My current
recommendation is to turn the knob to about eight o'clock as the best
compromise.

If you cannot bring the voltages close to zero even with changes to the
tuning, it may indicate a problem with the build, but one more thing to try
is swapping the two TL074 chips.  Occasionally, if the input offset on these
chips is near the boundary of its acceptable specification, it can make
adjustment difficult, and since the two chips will have different random
input offsets, swapping them can resolve the issue.

\section{Troubleshooting}

It would require several books to convey all the skills and knowledge useful
in troubleshooting even a simple electronic circuit like this one, but here
are some possible symptoms and some suggestions on diagnosis and treatment.

No response from the module at all:  no signal on the outputs even in
resonance mode, or with strong signals on the inputs.  Most likely a power
problem, such as a power cable plugged in wrong or a short circuit.  It
might even be a problem in the power supply and not the module itself.

No oscillation with resonance turned up to maximum:  it is normal that the
module may not oscillate over the entire range of the tuning knob, and for
the amplitude to vary a bit over the range where it does oscillate, but it
should cover most if not all of the range.  Be sure that the output offset
adjustment, described above, has been done; excessive DC offset may inhibit
oscillation.  However, if you want absolutely the most oscillation possible,
you may find that tweaking the trimmers a little bit away from zero-offset
may increase the amplitude a little bit.  This kind of experimentation is
best done with an oscilloscope on the output as you adjust the trimmers. 
It's easy to gain an intuitive feeling for the oscillation/offset tradeoff
when working live on the oscilloscope screen.

Module responds, but not as expected:  first attempt to localize the
problem.  Do both inputs work?  Do all three outputs work?  If you can find
two tests where one works and the other doesen't, then the problem is
probably related to whatever part of the module was involved in the failing
test and was not involved in both tests.  Unfortunately, the multiple
feedback loops in the state-variable filter mean that a large part of the
module is involved in pretty much every function; but the general principle
of narrowing things down remains at the heart of troubleshooting.

A specific problem noted on one prototype during testing:  it's possible for
contamination to be washed into the board-to-board connector during
cleaning, and this can make the connection flaky, with possible
hard-to-debug results including DC offsets that won't go to zero even with
the trimmers at the ends of their ranges.  If you've washed your boards with
solvent, and especially if you haven't been careful to keep it out of the
female connector, then try gently unplugging and re-plugging the
board-to-bard connection a few times.

Shorts on board-to-board connector:  in a couple of assembled builds I found
a problem where pins of the board-to-board connector were shorting to the
ground plane.  I think the solder mask may have been slightly misaligned
even though it passed quality control, allowing solder to short to an
exposed bit of ground plane; or, possibly, too much soldering heat
had burned through the mask to create a hole.  Anyway, the main symptom was
one or all of the output jacks giving DC voltages that didn't change with
input CV and front-panel controls.  To further diagnose it, check for
continuity from each pin of the connctor to ground.  Three of the pins are
supposed to be connected to ground, but if more seem to be, it's
probably a short.  Doing the test with the boards separated will allow
determining which board has the short.  On my boards,
once I knew the issue I was able to look closely and see the solder bridging
from pin to ground plane nearby, and cutting the extra away with a sharp
knife was enough to fix the problem.

Bad amplitude (including \emph{no} amplitude) or DC offset in oscillation,
combined with funny response to large-amplitude signals on the rectifier
input: this combination of issues suggests that one of the Zener diodes was
swapped with the nearly identical-looking switching diode.  If one of these
issues occurs without the other, it suggests that maybe all the correct
diodes were installed, but at least one of them is backwards.

General quality issues:  many problems can be diagnosed just by looking
closely at your work, preferably with at least one night's sleep between
when you assembled the module and when you examine it.  Look for bad solder
joints that fail to connect; solder bridges between nearby connections
(especially on the discrete transistors in the exponential converter);
components missing; components exchanged (especially resistors with similar
colour codes, such as 1k$\Omega$ swapped with 10k$\Omega$); polarized
components such as diodes mounted backwards; and so on.

General tips for debugging DIP ICs:  make sure for, for each IC, that
\begin{itemize}
  \item it really is the type of IC it's supposed to be, not something else
    (beware of cheap ICs you buy from Chinese sellers on eBay, though the
    ones in this project are common enough in more reputable channels that
    you probably wouldn't have attempted that anyway);
  \item it is plugged in snugly;
  \item all the legs of the chip go nicely into the corresponding holes in
    the socket, with none bent outside or folded up under the chip;
  \item it is plugged in \emph{at all} (forgetting to do so is a surprisingly
    common mistake!);
  \item it is plugged in the right way around, with the Pin 1
    indentation or notch matching the clues on the
    board (if this is wrong, the chip is probably destroyed and will need to
    be replaced);
  \item there are no solder bridges on the chip socket, unsoldered pins,
    debris clogging the socket holes, or similar;
  \item its decoupling capacitors (the small ceramic ones) are installed and
    there is nothing wrong with their solder joints; and
  \item in the case of the two TL074 chips, try swapping them and see if that
    makes any difference.
\end{itemize}
