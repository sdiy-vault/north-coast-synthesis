% $Id: general.tex 6473 2018-11-30 15:28:26Z mskala $

%
% MSK 009 general notes
% Copyright (C) 2018  Matthew Skala
%
% This program is free software: you can redistribute it and/or modify
% it under the terms of the GNU General Public License as published by
% the Free Software Foundation, version 3.
%
% This program is distributed in the hope that it will be useful,
% but WITHOUT ANY WARRANTY; without even the implied warranty of
% MERCHANTABILITY or FITNESS FOR A PARTICULAR PURPOSE.  See the
% GNU General Public License for more details.
%
% You should have received a copy of the GNU General Public License
% along with this program.  If not, see <http://www.gnu.org/licenses/>.
%
% Matthew Skala
% https://northcoastsynthesis.com/
% mskala@northcoastsynthesis.com
%

\chapter{General notes}

This manual documents the MSK~009 Coiler Multi-Mode Filter and Rectifier,
which is a module for use in a Eurorack modular synthesizer.  The module
contains a voltage-controlled two-pole state variable filter
implemented using inductors (coils, hence the name) as the main
energy-storing components in the integrators.  It also uses capacitors,
which have their main effect at bass frequencies, with the filter's
behaviour shading from capacitor-based to inductor-based between about 500Hz
to 2kHz.  There are separate outputs for high-pass, band-pass, and low-pass
transfer functions, and two audio inputs, one of which goes through a
full-wave rectifier before being fed into the filter.

\section{Controls and connections}

The front panel of the module is shown in Figure~\ref{fig:panel-mockup}.

\begin{figure}
{\centering\setlength{\fboxsep}{0pt}\setlength{\fboxrule}{0.6pt}%
\fbox{\includegraphics{panel-mockup.pdf}}\par}
\caption{Module front panel.}\label{fig:panel-mockup}
\end{figure}

\subsection{TUNE knob}

This knob adjusts the overall frequency of the filter, for all three
outputs.  Its setting is added to the control voltage inputs.  It should
cover the entire usable range of the filter, with a little bit of excess at
the low end to allow for ``closing'' the filter more completely when using
voltage control in a low-pass gate patch.

\subsection{res knob}

This sets the ``resonance'' of the filter by attenuating one of the feedback
paths.  Counterclockwise for a flatter response curve, clockwise for a
sharper peak.  Near the clockwise maximum resonance, the filter will
oscillate.  Because of the way the inductors respond differently to phase at
different points on the audio spectrum, this knob's effect interacts with
the current frequency setting; the height of the resonance peak and the
point at which oscillation begins will change with the cutoff frequency,
creating a wide range of varying-timbre effects.

\subsection{att knob}

This is an attenuator for the CV2 input---the lower of the two CV inputs, to
which this knob is joined by the zigzag resistor line in the panel art,
symbolizing attenuation.  With the knob fully clockwise the sensitivity of
this input is approximately 1V/octave, the same as the unattenuated input. 
At lower settings, the CV2 input is less sensitive.

\subsection{IN inputs}

Audio inputs to the filter.  The upper input, marked with a diode symbol and
the notation \textbf{fw rect}, is subjected to full-wave rectification
(positive and negative voltages translated into their absolute values)
before being applied to the filter.  The lower input is a direct connection. 
Both inputs may be used at once; their effects are summed.

The rectified input includes a phase inverter (both positive and negative
voltages are translated into \emph{negative}) to cancel out the
naturally-occuring phase inversion between the
input and LP output in this filter topology.  As a result, if you feed an
audio signal into the rectified input with the filter cutoff significantly
below the frequency of the audio, it will be rectified and filtered into a
\emph{positive} voltage tracking the overall amplitude of the input signal. 
This way the module can be used as an envelope follower.

The inputs can accept any voltages between the module's power rails ($-$12V
to $+$12V) without damage.  The module my be overdriven, creating
significant distortion, with inputs beyond about $\pm$5V.

\subsection{CV inputs}

Exponential control voltages for filter cutoff frequency.  The upper socket
(CV1) has a nominal sensitivity of 1V/octave; but the tracking of this
filter is not meant to be very accurate, and it cannot be made highly
accurate because of the somewhat unpredictable properties of the inductors. 
Tracking will differ in different parts of the audio spectrum.  The
CV-processing circuit is partially temperature-compensated, with
zeroth-order ``offset'' compensation but not first-order ``tracking''
compensation.  The lower socket's (CV2) sensitivity is adjustable with the
att knob, to a maximum of the same sensitivity as CV1.  The CV1 input,
attenuated CV2 input, and TUNE knob setting are all summed to produce the
control value for the filter core.

Both CV inputs can accept voltages anywhere between the module's power rails
($-$12V to $+$12V) without damage.  Which voltages are useful depends on the
patch and the setting of the TUNE knob, but a typical user might aim for 0V
to 5V.

\subsection{HP, BP, and LP outputs}

These are the three outputs of the filter core:  high-pass, band-pass, and
low-pass.  Because this is a two-pole filter, the asymptotic slopes of the
response curves are 12dB/octave for the high-pass and low-pass, and
6dB/octave on each of two slopes for the band-pass.

All three outputs are active simultaneously, driven by the combined input
from the two IN jack sockets.  The phase relationships among the three
outputs will change with frequency as the filter shifts between using its
capacitors and its inductors; that means mixing outputs to produce other
filter functions (such as notch filtering) may produce results that sound
good, but they are unlikely to be strong on measures like stopband
attenuation.

Voltage levels on the audio outputs will normally be similar to the voltage
levels on the inputs, with the maximum possible voltage limited by possible
clipping in the op amp chips at around $\pm$10V.  Output level at maximum
oscillation will be about $\pm$5V.  At the lowest resonance setting, the BP
output will be a little quieter than the other two, an effect which tends to
disappear at higher resonance.

\section{Specifications}

The nominal input impedance is 100k$\Omega$ for all inputs except the
rectifier input, which varies between 50k$\Omega$ and 100k$\Omega$.  Nominal
output impedance is 1k$\Omega$ for all outputs.

Any voltage between the power supply rails (nominally $\pm$12V) is safe for
the module, on any input; output voltages are limited by the capabilities of
the op amps to about $\pm$10V and will clip if the inputs are driven
sufficiently hard.  Distortion resulting from limiting in internal feedback
paths may show up before the outputs actually clip.

The circuit is DC-coupled throughout; as a result, it can operate at very
low frequencies, but small DC offsets may appear on the outputs.  Trimmers
are provided for minimizing offset effects.

Briefly shorting any input or output to any fixed voltage at or between the
power rails, or shorting two to each other, should be harmless to the
module.  Patching the MSK~009's output into some other module's output
should be harmless to the MSK~009, but doing that is not recommended because
it is possible the non-MSK~009 module may be harmed.

This module (assuming a correct build using the recommended components) is
protected against reverse power connection.  It will not function with the
power reversed, but will not cause or suffer any damage.  Some other kinds
of power misconnection may possibly be dangerous to the module or the power
supply.

In normal operation the maximum current demand of this module is 25mA from
the +12V supply and 25mA from the -12V supply.  Placing an unusually heavy
load on the outputs (for instance, with so-called passive modules) can
increase the power supply current beyond those levels.

\section{Voltage modification}

This circuit is designed for $\pm$12V power.  It should work acceptably on
$\pm$15V power without modification, assuming all components are rated for
the increased voltage, but some current levels and adjustment ranges are
related to the power supplies and so just applying $\pm$15V power with no
changes may not give optimal results.  In particular, I would expect doing
that to create ``dead zones'' at the ends of the tuning control range.  My
suggestion if using $\pm$15V power would be to increase all four
220k$\Omega$ resistors (R5, R8, R19, and R29) to 270k$\Omega$; that should
restore the intended current levels and adjustment ranges.

I have calculated but not tested these resistor changes.

\section{Source package}

A ZIP archive containing source code for this document and for the module
itself, including things like machine-readable CAD files, is available from 
the Web site at 
\url{https://northcoastsynthesis.com/}.  Be aware that actually building
from source requires some manual steps; Makefiles for GNU Make are provided,
but you may need to manually generate PDFs from the CAD files for inclusion
in the document, make Gerbers from the PCB design, manually edit the .csv
bill of materials files if you change the bill of materials, and so on.

Recommended software for use with the source code includes:
\begin{itemize}
  \item GNU Make;
  \item \LaTeX\ for document compilation;
  \item LaTeX.mk (Danjean and Legrand, not to be confused with other
    similarly-named \LaTeX-automation tools);
  \item Circuit\_macros (for in-document schematic diagrams);
  \item Kicad (electronic design automation);
  \item Qcad (2D drafting); and
  \item Perl (for the BOM-generating script).
\end{itemize}

The kicad-symbols/ subdirectory contains my customised schematic symbol and
PCB footprint libraries for Kicad.  Kicad doesn't normally keep dependencies
like symbols inside a project directory, so on my system, these files
actually live in a central directory shared by many projects.  As a result,
upon unpacking the ZIP file you may need to do some reconfiguration of the
library paths stored inside the project files, in order to allow the symbols
and footprints to be found.  Also, this directory will probably contain some
extra bonus symbols and footprints not actually used by this project,
because it's a copy of the directory shared with other projects.

The package is covered by the GNU GPL, version 3, a copy of which is
included in the file COPYING.

\section{PCBs and physical design}

The enclosed PCB design is for two boards.  Board 1 is
3.90$''$\linebreak[0]$\times$\linebreak[0]1.50$''$ or
99.06mm\linebreak[0]$\times$\linebreak[0]38.10mm.  Board 2 is a little
shorter,
3.40$''$\linebreak[0]$\times$\linebreak[0]1.50$''$ or
86.36mm\linebreak[0]$\times$\linebreak[0]38.10mm.
The two boards are intended to
mount in a stack parallel to the Eurorack panel, held together with M3
machine screws and male-female hex standoff hardware.  See
Figure~\ref{fig:panel-stack}.  Including 18mm of clearance for the mated
power connector, the module should fit in 46mm of depth measured from the
back of the front panel.

\begin{figure}
{\centering
\begin{tikzpicture}[scale=0.1]
  \path[draw=black,dashed,thick] (27.2,-44.14) rectangle (45.2,-24.14);
  \path[draw=black,fill=black!30!white] (-2.0,-64.25) rectangle (0,64.25);
  \path[draw=black,fill=white] (0,-30.48) rectangle (13,-24.48);
  \path[draw=black,fill=white] (0,30.48) rectangle (13,24.48);
  \path[draw=black,fill=blue!50!white] (13,-49.53) rectangle (14.6,49.53);
  \path[draw=black,fill=white] (14.6,-30.48) rectangle (25.6,-24.48);
  \path[draw=black,fill=white] (14.6,30.48) rectangle (25.6,24.48);
  \path[draw=black,fill=blue!50!white] (25.6,-45.72) rectangle (27.2,40.64);
  \path[draw=black,fill=white] (27.2,-30.48) rectangle (29.2,-24.48);
  \path[draw=black,fill=white] (27.2,30.48) rectangle (29.2,24.48);
  \path[draw=black,fill=black!10!white] (29.2,-28.98) rectangle (31.2,-25.98);
  \path[draw=black,fill=black!10!white] (29.2,28.98) rectangle (31.2,25.98);
  \node at (6.5,38) {\parbox{10mm}{\centering \small 13mm standoff}};
  \node at (20.1,38) {\parbox{11mm}{\centering \small 11mm standoff}};
  \node at (13,64) {\small 2mm front panel};
  \node at (21.4,58.5) {\small 2$\times$ 1.6mm PCBs};
  \draw[>=latex',->,very thick] (13.8,56.7) -- (13.8,50.53);
  \draw[>=latex',->,very thick] (26.4,56.7) -- (26.4,41.64);
  \node at (36.2,-34.14)
    {\parbox{17mm}{\centering \small 18mm clearance for mated power connector}};
  \draw (45.2,-48) -- (45.2,-64);
  \draw[>=latex',<->,thick] (0,-56) -- (45.2,-56);
  \node[fill=white] at (23.0,-56) {\small $\approx$46mm depth};
\end{tikzpicture}\par}
\caption{Assembled module, side view.}\label{fig:panel-stack}
\end{figure}

\section{Use and contact information}

This module design is released under the GNU GPL, version 3, a copy of which
is in the source code package in the file named \texttt{COPYING}.  One
important consequence of the license is that if you distribute the design to
others---for instance, as a built hardware device---then you are obligated
to make the source code available to them at no additional charge, including
any modifications you may have made to the original design.  Source code for
a hardware device includes without limitation such things as the
machine-readable, human-editable CAD files for the circuit boards and
panels.  You also are not permitted to limit others' freedoms to
redistribute the design and make further modifications of their own.

I sell this and other modules, both as fully assembled products and
do-it-yourself kits, from my Web storefront at
\url{http://northcoastsynthesis.com/}.  Your support of my business is what
makes it possible for me to continue releasing module designs for free. 
The latest version of this document and the associated source files can be
found at that Web site.

Email should be sent to\\ \url{mskala@northcoastsynthesis.com}.
