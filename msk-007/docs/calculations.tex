% $Id: calculations.tex 5860 2018-01-23 22:23:18Z mskala $

%
% calculations for the MSK 007 filter curve
% Copyright (C) 2017  Matthew Skala
%
% This program is free software: you can redistribute it and/or modify
% it under the terms of the GNU General Public License as published by
% the Free Software Foundation, version 3.
%
% This program is distributed in the hope that it will be useful,
% but WITHOUT ANY WARRANTY; without even the implied warranty of
% MERCHANTABILITY or FITNESS FOR A PARTICULAR PURPOSE.  See the
% GNU General Public License for more details.
%
% You should have received a copy of the GNU General Public License
% along with this program.  If not, see <http://www.gnu.org/licenses/>.
%
% Matthew Skala
% https://northcoastsynthesis.com/
% mskala@northcoastsynthesis.com
%

\chapter{Filter curve calculations}\label{cha:calculations}

This chapter goes through the calculations for choosing the component values
and calibration data for the MSK 007's Musical Near-Elliptic (MNE) Lowpass
response curve, which is the first and default response for the filter and
corresponds to the silkscreened markings on the circuit boards. 
Understanding these calculations is not necessary for building or using the
filter; they are provided for reference, education, and to support designing
other response curves in the future.  Calculations are usually given using
as many decimal places as the input values allow, up to a maximum of eight,
even though the actual hardware cannot be built to such precision.

\section{Transfer function}

The poles and zeroes were chosen by starting with those of an elliptic
filter that would have low and high cutoff frequencies 1 and 2.3
respectively, with 3dB of passband ripple and a stopband attenuation of
70dB; observing that it has response peaks near frequencies $\tfrac{2}{3}$
and 1 and nulls near 2 and 3, which are in nice harmonic relations to each
other; and then manually choosing the poles and zeroes to make those
relations near-exact.  The resulting selections of poles and zeroes are as
shown in Figure~\ref{fig:polezero}.

\begin{figure}
\hspace*{\fill}\begin{tikzpicture}[>=latex']
  \node at (0,4.5) {Poles: $-0.239$, $-0.178\pm0.680j$, $-0.060\pm1.022j$};
  \node at (0,4.0) {Zeroes: $\pm2j$, $\pm3j$};
  \draw[->] (-2,0) -- (2,0);
  \node at (2,-0.3) {$\Re$};
  \draw[->] (0,-3.5) -- (0,3.5);
  \node at (0.3,3.5) {$\Im$};
  \draw (0,0) circle[radius=1];
  \foreach \y in {-3,-2,2,3} {
    \draw (0,\y) circle[radius=0.1];
  }
  \foreach \x/\y in
    {-0.239/0,-0.178/-0.680,-0.178/0.680,-0.060/-1.022,-0.060/1.022} {
    \draw (\x-0.1,\y-0.1) -- (\x+0.1,\y+0.1);
    \draw (\x-0.1,\y+0.1) -- (\x+0.1,\y-0.1);
  }
\end{tikzpicture}\hspace*{\fill}\par
\caption{Poles and zeroes of the filter response}\label{fig:polezero}
\end{figure}

The Bode plot is in Figure~\ref{fig:bodeplot}, with a curve showing the
phase and some notable points marked.  Note that the frequency at which the
module will oscillate (with an inverting amplifier providing just enough
feedback) is the one where 180$^\circ$ phase shift occurs, namely 0.7300 on
the normalized scale of the plot.  The highest-frequency peak at normalized
frequency $1$ is basically the cutoff of the lowpass action, and is at
$1.3699$ of the oscillation frequency.

\begin{figure*}
{\hspace*{\fill}\includegraphics{mnelp-bode.eps}\hspace*{\fill}\par}
\begin{align*}
  H(s) &=
    \frac{(s^2+4)(s^2+9)}{(s+0.239)(s^2+0.356s+0.494084)(s^2+0.120s+1.048084)}
  \\
  &=
  \frac{s^4+13s^2+36}{s^5+0.715s^4+1.69865s^3+0.8112s^2+0.62119s+0.12376413}
  \\
  \hat{H}(s) &= \frac{0.02777778s^4+0.36111111s^2+1}{8.07988550s^5+
    5.77711814s^4+13.72489751s^3+6.55440312s^2+5.01914408s+1}
\end{align*}
\caption{Bode plot and formulas for the filter response}\label{fig:bodeplot}
\end{figure*}

The transfer function $H(s)$ is as shown in Figure~\ref{fig:bodeplot}.
Normalizing to make the constant terms 1 (that is, dividing the numerator
and denominator each by their constant terms, 36 and 0.12376413 respectively)
gives the normalized transfer function $\hat{H}(s)$, also shown in the
figure.

\section{Leapfrog design procedure}

Now we follow the procedure of Sun\footnote{Yichuang Sun, 2006, Synthesis
of Leap-Frog Multiple Loop Feedback OTA-C Filters, IEEE Transactions on
Circuits and Systems, Part 2: Express Briefs, 53, 9.} for a fifth-order
output summer type leapfrog filter.  In his notation, the coefficients of
the transfer function are:
\begin{gather*}
  A_5=0 \quad A_4=0.02777778 \quad A_3=0 \\
  A_2=0.36111111 \quad A_1=0 \quad A_0=1 \\
  B_5=8.07988550 \quad B_4=5.77711814 \\
  B_3=13.72489751 \quad B_2=6.55440312 \\
  B_1=5.01914408 \quad B_0=1 \, .
\end{gather*}

First we compute the rates for the integrators (design formulas from
Sun's equation display (25), for a fifth-order transfer function).  The
units of measure (not shown) are technically seconds; the numbers
represent the time required for the integrator to charge
from zero to its input voltage if presented with a constant input voltage,
when the filter is tuned to the normalized frequency 1 radian per second.
\begin{align*}
  \tau_5 &=\frac{B_5}{B_4} \\
    &= 1.39860140 \\
  \tau_4 &= \frac{B_4}{B_3-B_2\tau_5} \\
    &= 1.26747916 \\
  \tau_3 &= \frac{B_3-B_2\tau_5}{B_2-(B_1-\tau_5)\tau_4} \\
    &= 2.31905231 \\
  \tau_2 &= \frac{B_2-(B_1-\tau_5)\tau_4}{B_1-\tau_3-\tau_5} \\
    &= 1.51057355 \\
  \tau_1 &= B_1-\tau_3-\tau_5 \\
    &= 1.30146660
\end{align*}

Next we compute the weights for the output mixer (``output summer type''
fifth order topology, design formulas from Sun's equation display (48)). 
These are unitless numbers, representing the proportion of each integrator's
output that should be included in the sum for the global output.  Note the
last one forces them to sum to $A_0$, which (because of the normalization)
is 1.
\begin{align*}
  \alpha_0 &= \frac{A_5}{B_5} \\
    &= 0 \\
  \alpha_1 &= \frac{A_4-\alpha_0B_4}{\tau_2\tau_3\tau_4\tau_5} \\
    &= 0.00447430 \\
  \alpha_2 &=
    \frac{A_3-\alpha_0B_3-\alpha_1\tau_2\tau_3\tau_4}{\tau_3\tau_4\tau_5} \\
    &= -0.00483124 \\
  \alpha_3 &= \frac{A_2-\alpha_0B_2-\alpha_1(\tau_2\tau_3+\tau_2\tau_5+
      \tau4\tau_5)-\alpha_2\tau_3\tau_4}{\tau_4\tau_5} \\
    &= 0.19307299 \\
  \alpha_4 &= \frac{A_1-\alpha_0B_1-\alpha_1(\tau_2+\tau_4)
      -\alpha_2(\tau_3+\tau_5) -\alpha_3\tau_4}{\tau_5} \\
    &= -0.17101598 \\
  \alpha_5 &= A_0-(\alpha_0+\alpha_1+\alpha_2+\alpha_3+\alpha_4) \\
    &= 0.97829992
\end{align*}

\section{Integrator component values}

The integrators in the MSK~007 have rates determined by a global
frequency-control current $I_\textrm{ABC}$ derived from the module-input
control voltages and copied to all integrators; a diode-biasing current
$I_\textrm{D}$ local to a single integrator and actually determined by a
programming resistance $R$ (made up of a fixed resistor and a
trimmer in series); resistor dividers on the positive and negative
voltage inputs; and an integration capacitor $C$.  After setting most of the
other values from circuit-design requirements, we
will choose the value of $R$ to realize a given $\tau$.

For the initial calculation:  assume the module's operating frequency $f$ is
10kHz, corresponding to $I_\textrm{ABC}=1\textrm{mA}$.  With a 1V input, we
want the output capacitor to charge at a rate $\partial V/\partial
t=2\pi f/\tau\textrm{V}=62.831853\textrm{kV}/\textrm{s}$.  Choosing a
470pF capacitor for now (a decision which originated in working backwards
from reasonable scales for the final component values), we want the OTA to
produce an output current $I_\textrm{O}$ of
$62.831853/\tau(\textrm{kV}/\textrm{s})\cdot470pF = \tau29.53097091
\mu\textrm{A}$.

We assume the input of the LM13700 consumes half of the current from the 1V
input, and has infinite impedance.  These assumptions,
although false, have only a multiplicative-constant effect (so that the
ratios will end up right); we can scale the capacitors using experimental
data later to take up any slack, if necessary.

The 1V input is looking into a 141k$\Omega$ impedance and generates a
current of 7.09219858$\mu$A; half of that is the LM13700 input current
$I_\textrm{S}=3.54609929\mu$A.  Then from the formula in the LM13700
datasheet, solving for $I_\textrm{D}$,
\begin{align*}
  I_\textrm{O}
    &= I_\textrm{S}\left(\frac{2I_\textrm{ABC}}{I_\textrm{D}}\right) \\
  I_\textrm{D} &= \tau240.16137504\mu\textrm{A} \, .
\end{align*}

This current comes from imposing 2.496V (the TL0431 reference voltage)
across the programming resistor $R$; so the resistance to produce this is
$2.496\textrm{V}/\tau240.16137504u\textrm{A} =
10.39301178\textrm{k}\Omega/\tau$.  We apply that formula to each of the
programming resistances for the five dividers:
\begin{align*}
  \textrm{R}1+\textrm{R}7 &= 10.39301178\textrm{k}\Omega/1.30146660 \\
    &= 7.98561544\textrm{k}\Omega \\
  \textrm{R}2+\textrm{R}8 &= 10.39301178\textrm{k}\Omega/1.51057355 \\
    &= 6.88017593\textrm{k}\Omega \\
  \textrm{R}3+\textrm{R}9 &= 10.39301178\textrm{k}\Omega/2.31905231 \\
    &= 4.48157712\textrm{k}\Omega \\
  \textrm{R}22+\textrm{R}26 &= 10.39301178\textrm{k}\Omega/1.26747916 \\
    &= 8.19974963\textrm{k}\Omega \\
  \textrm{R}23+\textrm{R}27 &= 10.39301178\textrm{k}\Omega/1.39860140 \\
    &= 7.43100342\textrm{k}\Omega \, .
\end{align*}

Note we want to keep the programming current in the range of a few hundred
$\mu$A.  Texas Instruments recommends using as much as possible given gain
constraints and what the chip can handle, to keep distortion low, with 1mA
as a default target; but other experimenters have reported that 1mA is too
much and causes other distortion, and 500$\mu$A works better.  Here we have
about a 2:1 range between the largest and smallest $\tau$ values, so it's
possible to use the same capacitor values in all integrators and have all
the programming currents in the desired range.  The programming currents
here are from about 304$\mu$A to 557$\mu$A, by Ohm's law on the above
resistances and 2.496V.  If there were a wider range of $\tau$ values, we
might want to use unequal capacitors to bring the programming currents
closer together.

\section{Mixer resistor values}

Now we follow the procedure described in several sources, including
Sheingold\footnote{Dan Sheingold, Analog Dialogue Vol.  10, No.  1 (1976),
``Simple Rules for Choosing Resistor Values in Adder-Subtractor Circuits''}
and
Ardizzoni\footnote{\url{http://electronicdesign.com/ideas-design/efficiently-design-op-amp-summer-circuit}}
to choose resistor values for an op amp summer circuit.  The op amp output
voltage (\texttt{VOUTPUT} from U3C on the schematic) is supposed to be a
weighted sum of the integrator voltages, where the weights are the $\alpha$
coefficients from the leapfrog design procedure.  Some of them are negative;
and in principle \texttt{VINPUT} could be included in the sum too, but for
this particular curve, because $\alpha_0=0$, it will not be included and we
can leave out the components R28 and R29 which would set its value.

We compute the sums of the positive and negative coefficients, as
described by Sheingold.
\begin{align*}
  \Sigma_\textrm{a} &= \sum_{\alpha_i>0} \alpha_i \\
    &= 0.00447430+0.19307299+0.97829992 \\
    &= 1.17584721 \\
  \Sigma_\textrm{b} &= \sum_{\alpha_i<0} -\alpha_i \\
    &= 0.00483124+0.17101598 \\
    &= 0.17584722
\end{align*}

Because the $\alpha$ values always add up to 1 in the leapfrog design
procedure, the discriminant $\Delta$ will necessarily be 0; with the
numerical values above it comes out to 0.00000001 due to rounding, but we
will treat it as 0, which means that the resistor to ground R86 will not be
needed.  It is included in the schematic and PCB for extra flexibility,
should someone want to change the design to have non-unity total gain
through the mixer.

There are different ways to select the feedback resistance $R_\textrm{F}$. 
Sheingold suggests choosing a minimum impedance to appear at either op amp
input and choosing the feedback resistance to achieve that.  I instead
worked from the constraint of not wanting the maximum input resistor value
to be any more than 1M$\Omega$, which means $R_\textrm{F}$ can be at most
1M$\Omega$ times the absolute value of the smallest $\alpha$ (not counting
the zero input coefficient $\alpha_0$; in effect we are assuming we allow
\emph{infinite} resistors, just no finite ones greater than 1M$\Omega$):
\begin{align*}
  R_\textrm{F} &\le 1\textrm{M}\Omega \cdot \min \{ |\alpha| \} \\
    & \le 4.47\textrm{k}\Omega \, .
\end{align*}

The next smaller E24 standard value is 4.3k$\Omega$; we use that for R40.
Resistance values for the inputs (each of which is the sum of a fixed
resistor and a trimmer) follow by dividing the feedback resistor by the
absolute value of each $\alpha$:
\begin{align*}
  \textrm{R}30+\textrm{R}31 &= R_\textrm{F}/|\alpha_1| \\
    &= 961.0437\textrm{k}\Omega \\
  \textrm{R}32+\textrm{R}33 &= R_\textrm{F}/|\alpha_2| \\
    &= 890.0415\textrm{k}\Omega \\
  \textrm{R}34+\textrm{R}35 &= R_\textrm{F}/|\alpha_3| \\
    &= 22.2714\textrm{k}\Omega \\
  \textrm{R}38+\textrm{R}39 &= R_\textrm{F}/|\alpha_4| \\
    &= 25.1438\textrm{k}\Omega \\
  \textrm{R}43+\textrm{R}44 &= R_\textrm{F}/|\alpha_5| \\
    &= 4.3954\textrm{k}\Omega
\end{align*}

The breakdown of specific values for the fixed resistors and trimmers
follows by choosing standard values to give adjustment ranges of between
about $\pm2$\%\ and $\pm5$\%\ around the nominal totals.  The fixed
resistors corresponding to negative $\alpha$ values need to be connected
(using the optional pads on the board) to the op amp negative input; those
are R33 and R39.  The other three go to the positive input.

\section{Mixer pre-adjustment targets}

Most adjustments for the module are done by setting the trimmers in the
resistor network on Board~3.  The recommended procedure is to first
``pre-adjust'' Board~3 by itself, to bring the resistors to the values they
ought to have if the other boards behaved exactly according to the nominal
design values; then connect Board~3 to the other two boards and do any
further adjustments needed to account for imperfect component values on the
other boards (for instance, the tolerance of the integrator capacitors).

For the integrator programming resistances, the pre-adjustment is easy
because disconnecting the boards allows a direct ohmeter reading of each
programming resistance.  For the output mixer, however, things are more
complicated because some of the resistances that need to be adjusted are
permanently connected with other things and cannot be measured in isolation.

With Board~3 separated from the others, the relevant portion of the resistor
network is as shown in this simplified schematic:

{\centering\input{simplres.tex}\par}

There is also a connection
to the power supplies through R87, but its effect is negligible.  The
signals \texttt{VINTA} and \texttt{VINTD} are deliberately assigned to two
of the inter-board pins each, with the connections to the mixer joined to
the OTA input voltage dividers only through traces on Board~2; these
connections are broken when the boards are disconnected.  These two signals
were chosen for this purpose in an effort to maximize the sensitivity of the
remaining indirect measurements while economizing on inter-board pins, which
are in short supply.  An earlier draft design used an elaborate arrangement
of DIP switches on Board 3 to allow isolating the signals, but (given how
well the module performed in breadboard testing even when very inaccurately
adjusted) such complexity seemed unnecessary.

From the diagram, we can measure R30+R31 directly by testing resistance
between test points P8 and P22; R32+R33 between P9 and P23; and R38+R39
between P13 and P22.  That leaves R34+R35 and R43+R44 not directly
accessible.

If we measure between P10 and P22, we see R34+R35 in parallel with a series
combination of 70.5k$\Omega$, 70.5k$\Omega$, and 4.3954k$\Omega$.  The
target measurable value for adjusting R34 is given by $22.2714\textrm{k}\Omega\|(70.5\textrm{k}\Omega+
70.5\textrm{k}\Omega+4.3954\textrm{k}\Omega=19.3130\textrm{k}\Omega$.

If we measure between P16 and P22, we see R43+R44 in parallel with a series
combination of 70.5k$\Omega$, 70.5k$\Omega$, and 22.2714k$\Omega$.  The
target measurable value for adjusting R43 is given by
$22.2714\textrm{k}\Omega\|(70.5\textrm{k}\Omega+
70.5\textrm{k}\Omega+22.2714\textrm{k}\Omega=4.2802\textrm{k}\Omega$.

\section{Precise adjustment targets}

The precise adjustment procedure, with all three boards joined, starts by
tuning the module to oscillate at a specified frequency, then turning down
the feedback, feeding it the same frequency as input, and adjusting the
integrator time constants to get the right phase shift through each
integrator.  The reason to adjust with an external signal rather than while
the filter is self-oscillating is that during self-oscillation, the phase shift
adjustments interact with the resonant frequency in a way that is difficult
to control.

The reference frequency ought to be one we can conveniently measure, and
around the typical setting of the module in actual use.  Guessing that a
cutoff frequency of 1kHz is typical, it corresponds to an oscillation
frequency of 730Hz, and the closest musical note (concert pitch 12-EDO) is
F$\sharp$ an augmented eleventh, or one and a half octaves, above middle C;
that is 739.9888Hz.  That's a
convenient reference point.  Users can call it 740Hz if they wish.  One
period of this frequency is 1351.3757$\mu$s.

The behaviour of the filter, given a sine wave at some fixed frequency, can
be described by a set of simultaneous equations over the complex numbers. 
Where $V_\textrm{I}$ and $B_\textrm{O}$ are the input and output voltages
respectively, and the integrator outputs are $V_\textrm{A}$, $V_\textrm{B}$,
$V_\textrm{C}$, $V_\textrm{D}$, $V_\textrm{E}$, with $f$ a variable
representing the frequency scaling, we have
\begin{align*}
  V_\textrm{A} &= (V_\textrm{I}-V_\textrm{B})fj/\tau_1 \\
  V_\textrm{B} &= (V_\textrm{A}-V_\textrm{C})fj/\tau_2 \\
  V_\textrm{C} &= (V_\textrm{B}-V_\textrm{D})fj/\tau_3 \\
  V_\textrm{D} &= (V_\textrm{C}-V_\textrm{E})fj/\tau_4 \\
  V_\textrm{E} &= (V_\textrm{D}-V_\textrm{E})fj/\tau_5 \\
  V_\textrm{O} &= \alpha_0V_\textrm{I}+\alpha_1V\textrm{A}
    +\alpha_2V\textrm{B}+\alpha_3V\textrm{C} \\
    & \quad\quad +\alpha_4V\textrm{D} +\alpha_5V\textrm{E} \, .
\end{align*}

We can find the oscillation frequency by fixing $V_\textrm{I}=-V_\textrm{O}$
(180$^\circ$ phase shift), and the response to other signals (of each
integrator, not only the overall output response) by setting other values of
$V_\textrm{I}$ and looking at the resulting values of the variables. 
However, instead of solving these equations explicitly, I used the Qucs
circuit simulator with voltage-controlled current sources feeding into ideal
capacitors to simulate the integrators, referencing everything to 1Hz.  An
integrator with a time constant of $\tau$ would be simulated by a
$(1/\tau)$A/V voltage-controlled current source feeding a $(1/2\pi)$F
capacitor.  I also added more voltage-to-current converters with rates equal
to the $\alpha$ values, all feeding into a common summing node with a
current probe to compute the response at the output (technically, it becomes
$I_\textrm{O}$ instead of $V_\textrm{O}$).

Assume the earlier computation of the oscillation frequency as 0.7300 of the
cutoff frequency (which came from using Gnuplot to measure off the
input-to-output transfer-function phase chart) was correct.  Giving the
simulated circuit a sine wave input of 0.7300Hz and 1V allows computing the
complex amplitudes that should be observed in the oscillating filter.  In
magnitude and phase form, they are:
\begin{alignat*}{2}
  V_\textrm{A} &= 1.42150457\textrm{V}&@&-14.3969145^\circ \\
  V_\textrm{B} &= 1.46708733\textrm{V}&@&-63.0804611^\circ \\
  V_\textrm{C} &= 1.08763802\textrm{V}&@&-93.4388609^\circ \\
  V_\textrm{D} &= 1.67693332\textrm{V}&@&-134.421988^\circ \\
  V_\textrm{E} &= 1.17339866\textrm{V}&@&\quad 179.983267^\circ \\
  I_\textrm{O} &= 0.956850634\textrm{V}&@&\quad179.983267^\circ
\end{alignat*}

The closeness of the $I_\textrm{O}$ phase to 180$^\circ$ gives some idea of
the amount of precision we can really expect from this calculation.  Taking
differences of the phases in degrees, and multiplying by the period of the
739.9888Hz oscillation frequency (which is 1351.3757$\mu$s/360$^\circ$), we
get the following phase and time difference targets.  For a perfectly
adjusted module oscillating at 739.99888Hz:
\begin{itemize}
  \item the integrator A output should lead the core input by
    14.3969145$^\circ$ or 54.04344558$\mu$s;
  \item the integrator B output should lead the integrator A output by
    48.6835466$^\circ$ or 182.74933851$\mu$s;
  \item the integrator C output should lead the integrator B output by
    30.3583998$^\circ$ or 113.96001050$\mu$s;
  \item the integrator D output should lead the integrator C output by
    40.9831271$^\circ$ or 153.84333909$\mu$s;
  \item the integrator E output should lead the integrator D output by
    45.594745$^\circ$ or 171.15452900$\mu$s; and
  \item the integrator E output and mixer output should be perfectly in
    phase.
\end{itemize}

Note the core input and mixer output are not the same as the module overall
input and output; there are inverting buffers between the core and the
outside world at both ends, and in feedback mode, the core output goes
through the input buffer, which supplies the remaining 180$^\circ$ of phase
shift.
