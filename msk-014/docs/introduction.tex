% $Id: introduction.tex 9826 2022-02-09 20:05:05Z mskala $

%
% MSK 014 programmer's manual introduction
% Copyright (C) 2022  Matthew Skala
%
% This program is free software: you can redistribute it and/or modify
% it under the terms of the GNU General Public License as published by
% the Free Software Foundation, version 3.
%
% This program is distributed in the hope that it will be useful,
% but WITHOUT ANY WARRANTY; without even the implied warranty of
% MERCHANTABILITY or FITNESS FOR A PARTICULAR PURPOSE.  See the
% GNU General Public License for more details.
%
% You should have received a copy of the GNU General Public License
% along with this program.  If not, see <http://www.gnu.org/licenses/>.
%
% Matthew Skala
% https://northcoastsynthesis.com/
% mskala@northcoastsynthesis.com
%

\chapter{Introduction}

This manual documents the MSK~014 Gracious Host from a programmer's
perspective.  The Gracious Host is a module for use in a Eurorack modular
synthesizer, with the main function of interfacing USB MIDI controller
devices to CV/gate synthesizer patches.  It can be programmed in the field
with alternate firmware, potentially allowing an unlimited range of other
functions, and this manual is intended for programmers interested in
creating alternate firmware, modifying the standard firmware, or studying
how it works.

Writing software for the Gracious Host requires many skills, and a deep
understanding of software and hardware engineering issues that are not taught
in this manual.  This is primarily a reference for qualified programmers,
not a tutorial.  Most users of the module will not be well served by
attempting to modify the firmware themselves, and would do better to read
the \emph{MSK~014 Gracious Host User/Build Manual} (UBM) instead of this
one.  This manual assumes knowledge of the material included there.  You
will also need the manuals, data sheets, and errata published by Microchip
Corporation for the PIC24F microcontroller family; the PIC24FJ64GB002 chip
in particular; the other chips on the board; the assembler and linker (both
the software tools and the manuals for them); and so on.

\section{This manual's organization}

After this introduction, there are a couple of chapters describing the
hardware; then the bulk of the manual is about the standard firmware,
structured as a chapter on tools and building, one on code conventions and
programming tips, then a chapter for each major source file.  The source
chapters are arranged by increasing abstraction level from basic services
close to the hardware, through core subsystems like the USB host driver and
MIDI backend, and finally to the per-device USB drivers, which are basically
applications running on top of the core subsystems.  The manual ends with a
glossary mostly focused on expanding the (many) abbreviations used in my and
Microchip's documentation.  Most all-caps abreviations including TLAs, and
many terms used in \emph{italics}, are defined in the glossary.

Assembly language instructions are printed in lowercase bold, like
\insn{nop}.  Prefix 0x indicates hexadecimal and other numbers are decimal,
as in 0xF00 = 3840.

\section{A note on standards}

The Gracious Host is intended to work with USB and MIDI devices but it is
not a compliant implementation of the associated standards.  Both USB and
MIDI are managed by industry organizations who attempt to enforce rules on,
and collect high membership fees from, companies who use their trademarks or
advertise standards compliance.  As such, it would be inadvisable to use the
trademarked logos of those organizations in connection with the Gracious
Host.

\section{Use and contact information}

This module design, including the firmware described in this manual, is
released under the GNU GPL, version 3, a copy of which is in the source code
package in the file named \texttt{COPYING}.  One important consequence of
the license is that if you create new firmware incorporating parts of my
standard firmware and you distribute the modified firmware in binary form --
for instance, as a loadable firmware image or loaded into a Gracious Host
module -- then you are obliged to make the source code available to
whoever gets the binary.  You are not permitted to limit others' freedoms to
redistribute the code and make further modifications of their own.

I sell the Gracious Host and other modules, both as fully assembled products
and do-it-yourself kits, from my Web storefront at
\url{https://northcoastsynthesis.com/}.  Your support of my business is what
makes it possible for me to continue releasing module designs for free.  The
latest version of this document and the associated source files can be found
at that Web site.

Email should be sent to\\ \url{mskala@northcoastsynthesis.com}.
