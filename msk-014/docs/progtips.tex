% $Id: progtips.tex 9826 2022-02-09 20:05:05Z mskala $

%
% Programming tips for Gracious Host programming
% Copyright (C) 2022  Matthew Skala
%
% This program is free software: you can redistribute it and/or modify
% it under the terms of the GNU General Public License as published by
% the Free Software Foundation, version 3.
%
% This program is distributed in the hope that it will be useful,
% but WITHOUT ANY WARRANTY; without even the implied warranty of
% MERCHANTABILITY or FITNESS FOR A PARTICULAR PURPOSE.  See the
% GNU General Public License for more details.
%
% You should have received a copy of the GNU General Public License
% along with this program.  If not, see <http://www.gnu.org/licenses/>.
%
% Matthew Skala
% https://northcoastsynthesis.com/
% mskala@northcoastsynthesis.com
%

\chapter{Programming tips, conventions, and tools}

This chapter gives some general notes on programming the PIC24, as well as
the conventions followed by the Gracious Host firmware.  Details of specific
source files are in subsequent chapters.

\section{Case and spelling}

I usually like to give global symbols names in all caps, with words
separated by underscores, like SAMPLE\_GLOBAL\_SYMBOL.  Local symbols are
lowercase, like sample\_local\_symbol.  If I want to make something less
visible in scopes where it has to exist for technical reasons, I usually put
an underscore at the start, like the global symbol \_common which is only
used inside macros.  Symbol names created by Microchip's tools often start
with a double underscore and use other capitalization, like
\_\_DefaultInterrupt.

\section{Labels and indentation}

I put labels on lines of their own, because that makes it easier to edit
them.  Often it is desired to move a label without moving the instruction it
points to, and trying to combine a label with an instruction on the same
line would make that harder.  I also indent instructions, not labels, with
one tab character to keep the listings consistently formatted.  Users of
high-level ``structured'' programming languages should be aware that the
customs of assembly language are different: nesting level of program control
flow is \emph{not} normally indicated by indentation in assembly language --
partly because assembly language often does not have a strict nesting
structure anyway -- and indentation customs designed to explicate
deeply-nested code structures are not relevant here.  Nonetheless, in some
nested macro and conditional assembly directives where it seems to make
sense, I do use a two-space indentation per level to make the structure
clear.

The assembler supports a feature called \emph{local symbols}.  The digits
0 through 9 can be used as labels (like ``3:'') any number of times each
without conflicting.  Then in any instruction you can refer to one of these
digits with ``f'' for ``forward'' or ``b'' for ``back,'' and it will
automatically refer to the next or previous instance of that label.  Here is
an example.
\begin{tabbing}
\qquad\=\qquad\qquad\=\kill
\>\insn{bra}\>1f ; skip past the do-nothing loop\\
2:\\
\>; do-nothing loop\\
\>\insn{nop}\\
\>\insn{nop}\\
\>\insn{bra}\>2b\\
1:\\
\end{tabbing}

Local symbols may be a somewhat controversial feature, because of a
perception that they can cause bugs.  My own view is that they should be
used frequently, whenever a label is referred to from nearby and does not
need to be available at a greater distance.  Local symbols make it easier to
cut and paste chunks of code without name collisions or unexpected control
flow; I think they prevent more bugs than they cause.  Bearing in mind that
this form of assembly language basically has no other way of scoping symbols
narrower than an entire source file, it is useful to have a way to refer to
``that instruction right \emph{there}, I know the one I mean'' without
forcing the programmer to always invent a widely scoped unique name for it.

The assembler implements local symbols by renaming them to ordinary symbols
with control characters and serial numbers in their names.  The control
characters are intended to be impossible to create any other way and thus
not collide with ordinary programmer-specified symbol names, and the serial
numbers are intended to prevent the local symbols from colliding with each
other.

\section{Calling conventions}

Programming in assembly language makes it less necessary to have formal
calling conventions of the kind that might be needed by a high-level
language compiler.  Most subroutines are used in only a few places, and in
general I feel free to modify the subroutine's input and output effects to
suit the callers and vice versa.  When a subroutine is global (that is,
visible outside the current source file) I try to document its conventions
clearly in a comment at the start of the subroutine, but it is always a good
idea to read the actual code to make sure there will not be unexpected side
effects.

Usually, subroutines take their inputs in low-numbered working registers
like W2--W6 and return results in W0 or W1.  Low-level subroutines may also
trash some of the low-numbered registers, but will usually try to leave
higher-numbered registers unchanged.  That frees the higher-numbered
registers for use by higher-level code that calls the low-level code, so in
general, register numbers increase with higher levels of abstraction.

In general, I do not store function arguments or results on the stack. 
There is no flexible malloc()-style memory allocation in the firmware;
dynamic memory allocation to the extent it is used is all on the stack.

In some places, I use W14, the stack frame pointer that is special to the
\insn{lnk}/\insn{ulnk} instructions, for special purposes.  Some of the USB
code expects to work in buffers pointed to by W14.  The exception handling
code saves and restores W14 as part of its own stack-frame handling.  This
register is also used -- incompatibly with stack frames -- for the other
thread's program counter in the multithreaded code of the calibration
routine.

\section{Conserving space}

This microcontroller has a lot of speed and not very much memory.

Some of Microchip's example USB programs for PIC24, when compiled by the C
compiler they distribute, will not actually fit on this particular chip. 
Microchip represents it as necessary and appropriate to pay them \$40 per
month for an online-activated license key to restore the optimization
features they deliberately crippled in their distribution of GNU~C, a demand
that seems unlikely to be legal, let alone morally acceptable or appealing
to my own sensibilities, given Microchip's obligations under the General
Public License to the original authors of GNU~C.

But even with uncrippled optimization, the C compiler would be hard pressed
to fit everything into the microcontroller's memory -- especially given the
additional demands associated with being able to re-flash the firmware from
a USB source.  The Microchip-provided USB driver contains multiple
abstraction layers not really relevant to hardware as small as ours, and it
depends on language and library features like ``heap'' memory management. 
It is really meant for use on larger members of the PIC24 family; our chip
is near the bottom end of the range.

When first planning this project, I thought it would be necessary to hold
\emph{two copies} of the firmware in flash at a time, so that the old
firmware could load the new and then transfer control, and that effectively
halved the available space, increasing the pressure further.

With all that in mind, the Gracious Host firmware is written in
hand-optimized assembly language instead of C, with the priority pretty much
always being smaller memory consumption in preference to speed or clarity. 
It is possible that I have taken this emphasis too far, because as of this
writing the firmware fills much less than half the space on the chip, even
including some features I originally thought I might be forced to leave out. 
My code has turned out to be tighter than I expected from my earlier tests
and estimates.  But at least that means there is plenty of room for future
expansion.  In this section, I go through some techniques that may be useful
to programmers attempting to keep the code as small as possible.

\subsection{Use space-saving instructions}

The program memory address space is 24 bits, and so are the instruction
words; so a single-word instruction cannot contain a whole program memory
address when it also needs some bits to say what kind of instruction it is. 
As a result, instructions affecting control flow tend to have ``long'' and
``short,'' or ``far'' and ``near'' versions, depending on whether they use a
second instruction word to have space for a full 24-bit address, or use some
kind of abbreviated target (usually a 16-bit signed number of words offset
from the current program counter) to keep the instruction to just one word. 
But here's the thing:  we only have about 21K words of program memory on this
particular chip.  A 16-bit offset is enough to hit any instruction from any
other; so the short/near control-flow instructions are almost always good
enough.

As such, it's preferable to use unconditional \insn{bra} instead of
\insn{goto}, and \insn{rcall} instead of \insn{call}.

A similar issue applies to data:  some instructions that touch specified
locations in data memory work better, or are only usable at all, for
addresses located in the first 8K of the data memory space.  That is the
range 0x0000 to 0x1FFF.  We only have 8K of RAM, but that RAM is not
all within the first 8K of the address space because the RAM starts at 0x0800,
after the 2K special function register area.  So the first 6K of RAM is more
accessible than the final 2K.

The Gracious Host firmware tries hard to first, keep all variables with
defined locations allocated inside the first 6K of RAM (the final 2K should
normally be part of the stack reservation), and second, make sure symbols
and sections are marked up in such a way that the assembler will \emph{know}
these addresses are in the first 6K and will allow accessing them with the
better instructions.  In particular, the common\_data section defined in
firmware.s is given the ``near'' attribute, so that variables defined within
it should be available with instructions that make that assumption.

The ``skip'' instructions, \insn{btsc}, \insn{btss}, \insn{cpseq},
\insn{cpsgt}, \insn{cpslt}, and \insn{cpsne}, can often save an instruction
here or there, especially when (as is often possible) they're combined with
making one branch of an if/then/else unconditional.  Consider changing
something like this:

\begin{tabbing}
\qquad\=\qquad\qquad\=\kill
\>\insn{cp}\>W0, W1\\
\>\insn{bra}\>lt, 1f\\
\>\insn{mov}\>\#0x123, W2\\
\>\insn{bra}\>2f\\
1:\\
\>\insn{mov}\>\#0x456, W2\\
2:\\
\end{tabbing}

to something like this:

\begin{tabbing}
\qquad\=\qquad\qquad\=\kill
\>\insn{mov}\>\#0x456, W2\\
\>\insn{cpslt}\>W0, W1\\
\>\insn{mov}\>\#0x123, W2\\
\end{tabbing}

Making the \insn{mov} \#0x456, W2 unconditional doesn't matter because it is
immediately overwritten by the other branch if applicable, and this rewrite
saves two instructions.  Note that compare and branch instructions, like
\insn{cpbeq}, are mentioned in the assembly language manual because they are
available in some other PIC24 families, but if you check the fine print you
will realize that those are not actually available in PIC24F.

\subsection{Sharing a tail}

If two subroutines end with the same sequence of two or more instructions,
then one of them can branch to the other.  This costs two cycles for the
branch, but it saves $n-1$ instructions in each place where it's used. 
Consider these two subroutines, which save and restore registers in the same
way:

\pagebreak

\begin{tabbing}
\qquad\=\qquad\qquad\=\kill
\\
foo:\\
\>\insn{push}\>W0\\
\>\insn{push}\>W1\\
\>; do foolish things\\
\>\insn{pop}\>W1\\
\>\insn{pop}\>W0\\
\>\insn{return}\\
\\
bar:\\
\>\insn{push}\>W0\\
\>\insn{push}\>W1\\
\>; do barbaric things\\
\>\insn{pop}\>W1\\
\>\insn{pop}\>W0\\
\>\insn{return}\\
\end{tabbing}

The last three instructions are the same for both, so we can replace
those instructions in one subroutine with a branch to the other, saving two
instruction words:

\begin{tabbing}
\qquad\=\qquad\qquad\=\kill
\\
foo:\\
\>\insn{push}\>W0\\
\>\insn{push}\>W1\\
\>; do foolish things\\
\>\insn{bra}\>RETURN\_W0\_1\\
\\
bar:\\
\>\insn{push}\>W0\\
\>\insn{push}\>W1\\
\>; do barbaric things\\
RETURN\_W0\_1:\\
\>\insn{pop}\>W1\\
\>\insn{pop}\>W0\\
\>\insn{return}\\
\end{tabbing}

\subsection{Convenience labels}

The Gracious Host firmware uses the tail-sharing technique above
extensively, and some subroutine tails that seem like they may be of general
interest are exported as global symbols with consistent names.  In
particular, for returning from an ISR and restoring the first few working
registers that have been pushed on the stack in ascending order, there are
the labels RETFIE\_W0 and RETFIE\_W0\_1 through RETFIE\_W0\_5, defined in
usb.s.  There is also ULNK\_RETURN, for returning from an ordinary
subroutine while discarding a \insn{lnk}/\insn{ulnk} stack frame.

Sharing a tail usually only saves space when the tail is at least two
instructions long, because of the need for an unconditional branch to get to
the shared tail.  But in some cases, when a subroutine only terminates
through a jump anyway, it can be useful to share a single instruction. 
Consider this ``while'' loop:
\begin{tabbing}
\qquad\=\qquad\qquad\=\kill
\\
1:\\
\>\insn{cp}\>W0, W1\\
\>\insn{bra}\>gt, 2f\\
\>; do things\\
\>\insn{bra}\>1b\\
2:\\
\>\insn{return}\\
\end{tabbing}

The \insn{bra gt} instruction doesn't really need to go to \emph{that
particular} \insn{return}; it could equally well point at \emph{any}
\insn{return} \emph{anywhere}.  So if there is another \insn{return}
somewhere in program memory, then we do not need this one in particular to
exist at all.  The firmware provides a global label called RETURN\_INSN,
pointing to a \insn{return} instruction that needed to exist anyway, so the
branch that exits the loop can be changed to \insn{bra} gt, RETURN\_INSN
and foo no longer needs a \insn{return} of its own.

The firmware uses all-caps names ending in \_INSN for single-instruction
convenience labels.  Others it provides are GOTO\_W4\_INSN, RESET\_INSN, and
RETFIE\_INSN.

PIC24 assembly language provides a special instruction called \insn{retlw},
which is a \insn{return} that also moves a 10-bit literal value to a working
register.  This seems to be intended for returning values from functions in
higher level languages.  It's worth knowing about, but in fact I have seldom
actually found it useful in assembly language.  Something similar that I
have found useful is for a subroutine to return ``zero'' or ``non-zero''
status to be checked by an instruction like \insn{bra z} in the caller; and
to support that, there are global labels Z\_RETURN and NZ\_RETURN provided
by the firmware.  A subroutine that wants to return with zero or non-zero
status can \insn{bra} to the appropriate one of these.

\subsection{Tail call and FALL THROUGH}

Suppose the last thing one subroutine does is to call another, like this.
\begin{tabbing}
\qquad\=\qquad\qquad\=\kill
\\
foo:\\
\>; do foolish things\\
\>\insn{return}\\
\\\\
bar:\\
\>; do barbaric things\\
\>\insn{rcall}\>foo\\
\>\insn{return}\\
\end{tabbing}

Then the \insn{rcall} can be changed to a \insn{bra}, eliminatng the
\insn{return}.  The entirety of the call to foo is, in effect, being
used as a shared tail.
\begin{tabbing}
\qquad\=\qquad\qquad\=\kill
\\
foo:\\
\>; do foolish things\\
\>\insn{return}\\
\\
bar:\\
\>; do barbaric things\\
\>\insn{bra}\>foo ; tail call\\
\end{tabbing}

I try to include a comment saying ``tail call'' whenever I use this
technique, to make it clearer to readers that that is what's going on.

We can use tail call for every subroutine that happens to end with a call to
foo, however many of those there may happen to be.  However, with just one
of them we can also eliminate the \insn{bra} instruction by reordering the
subroutines to put the caller immediately before foo in memory and just
letting execution continue past the end of the caller, like this:

\begin{tabbing}
\qquad\=\qquad\qquad\=\kill
\\
bar:\\
\>; do barbaric things\\
\>; bra\>foo ; tail call\\
\>; FALL THROUGH\\
foo:\\
\>; do foolish things\\
\>\insn{return}\\
\end{tabbing}

When using the fall-through technique, I like to leave the \insn{bra}
instruction that was eliminated in place in the source code but commented
out, and add a comment saying ``FALL THROUGH,'' to make the special control
flow more visible.  If I ever move the subroutines around again in the
future, I want to be reminded that then I will need to put the \insn{bra}
back in; and if I ever go looking for ``missing \insn{return} at the end of
subroutine'' bugs, I want to be reminded that in this case it is being done
on purpose.

I also write ``FALL THROUGH'' comments in some other similar cases, such as
in a couple of jump tables, whenever control is deliberately intended to
proceed past a point where readers might expect it to go somewhere else.

\subsection{Star section subroutines}

If a sequence of $n$ instructions is used identically in $k$ different
places in the code, it costs $nk$ ($n$ times $k$) instruction words.  If the
control flow and stack effects do not make it a problem to do this, then
those $nk$ instructions can be replaced with $n+k+1$ instructions by pulling
it out into a subroutine: then there are $n$ \insn{rcall} instructions, plus
$k$ for the one copy of the original sequence in the subroutine, and one
more for the \insn{return}.  A sequence of two instructions can profitably
be made into a subroutine if it is used in at least four places (eight
instructions become seven); a sequence of three instructions needs to be
used in three places to be profitably made into a subroutine; and four or
more only need be used in two places.

When the shared sequence is longer, it can be profitable to turn it into a
subroutine even if the different copies are not \emph{identically} shared,
or when there are in fact stack effects that make a subroutine call more
complicated: the memory saving by having only one copy may be enough to pay
for some additional instructions spent rearranging the stack or handling the
differences between different calling cases.  The subroutine
find\_press\_tbl\_entry in qwerty.s is an example where similar but not
identical logic used in more than one place was first modified so that it
could be identical, and then collapsed into a subroutine.

Although I'm not sure every programmer would agree, I think that readability
of collapsed identical instruction sequences can be improved by making use
of the \emph{star section} feature of the toolchain.  A subroutine cannot be
located in memory exactly where it is used; else we would need to somehow
jump over it.  But the .pushsection assembler direction allows us to
temporarily break out of the stream of instructions we were assembling, and
write some instructions (namely, the subroutine we're defining) that will
actually go somewhere else in memory.  Specifying the name of the new
section as an asterisk tells the assembler to invent (gensym) a locally-valid
name for it that will not conflict with anything else.  Then after writing
the text of the subroutine, .popsection returns assembly to the original
stream.  Putting the pieces together, we can define a subroutine in the
source code in one of the places where it is called and where a human might
want to read it, even though it will actually go somewhere else in memory
and be callable elsewhere.

Here's an example.  Note the instruction sequence limiting W0 to at most 100
and copying it to W2.
\begin{tabbing}
\qquad\=\qquad\qquad\=\kill
\\
foo:\\
\>; do stuff\\
\>\insn{mov}\>\#100, W1\\
\>\insn{cpslt}\>W0, W1\\
\>\insn{mov}\>W1, W0\\
\>\insn{mov}\>W0, W2\\
\>; do stuff\\
\>\insn{return}\\
\\
bar:\\
\>; do other stuff\\
\>\insn{mov}\>\#100, W1\\
\>\insn{cpslt}\>W0, W1\\
\>\insn{mov}\>W1, W0\\
\>\insn{mov}\>W0, W2\\
\>; do other stuff\\
\>\insn{return}\\
\end{tabbing}

If we want to make those four instructions into a subroutine, we could put
the subroutine somewhere else in the source file, but that would be harder
to read.  Using a star section, we can put them inline while still getting
the space saving:
\begin{tabbing}
\qquad\=\qquad\qquad\=\kill
\\
foo:\\
\>; do stuff\\
\>\insn{rcall}\>limit\_and\_store\_w0\\
.pushsection *, code\\
limit\_and\_store\_w0:\\
\>\insn{mov}\>\#100, W1\\
\>\insn{cpslt}\>W0, W1\\
\>\insn{mov}\>W1, W0\\
\>\insn{mov}\>W0, W2\\
\>\insn{return}\\
.popsection\\
\>; do stuff\\
\>\insn{return}\\
\\
bar:\\
\>; do other stuff\\
\>\insn{rcall}\>limit\_and\_store\_w0\\
\>; do other stuff\\
\>\insn{return}\\
\end{tabbing}

For a short subroutine it might even be worthwhile to include a copy of the
eliminated instructions, commented out, at the site of the second call, just
so readers will be able to know what the call does without
cross-referencing.

Putting a subroutine into a star section makes the linker's job more
complicated, because it will have to find a place for one more section in
the final memory map.  However, that can be an advantage: when the linker is
trying to find places for all the code sections, fitting them in between
things like the calibration page that need to be located at specific
addresses, it benefits from having some small relocatable sections that can
fit into otherwise hard-to-use gaps.  Creating at least a few of these small
one-subroutine sections then tends to improve the overall efficiency of
memory utilization.

\section{Common data}

Because of the small amount of RAM on the microcontroller chip, it is
preferable to re-use the same RAM addresses as much as possible.  If two
variables will never be used at the same time, then ideally they should go
at the same address rather than having separate permanently assigned
addresses.

One way of re-using memory is to assign it dynamically at run time, either
with stack frames (used often in the Gracious Host firmware) or with
malloc()-style allocation infrastructure (not used in the Gracious Host
firmware; too much overhead).  For sharing statically allocated addresses,
the PIC24 toolchain has features to support \emph{data overlay} and
\emph{common data} features, each of which has limited usefulness because of
bugs in the assembler and linker.  Data overlay doesn't work because the
linker adds up the lengths of overlaid sections when allocating memory,
instead of taking their maximum; and common data doesn't work for our
purposes because a symbol defined at an offset from a common symbol, loses
the common attribute.  Instead of using the toolchain's broken support, the
Gracious Host firmware simulates common data using macros and a little bit
of manual bookkeeping.

The framework code in firmware.s defines a ``common data'' area to be shared
by all modules that wish to use this support.  The size is set to 1542 bytes
by the \_\_common\_size symbol in global.inc; that represents the largest
amount of common data needed by any module in the current firmware.  That is
the loader: it uses 1536 bytes to buffer one page of program memory for
flash rewriting, plus six bytes of miscellaneous variables.  This allocation
would need to be enlarged if some new module needed more.  The common data
area is tagged ``near'' to ask the linker to keep it within the first 8K of
data memory, allowing the use of shorter instructions; in practice, it is
likely to be allocated at address 0x0850, immediately after the in-circuit
debugging reservation.

The global.inc file also defines a macro called \insn{in\_common} which is
used for allocating variables within the common area.  Call this macro with
a label and size to allocate a symbol of the specified size in the common
area, as follows.
\begin{tabbing}
\insn{in\_common}\quad\quad foo, 2\\
\insn{in\_common}\quad\quad bar, 4\\
\insn{in\_common}\quad\quad baz, 2\\
\end{tabbing}

The \insn{in\_common} macro allocates symbols consecutively into the common
area, without doing any automatic alignment.  Each source file starts fresh
at offset zero in the common area, so symbols from different source files
will overlay each other; it is also possible to start over within a source
file by zeroing \_\_common\_loc, as is done in loader.s.

The calibration routine, the loader, the part of the general USB code that
identifies what kind of device is attached, and several of the per-device
USB drivers all use the common area.  The MIDI backend and the USB I/O
routines that might be called during device driver operation, do not.  In
general, code that may be called from another high-level module should not
touch the common area, but high-level modules like these, of which only one
is active at a time, may use the common area.  The FIND\_IN\_OUT\_ENDPOINTS
routine in qwerty.s is a special case: it assumes a specific layout of
endpoint data structures at the start of the common area, and other drivers
that share this routine (such as the USB MIDI driver) must define equivalent
fields at the same locations.

\section{Exception handling}

Catch and throw exception handling is implemented in utils.s by means of the
global symbols TRY, TRIED, and THROW.  Example code for using them looks
something like the following.

\begin{tabbing}
\qquad\=\qquad\qquad\=\kill
\>\insn{mov}\>\#handle(catch), W1\\
\>\insn{rcall}\>TRY\\
\>\quad; ...\\
\>\quad; if an exception occurs:\\
\>\quad\insn{rcall}\>\quad THROW\\
\>\quad; ...\\
\>\quad; also useful as branch target:\\
\>\quad\insn{bra}\>\quad z, THROW\\
\>\quad; ...\\
\>; if no exception:\\
\>\insn{rcall}\>TRIED\\
\>; ...\\
\\
catch:\\
\>; exception handler \\
\end{tabbing}

The call to TRY starts an exception-handling context, which
will last until the matching call to TRIED.  Exception-handling contexts can
be nested.  Each exception-handling context is associated with a three-word
frame set up on the stack.  The call to TRIED, for non-exceptional
execution, should be made with the same stack pointer that existed
immediately after the return from TRY (thus, normally in the same subroutine
or at the same nesting depth).  But THROW may be called, or branched to, at
arbitrary stack depth, and it restores the stack to its pre-TRY condition,
including the frame pointer W14, when it jumps to the exception handler
address set by the TRY call.  That is the intended purpose of exception
handling: a nested subroutine can signal an exception by calling THROW to
blow out of a variable number of levels of nesting, to get to the
outer-level code that expects to handle the exception.

Put the address of the exception handler in W1 when calling TRY.  Code
symbol addresses need to be marked with the handle() operator, as shown in
the example, to typecast their officially 24-bit values into 16-bit values
that will fit in data registers.  The assembler will complain if this
casting is not done.  On our chip, the high byte of a program memory address
is always zero anyway, so cutting the 24-bit address to 16 bits is easy and
harmless.  On other PIC24 chips, handle() might do more elaborate things
like forcing the toolchain to create a jump table in low memory, allowing a
call to the 16-bit address to lead to code elsewhere in the 24-bit space.

All three of TRY, TRIED, and THROW trash the W0 register.

The framework code in firmware.s creates an initial, default context whose
handler is RESET\_INSN, so a stray THROW in arbitrary code will reset the
module.  Exception handling is used extensively in the general USB driver
for handling error exits from per-device drivers, and in the targeted
peripherals list (next section) for detecting a successful device or
interface match.  A few per-device drivers may use it internally for their
own purposes.

\section{Linker-supported tables}

The PIC24 linker is capable of doing complicated things in the line of
arranging pieces of code from different source files according to
constraints stated in a \emph{linker script}.  The Gracious Host uses a
customized linker script to do what software ``engineers'' might call
dependency injection.  Code defining per-device drivers is inserted into the
USB device-recognition code, without each driver needing to be mentioned in
the general USB source file.  Using the linker to provide this abstraction
means the core USB code does not require changes when support for new
devices is added.  They can be just defined in their own source files and
added to the list of linked object files in the Makefile.  The same
mechanism is used to gather together information about all the
typing-keyboard maintenance codes in the current configuration, defined in
whichever source files contain the actual code to support them without
requiring updates to the keyboard driver as codes are added or changed. 
Changes to these things are made in as few files as possible, reducing the
opportunity for bugs to be introduced by failing to keep disparate files
synchronized.

In more detail:  when any USB host detects a device has been inserted, it
retrieves a \emph{device descriptor} from the device, through which the
device identifies both the general type of device it is and its specific
manufacturer and model; and then one or more \emph{interface descriptors},
through which the device describes which standard or non-standard USB
protocols it can support.  USB hosts in general are supposed to check this
information against a \emph{targeted peripherals list} (TPL), to see whether
they can talk to the inserted device and if so, which driver to use.  The
necessary matching may be complicated because there may be drivers for
specific devices; for specific classes of devices; or for specific
interfaces within a device; and sometimes more than one driver could
possibly match a given device and it is necessary to choose which one is
preferable, which will normally be the one with the narrowest scope, more
specifically tailored to that particular device.

The Gracious Host implements the TPL by splitting it into a \emph{targeted
device list} (TDL) and a \emph{targeted interface list} (TIL), each of which
is a chunk of executable code.  Upon loading the device descriptor, the
general USB driver executes the TDL under certain calling conventions.  The
TDL is expected to throw an exception if some driver takes responsibility
for the device, with W4 pointing at the driver in question.  Otherwise, the
USB driver loops over the interface descriptors, executing the TIL for each
one, until the TIL throws an exception if there is a driver matching the
interface, again with W4 pointing at the driver.  Finally, if no exception
has been thrown, the general USB driver treats the device as unrecognized.

Source files for device drivers register themselves as being able to handle
specific devices, by defining code snippets to recognize those devices and
requesting the linker insert those snippets in the TDL or TIL.  There are
some utility subroutines available for use in the snippets to handle common
types of matching.  The code snippets go in assembly-language sections with
special names, that are picked up by the linker script and gathered together
with similar snippets from other drivers, eventually inserted at the
appropriate points in the USB code.  The order of execution for TDL and TIL
entries is significant, because the first code snippet to recognize the
device or interface and throw its exception will determine which driver
executes.  In fact, a system as small as the Gracious Host is unlikely to
need any really complicated logic for choosing among device drivers; but
given I was implementing this support at all, it costs very little to make
it flexible.

Details of the calling conventions for TDL and TIL code are discussed in the
chapter on the usb.s source file.  From the point of view of the linker, the
precedence order is as follows.  Choosing the section names carefully gives
fine-grained control over which fragments execute first, and therefore which
drivers take priority over others.

\begin{itemize}
  \item The entire TDL runs before the TIL, so any match on the TDL takes
    precedence over any match on the TIL.
  \item TDL sections with numeric-suffix names ``tdl00'' through ``tdl49'' run
    first, in increasing order by number.
  \item Sections named ``tdl'' or starting with ``tdl\_'' run next, in
    arbitrary order.
  \item TDL sections with numeric-suffix names ``tdl50'' through ``tdl99'' run
    last within the TDL, in increasing order by number.
  \item The entire TIL runs for each interface descriptor, in the order
    interface descriptors are returned by the device, so any match on an
    earlier-returned interface descriptor takes priority over any match on
    a later descriptor.
  \item TIL sections with numeric-suffix names ``til00'' through ``til49''
    run before other TIL entries, in increasing
    order by number.
  \item Sections named ``til''; ``tpl''; or starting with ``til\_'' or
    ``tpl\_'' run next, in arbitrary order.
  \item TIL sections with numeric-suffix names ``til50'' through ``til99''
    run last, in increasing order by number.
\end{itemize}

In most cases the order of executioon is not actually important, and most
drivers are expected to use section names like tdl\_foo and til\_foo,
identifying themselves for clearer visibility in debugging output; the other
names supported by the linker script are intended for special circumstances
when one driver needs to do its checks before or after another driver.

Similar, but simpler, linker handling is used to define maintenance codes
for the typing keyboard driver in qwerty.s.  Here the fragments from
different source files are used to define a table of keyboard codes and jump
destinations that, although it is defined in program memory, is scanned as
data and not executed as code.  Each entry should be two words: first word
the maintenance code for the user to type, in BCD, and second word the
address to jump to.  Entries from sections named ``mtbl'' or starting with
``mtbl\_'' will be gathered together to create the table.  The order of
entries is not expected to be important.  For more information, see the
qwerty.s source file and the chapter documenting it.
