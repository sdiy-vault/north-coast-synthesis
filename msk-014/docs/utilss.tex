% $Id: utilss.tex 9738 2021-12-23 21:06:33Z mskala $

%
% Miscellaneous utilities
% Copyright (C) 2022  Matthew Skala
%
% This program is free software: you can redistribute it and/or modify
% it under the terms of the GNU General Public License as published by
% the Free Software Foundation, version 3.
%
% This program is distributed in the hope that it will be useful,
% but WITHOUT ANY WARRANTY; without even the implied warranty of
% MERCHANTABILITY or FITNESS FOR A PARTICULAR PURPOSE.  See the
% GNU General Public License for more details.
%
% You should have received a copy of the GNU General Public License
% along with this program.  If not, see <http://www.gnu.org/licenses/>.
%
% Matthew Skala
% https://northcoastsynthesis.com/
% mskala@northcoastsynthesis.com
%

\chapter{Miscellaneous utilities (utils.s)}

The utils.s source file contains a few small subroutines that may be used
in multiple places throughout the firmware.

\section{Exceptions}

The API for exceptions is described in the ``Programming tips, conventions,
and tools'' chapter of this manual.

As for the implementation:  a static variable named exception\_frame records
the currently active exception frame.  It literally points at the word just
after the exception frame, because it records the value of the stack pointer
(W15, next available word of stack) immediately after the three-word
exception frame was pushed on the stack.

The exception frame consists of three words:  first word stores the old
value of the exception\_frame variable, second word stores the old value of W14
(hardware local variable stack frame), and third word stores the program
memory address of the exception handler.

TRY creates one of these exception frames.  It pops the caller's return
address, builds the frame on the stack, and then does a \insn{goto} to the
return address, so that the exception frame will be left on the stack in the
caller's context.

TRIED marks the end of the exception frame's life.  It restores the values
of W14 and the exception\_frame variable that were stored in the current
exception frame, doing the same \insn{pop}/\insn{goto} routine as in TRY to
access the stack underneath the return address passed by the caller.

THROW redirects flow to the exception handler, blowing out of any
intermediate subroutine calls and W14 stack frames that may have come into
existence between the TRY and the THROW.  It restores the stack pointer W15
to the value stored in exception\_frame, restores W14 and the next outer
exception frame, then branches to the exception handler.  The last three
instructions of THROW coincide with those of TRIED and so are shared.

\section{Linked lists}

These three subroutines are designed to handle single-linked lists in data
memory, where the first field of each list element is a pointer to the next
element and the last element's ``next'' pointer is null, defined to be
0x0000.  The list operations are performed under \insn{disi}
interrupt-disable, so that it will be safe to use them for updating data
structures read by ISRs.

An earlier design for the USB driver used such lists extensively.  These
routines were originally written for that version.  The design has
subsequently been simplified, to the point that the ``insert'' and
``remove'' operations are no longer used anywhere in the current firmware;
so I have commented those two subroutines out, to keep them available for
possible future use without having them consume space.  The ``append''
subroutine is still used in one place; and because its entry point in the
middle of the loop would make inlining it difficult, I do not think there is
anything to be gained by inlining it instead of keeping it as a separate
subroutine.

LL\_APPEND\_ATOMIC appends two single-linked lists.  Call it with W0
pointing at the item(s) to add and W1 pointing at \emph{a pointer to} the
head of the list.  The subroutine traverses the W1 list to find the
terminating null and replaces it with the value of W0.  So W0 should be a
properly-terminated list ([W0]$=$0x0000 if it is a single item).  Requiring
W1 to be pointer to pointer is to allow for appending to a currently-empty
list.

LL\_INSERT\_ATOMIC (currently commented out) requires the same inputs as
LL\_APPEND\_ATOMIC, but inserts the new element pointed to by W0 before the
start of the list pointed to by W1.  The insert operation assumes inserting
exactly one element, and the new element's next pointer is overwritten, thus
need not be initialized first.

LL\_REMOVE\_ATOMIC (currently commented out) removes the element pointed to
by W0, from the list where W1 points to a pointer to the head of the list. 
It traverses the list to find the W0 element and then removes that element. 
Calling it on a list that does not in fact contain the specified element,
is unsafe.

\section{Pseudo-random number generator}

This section implements a stirred entropy pool, similar in nature to
operating system drivers like Linux's /dev/urandom, although much smaller. 
\emph{This PRNG is not rated for cryptographic use.} The USB Mass Storage
driver uses the PRNG to generate ``tag'' values exchanged with the device to
make sure responses match commands, and the MIDI backend uses the PRNG for
random arpeggiation.  This PRNG is probably way over-engineered for these
applications; I just thought it would be fun to implement one using the
PIC24's built-in CRC32 and bit-counting features, in very few code
bytes.  The unpredictability of this PRNG's output is certainly more than
good enough for rock'n'roll.

The PRNG uses, and occupies, the hardware CRC32 peripheral.  Call START\_CRC
to set that up with the proper polynomial and other options before calling
PRNG subroutines, and be aware that PRNG calls will alter the state of the
CRC32 hardware, so they cannot be mixed with other uses of the CRC32
hardware.

Code that will use this facility should call PRNG\_HASH\_TIMERS
occasionally, at times that are not completely predictable.  After interrupt
waits would make sense, because those are at least sometimes determined by
unpredictable external conditions.  The PRNG\_HASH\_TIMERS call looks at the
current count values of Timers~3, 4, and~5.  Timer~3, at least, is always
counting at 2\,MHz, so any uncertainty on the scale of a microsecond in the
timing of the PRNG\_HASH\_TIMERS call will create uncertainty in the count
value.  A 16-bit word constructed from the timer values using \insn{xor}
gets hashed into the CRC32 peripheral on each call.  Part of the design goal
is that PRNG\_HASH\_TIMERS (called frequently and unconditionally) should be
a relatively lightweight operation.  The more expensive processing is
reserved for calls to extract random bits, which are less frequent and may
be conditional on user requests.

In addition to the 32 bits of state in the CRC32 hardware's shift register,
the PRNG keeps eight words in the prng\_pool variable for an additional 128
bits of state.  On a request for random bits, the three nested subroutines
prng\_stir8, prng\_stir4, and prng\_stir execute to mix bits from the CRC32
hardware with the bits in prng\_pool.  The basic stirring operation consists
of adding words, using \insn{ff1l} to count zeros at the left of words, and
permuting bits with \insn{rrnc} and \insn{swap}.  These operations are
selected to be reasonably balanced with respect to 1s and 0s (so that
uniformly distributed words, once stirred, will still be uniformly
distributed), but to also have a little bit of nonlinearity, and good
avalanche among different bit positions.  The nested calls repeat the
stirring between the CRC32 hardware and each word of the pool, enough times
that any uncertainty in the CRC value should affect all bits of the pool.

Random bits can be requested with PRNG\_READ\_WORD, which does the stirring
operation and then takes a 16-bit value from the CRC32 hardware.  The value
is returned in W0 and is expected to be uniformly distributed over all
possible 16-bit values.  This call trashes W2.

The PRNG\_READ\_INT subroutine is a wrapper that limits PRNG\_READ\_WORD to
a selectable range, from 0 to the value of W1 inclusive.  It works by
calling PRNG\_READ\_WORD once, cutting off high bits if necessary to make
the return value the same number of bits as W1, and then if the result is
out of range, it runs the CRC32 hardware further until it gets a result that
is in range.
