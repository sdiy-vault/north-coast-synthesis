% $Id: usbmidis.tex 9802 2022-01-28 23:27:05Z mskala $

%
% USB MIDI interface driver
% Copyright (C) 2022  Matthew Skala
%
% This program is free software: you can redistribute it and/or modify
% it under the terms of the GNU General Public License as published by
% the Free Software Foundation, version 3.
%
% This program is distributed in the hope that it will be useful,
% but WITHOUT ANY WARRANTY; without even the implied warranty of
% MERCHANTABILITY or FITNESS FOR A PARTICULAR PURPOSE.  See the
% GNU General Public License for more details.
%
% You should have received a copy of the GNU General Public License
% along with this program.  If not, see <http://www.gnu.org/licenses/>.
%
% Matthew Skala
% https://northcoastsynthesis.com/
% mskala@northcoastsynthesis.com
%

\chapter{USB-MIDI interface driver (usbmidi.s)}

The code in usbmidi.s is the per-device driver for USB-MIDI devices.  These
would typically include MIDI interfaces (such as to connect DIN-MIDI
devices) and MIDI keyboards or other controllers.  In principle, MIDI
\emph{synthesizers} follow the same standard and the Gracious Host can
connect to them too, but because it only handles MIDI input and does not
generate MIDI events of its own, connecting a synthesizer may not be a
popular thing to do.

Although the MIDI and USB drivers are each quite complicated, the USB-MIDI
driver is simple, because it just serves as glue between these other two.

\section{Data structures}

The source file starts with an entry which the linker will insert into
the executable TPL data structure to recognize USB devices that this driver
can handle.  This is a call to TPL\_MATCH\_INTERFACE\_CLASS that looks for
an interface descriptor of class~1 (``audio''), subclass~3 (``MIDI
streaming'').

Then it defines some data structures in the common area.  The very start of
the common area is laid out in the way assumed by FIND\_IN\_OUT\_ENDPOINTS
in qwerty.s, so that we can reuse that code.  This layout necessitates two
words at the start for pointers to the first in, and first out, endpoint
found; then there follows an array of EPs which will be filled in by
USB\_CONFIGURE\_DEVICE.

USB-MIDI devices, according to the~1.0 standard, may theoretically have
complicated interfaces with many endpoints serving different purposes; but
in practice, real devices usually have exactly one bulk input and one bulk
output endpoint.  Later versions of the standard stopped allowing some of
the more complicated stuff.  To increase the chance of success when
presented with a complicated USB-MIDI~1.0 configuration, the code here
reserves space for up to eight endpoints and then will use the first bulk
input endpoint detected among those as the source for MIDI input.  This
works well on nearly all MIDI devices that someone could reasonably use with
the module.

After the endpoint array comes an IRP structure, and a buffer sized to hold
a maximum-length packet (64~bytes plus 8 bytes of padding to handle possible
DMA overrun, 72 total).

\section{Driver initialization and bulk transfer}

The entry point for the driver, pointed to by the TPL entry, is at
usb\_midi\_driver.  It calls USB\_CONFIGURE\_DEVICE to set the configuration
and clean up the stack, then FIND\_IN\_OUT\_ENDPOINTS from qwerty.s to find
the first input and first output endpoints in the array.  It checks that at
least one input endpoint was actually found (triggering a device unsupported
error if not) and calls MIDI\_INIT to start the backend MIDI driver.

Then the main loop starts, at prepare\_bulk\_request.  It sets up the bulk
in endpoint and IRP for a bulk in transfer with maximum packet size, and
infinite NAKs allowed.  It also sets TOKEN\_ALLOWANCE to 2, to allow a
baseline polling rate of 2000 polls per second.

The wait\_for\_data label starts an inner loop which basically replaces the
waiting loop of USB\_WAIT\_ON\_IRP, calling MIDI\_BACKGROUND and USB\_POKE
until there is data available, with error and disconnect checking.  This
loop increments TOKEN\_STORE on each cycle, so the actual polling rate will
be faster than the 2\,kHz TOKEN\_ALLOWANCE, primarily limited by the time it
takes to call MIDI\_BACKGROUND.

\section{Packet decoding and garbage checking}

Given a packet returned by the bulk endpoint, the first step is to check
that the transfer length is at least 4 bytes, because valid transfers from a
USB-MIDI device are always at least that long.

One of my USB-MIDI devices (an Akai MPK Mini keyboard) has a habit of
sending a 64-byte transfer of what seems to be random garbage upon initial
connection.  To deal with that, and with other devices that may do something
similar, there is a scan of the USB packet for invalid USB-MIDI data. 
USB-MIDI formats its data as 32-bit packets, one or more of which may be
stacked up in a single USB transfer.  Depending on the type of 32-bit
packet, quite often only two or three bytes are used and the remaining bytes
are supposed to be padded with zeros.  Any packet with nonzero data in what
should be the padding bytes according to its apparent packet type, is not
valid USB-MIDI data.

So the validity check steps through each 32-bit packet in the buffer,
looking at the low four bits of the first byte of the packet, which are a
field named CIN from which we can infer the number of pad bytes.  Those bits
are used as indices into the constants 0xB054, which identifies CIN values
that have at least one pad byte, and 0x8020, which identifies CIN values
that have two pad bytes.  All the pad bytes get checked, and if any are
nonzero, then the whole USB transfer is discarded by a branch back to
prepare\_bulk\_request.

After the validity check, there is another loop over the 32-bit packets in
the transfer, looking again at the CIN values.  USB-MIDI usually encodes one
MIDI message (of up to three bytes) into each 32-bit packet.  The first byte
contains the CIN field, which says what type of message it is and basically
duplicates the function of the high nybble of the MIDI status byte.  The
other half of the first byte is a ``cable number,'' which the Gracious Host
ignores.  USB-MIDI also has a mode where it sends one byte of MIDI at a time
instead of an entire message; that is indicated by a special CIN value.  So
this loop checks first whether the packet's CIN value corresponds to a
complete MIDI message we handle at all (some, such as System Exclusive and
some reserved-for-future-use CIN values, at ignored).  If so, then the three
bytes of the message are shuffled into the appropriate registers and a call
to MIDI\_READ\_MESSAGE sends it to the backend.  Otherwise, CIN value is
checked against 15, which indicates a single byte of MIDI data, and if that
matches, the single byte is sent to the backend by a call to
MIDI\_READ\_BYTE.  USB-MIDI stipulates that single-byte and single-message
packets may be mixed; apparently, some devices really do that; and so the
Gracious Host's MIDI backend is designed to support mixed calls to 
MIDI\_READ\_MESSAGE and MIDI\_READ\_BYTE.

The file concludes by a branch back to prepare\_bulk\_request to look for more
data.
