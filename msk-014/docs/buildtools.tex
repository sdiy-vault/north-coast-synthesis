% $Id: buildtools.tex 9826 2022-02-09 20:05:05Z mskala $

%
% MSK 014 software build tools description
% Copyright (C) 2022  Matthew Skala
%
% This program is free software: you can redistribute it and/or modify
% it under the terms of the GNU General Public License as published by
% the Free Software Foundation, version 3.
%
% This program is distributed in the hope that it will be useful,
% but WITHOUT ANY WARRANTY; without even the implied warranty of
% MERCHANTABILITY or FITNESS FOR A PARTICULAR PURPOSE.  See the
% GNU General Public License for more details.
%
% You should have received a copy of the GNU General Public License
% along with this program.  If not, see <http://www.gnu.org/licenses/>.
%
% Matthew Skala
% https://northcoastsynthesis.com/
% mskala@northcoastsynthesis.com
%

\chapter{Build environment and tools}

The Gracious Host firmware is designed to be built in a command-line Linux
environment, with the GNU Assembler for PIC24 as distributed by Microchip
under the name ``XC16.'' I have not tested the build system in other
command-line environments and recommend caution if you attempt to build
firmware other than under Linux.

\section{XC16 Assembler}

Microchip distributes a ``compiler'' package called XC16, primarily intended
for C programming, which actually contains the assembler and the rest of the
toolchain as well.  Downloading and installing this is reasonably
straightforward.  Make a note of where you installed it, because you will
need to edit that into the Gracious Host Makefile later.  The default
installation location seems to be \texttt{/opt/microchip/xc16/v1.23} with
the version number modified to match the version of the package that was
just installed.  I did most of the development using v1.60, and v1.70 seems
to be the latest as of this writing.

I do not recommend paying Microchip money for this package.  Almost all of
it is simply a copy of the GNU multi-architecture toolchain, covered by the
GNU General Public License.  Microchip only added some customizations of
their own, including the machine definition for PIC24 (which is of some
significant value beause of the weirdness of the architecture) but also
including the \emph{deliberate crippling} of the C compiler's optimization
features just so that they could demand a payment for a ``license key'' to
restore those features!  That seems to be not in the spirit of fulfilling
their obligations to the original authors of the code in question.

\section{Building the firmware}

Unpack the source distribution and go to the \texttt{firmware} subdirectory.
Edit the XC16DIR environment variable at the top of the
firmware Makefile to point at where you have installed XC16.

To build the firmware, run \texttt{make} in the \texttt{firmware}
subdirectory of an unpacked Gracious Host source distribution.  You can also
run \texttt{make} in the root of the distribution (one level above) to
recurse into subdirectories and build documentation too.

The Makefile is written assuming GNU Make and supports a ``clean'' target to
remove object and temporary files, and a ``debug'' target which actually
does the same thing as \texttt{make} or \texttt{make all}; this is
intended to help support the MPLAB X IDE, which tries to draw a distinction
between ``debug'' and ``production'' builds.

The build process is a bit complicated, especially when it comes to
preparing a loadable image for the module to re-flash itself from USB. 
There are several steps of scanning object files, extracting addresses, and
rewriting include files to get information into the boot loader.  Because
the loadable image includes checksums covering other checksums, there is
also a loop that builds an image with checksums, recalculates the checksums
and writes them in, then tries again until all checksums match and require
no further changes.  It will \emph{not} work to just assemble the source
files and link them; you need to really use the supplied Makefile and
associated Perl scripts.  Note that this also means you should not allow a
tool like MPLAB X IDE to destroy the supplied Makefile.

A Perl interpreter is required, and it needs to have the String::CRC32
module installed.  Other Perl modules may possibly also need to be
installed; I'm not sure which ones are commonly found on default
installations.  \emph{Pay attention to build error messages} and supply the
missing pieces as necessary.

\section{MPLAB X IDE}

Microchip's ``MPLAB X IDE'' product has many issues, but it may be a
necessary evil if you wish to do in-circuit debugging with a tool like their
PICkit programmer/debugger.  The main problem is that MPLAB X IDE wants to
\emph{put itself on top}:  that is, it wants to be in charge of the build
process and have other tools become part of its system instead of it being a
part of some larger system.  If you must use MPLAB X IDE, here are some tips
on getting it to work.

You will need to create a ``project.''  When doing so (first screen of the
wizard, titled ``Choose Project''), choose ``User Makefile Project'' as the
type to create.

On the ``Select Device'' screen, choose ``PIC24FJGB002'' as the device,
and your choice of tool (Simulator, PICkit, or whatever).  The ``Family''
field seems to only narrow down the drop-down options for the device list;
the most useful value is ``16-bit MCUs (PIC24)'' but you may get better
results by typing the number directly into the device box instead of digging
through the (very long, even when narrowed) list of all possible MCUs.

On the ``Select Project Name and Folder'' screen, it is important to check
the ``Use project location as the project folder'' box (which is not
default) and select the ``project location'' to be the \texttt{firmware}
subdirectory of the unpacked Gracious Host source distribution.  If this is
not done, MPLAB X IDE will attempt to create a ``project'' directory with a
.X extension and will expect all builds to happen there, which will not work.

On the ``Create User Makefile Project'' screen, set the ``working
directory'' to ``.'' (that is one period, meaning current directory) and set
the following values.
\begin{itemize}
  \item Build command:  \texttt{make}
  \item Debug build command:  \texttt{make debug}
  \item Clean command:  \texttt{make clean}
  \item Image name:  \texttt{firmware.elf}
  \item Debug image name:  \texttt{firmware.elf}
\end{itemize}

Creating a project this way will probably overwrite the existing Makefile,
so you may have to re-extract it from the source distribution.  I believe
these settings will prevent MPLAB X IDE from overwriting the Makefile
\emph{again} in the future, however -- it only seems to do it on the initial
project creation.

In-circuit debugging with MPLAB X IDE and a debugging tool like a PICkit has
two modes for breakpoints: hardware breakpoints and software breakpoints,
the latter also called ``unlimited'' breakpoints.  Hardware breakpoints make
use of features built into the silicon.  They have less performance impact,
and they can do clever things like breaking on access to a data memory
address (not only an instruction execution).  However, there is an overrun
associated with hardware breakpoints: after a hardware breakpoint triggers,
one more instruction will execute before the code actually stops.  That can
be annoying.  It is possible to work around the overrun by simply placing
the breakpoint one instruction earlier than the point where you really want
it to break, but in densely branching control flow it may not always be
possible to know which instruction is ``one instruction earlier,'' and this
workaround is not applicable to breaking immediately upon wakeup from idle
mode, nor to the debugger's ``run until return'' feature.  Also, you only
get four hardware breakpoints, and \emph{really} you only get three, because
debugger features like single-stepping also depend on the silicon resources
involved.

So MPLAB X IDE will urge you to switch to software breakpoints when you
approach the limit for hardware breakpoints, with pop-ups that do not
explain the consequences of saying ``yes.'' It is easy to make this switch
by accident.  I have several times found software breakpoints selected when
I was not aware of having deliberately selected them.  The number of
software breakpoints is unlimited, and they really break at the specified
locations.  The trouble is, they are implemented by repeatedly erasing and
reprogramming the chip's program memory -- the debugger actually edits the
code on the fly, presumably to insert the undocumented \insn{break}
instruction.  That can wear out the chip's program memory relatively
quickly.  The chip's program memory is supposed to be good for 10,000 erase
cycles, which should be plenty (comparable to the lifespan of components
like jack sockets) in normal module use where someone might be recalibrating
their module or swapping in new firmware at most once per day long-term
average.  But 10,000 erase cycles could quickly be used up if someone is
doing intense debugging with software breakpoints enabled.  So I recommend
avoiding software breakpoints, unless \emph{maybe} on a module specifically
built for development purposes that you are willing to consider sacrificial
if you wear the chip out.

\section{The configuration include file}

There are some configurable options in the file config.inc, which you can
adjust as desired.  These are described in detail in the comments inside the
file.  Some that may be of interest even in ``production'' firmware are for
controlling things like the rate of pitch bend with the shift keys in the
typing keyboard driver.

Testing firmware during development is difficult on the live hardware, even
with ICD, because the firmware needs to communicate with off-chip
peripherals that have their own timing requirements, and the debugger only
directly controls the chip and its on-chip peripherals.  If the firmware is
in the middle of talking to a USB device when the debugger stops it, then it
will probably be at least a few seconds before the microcontroller starts
executing code again, and by then the USB device will long since have given
up.  When first writing the firmware, I also wanted to do development on it
before the Gracious Host module hardware existed at all; and when it existed
but was not in its final form.  On non-live hardware or in a software
simulator, peripherals designed for interacting with the outside world may
not present sensible results to the firmware, because they do not have the
normal outside world to work with.

So there are a number of options in config.inc designed to help with testing
the firmware other than in the normal hardware configuration.  The FRC\_OSC
option activates the on-chip oscillator, for running on development boards
with no external clock oscillator.  The NO\_WDT option disables the watchdog
timer (although the debugger will probably handle that anyway).  And there
are several symbols starting with SIMULATE\_ that assemble extra code to
simulate different hardware devices included in the Gracious Host module;
that way, the higher-level code that requires results from those devices can
be tested on an evaluation board or even in Microchip's software simulator,
where those devices cannot produce realistic results when accessed directly.

Some other debugging aids are intended for use when running on the real
hardware.  The FILL\_RAM\_DEAD symbol causes the firmware to fill all the
general-purpose RAM with the value 0xDEAD at start-up.  That makes it easier
to see where other data has been written, for detecting buffer overruns and
similar.  The LEDS\_ON\_USB\_ATTACHED symbol is specifically for debugging
detection of USB attach and detach (which took a lot of debugging when I was
writing that code); it may conflict with other uses of the LEDs implemented
later.  Define SEQUENTIAL\_CALIBRATION to make the two calibration threads
run sequentially instead of simultaneously; the stack manipulation needed to
get them running simultaneously screws up Microchip's debugger, so it's
usually easier to turn it off when the multitasking itself is not the thing
being debugged.  The TRAP\_HANDLERS symbol is a bitmask of stub trap
handlers that should be defined, each of which will contain a couple of
\insn{nop} instructions as a target for debugger breakpoints, before they
all feed into a master trap handler that also serves as a target and then
resets the microcontroller.  Putting breakpoints in trap handlers is useful
when trying to figure out what kind of trap actually occurred, and to get a
look at the stack, before the microcontroller resets.

Pin~14 of the microcontroller (GPIO pin RB5) is run out to the test point~P5
on the back of the module, and it can be useful for examining timing of
firmware events while the code is running.  You can add instructions like
\insn{bset} LATB, \#5 and \insn{bclr} LATB, \#5 at appropriate points in the
code and then watch the test point with an oscilloscope to see what is going
on, in something like the hardware equivalent of ``printf debugging.'' Two
symbols are provided in config.inc, PULSE\_PIN14\_ON\_BUSY and
PULSE\_PIN14\_ON\_SOF, for events that I wanted to watch often enough to be
worth adding some infrastructure; but I also found it useful during
development to add instructions at other points ad hoc, modifying the code
as necessary, to trigger the scope on events of interest.

Some more elaborate tests are written in assembly language and designed to
be run using the debugger or simulator.  These are controlled by symbols
starting TEST\_, each of which assembles a short infinite-loop test routine. 
If SKIP\_TESTS is \emph{not} defined, then after initial reset the code will
jump to one of these tests instead of the firmware main loop (and cause a
crash if none of them are activated); so SKIP\_TESTS needs to be defined in
a ``production'' version.  The different tests are described in their own
chapter, later in this manual.

\section{Special-purpose include files}

As often happens in complicated software builds, it is sometimes necessary
to pass small chunks of information among different files outside the usual
source/object file pattern.  For instance, building an installable image for
firmware updating requires knowing the final addresses of symbols in memory,
which have to be determined by scanning the linker output.
The Gracious Host firmware build generates
several include files automatically, which get included in appropriate
source files to pass this kind of information around.  They are summarized
here and discussed in more detail in the chapters for the specific source
files involved.

\begin{description}
\item[notetbl.inc] A table of period values to be used with output compare
units for generating musical-note frequencies; generated by the mknotetbl
Perl script and included in firmware.s.

\item[rndpage.inc] Defines a randomly-chosen page of program memory that
will be used for calibration data by calibration.s.  The random selection is
made by inline Perl code in the Makefile.

\item[simdrive.inc] Defines the contents of the simulated USB Mass Storage
device for testing the FAT code.  Included by usbmass.s, conditionally on
drive simulation being enabled by config.inc.  This include file is
handwritten, but in some configurations it depends on binary data generated
by the mksimdrives script, which in turn needs to be manually invoked (the
Makefile will not do it) because it needs to run as root to mount and
unmount loopback devices.

\item[loader-addr.inc] Included by loader.s when assembling the ``high''
version of the loader, to tell it exactly where in memory that copy should
be placed.  Created by the mkloaddr Perl script, after measuring the size of
the ``low'' version of the loader.

\item[image-syms.inc]  Defines addresses and CRC values that will need to be
written into the loadable firmware image.  The Makefile creates this using
the mkimagesyms Perl script, in a loop:  first it creates an image using
placeholder values (which will \emph{not} be loadable), then generates new
values for all the symbols based on that image, re-generates the image with
the new values, and repeats until the values do not change.

\item[fw-pages.inc]  Page occupancy (symbols identifying which program
memory pages will need to be overwritten) for the loadable image.  Generated
by the dmp2bin perl script after scanning the firmware ELF image.

\item[image-id.inc]  Identifying information (hostname, username, and the
date) added to the header of the loadable image to make it easier to
recognize the provenance of random binary images.
\end{description}

\section{The listing file}

Run \texttt{make listing.pdf} to generate a pretty-printed PDF of all the
listing files generated during firmware assembly.  This mostly just exists
because I think it looks cool, but it may be easier to read than the
original .s source files, and because it also includes a hex dump of the
assembler's output, it can sometimes be useful for purposes like debugging
loader image files, when it may be desired to find the exact bytes that
correspond to particular lines of code.
