% $Id: qwerty.tex 9718 2021-12-19 19:26:46Z mskala $

%
% Typing (QWERTY) keyboard interface
% Copyright (C) 2022  Matthew Skala
%
% This program is free software: you can redistribute it and/or modify
% it under the terms of the GNU General Public License as published by
% the Free Software Foundation, version 3.
%
% This program is distributed in the hope that it will be useful,
% but WITHOUT ANY WARRANTY; without even the implied warranty of
% MERCHANTABILITY or FITNESS FOR A PARTICULAR PURPOSE.  See the
% GNU General Public License for more details.
%
% You should have received a copy of the GNU General Public License
% along with this program.  If not, see <http://www.gnu.org/licenses/>.
%
% Matthew Skala
% https://northcoastsynthesis.com/
% mskala@northcoastsynthesis.com
%

\newcommand{\myFlFl}{\raisebox{0.2ex}{\musDoubleFlat}}
\newcommand{\myFl}{\raisebox{0.2ex}{\musFlat}}
\newcommand{\mySh}{\raisebox{0.4ex}{\musSharp}}
\newcommand{\myShSh}{\raisebox{0.4ex}{\musDoubleSharp}}

\chapter{Typing keyboard interface}

The Gracious Host supports two kinds of USB keyboards:  musical keyboards
with a USB MIDI interface as described in the previous chapter, and the
other kind of keyboard that is commonly attached to a PC and used for
typing.  The driver for typing keyboards, described in this chapter, allows
them to be used for playing synthesizers -- with some limitations, but at a
much lower cost than most full-featured MIDI keyboards.

%%%%%%%%%%%%%%%%%%%%%%%%%%%%%%%%%%%%%%%%%%%%%%%%%%%%%%%%%%%%%%%%%%%%%%%%

\section{General comments on USB keyboards}

The typing keyboard driver supports what the USB standards call the ``boot
keyboard'' protocol.  In technical USB terms that means it will support a
USB device with an interface descriptor of class 3, subclass 1, protocol 1. 
This is the standard USB keyboard protocol used by the typing keyboards
people commonly plug into personal computers.  It is called ``boot
keyboard'' because the authors of the relevant standard apparently imagined
that the simple protocol would only be used by the BIOS configuration menus
of PCs, and most operating systems would instead use the ridiculously
complicated full-generality ``HID report protocol'' instead; but in fact,
most real-world implementations just use the boot keyboard protocol.

Wiring switches together to form a keyboard becomes more complicated if it
is desired to correctly detect when the user presses more than one key at
the same time.  In musical keyboards this issue is called ``polyphony'';
with typing keyboards, the same thing is called ``rollover.'' The USB boot
keyboard protocol is only capable of reporting to the host at most six
simultaneous keypresses (of regular typing keys; modifier keys like shift
are handled separately and do not count toward this limit) because it sends
six bytes of key information per report packet with each key consuming one
byte.  So a USB keyboard plugged into the Gracious Host can play \emph{at
most} six simultaneous MIDI notes in the ordinary way by pressing keys
at once.

However, reaching the limit requires keyboard hardware actually capable of
six-key rollover.  Some keyboards, especially fancier ones marketed as
``gaming'' keyboards, are able to do it; but cheaper generic USB keyboards
cannot really do full six-key rollover.  The average USB typing keyboard
that does not specifically advertise a rollover feature may give
incorrect results (not detecting some keys, or spuriously detecting
unpressed ``ghost'' keys) when multiple keys are pressed, usually depending
in a complicated way on the specific key combination involved.  The
Gracious Host only knows what the keyboard tells it, and cannot easily
compensate for such behaviour.

For the quantizer and arpeggiator features it's useful to play many notes at
once; so to get around both keyboard rollover limits and the user's limited
number of fingers, the Gracious Host typing-keyboard driver supports a
\emph{sustain} feature using the Caps Lock key to lock keys in a pressed
state without needing to hold them down.  See the section on that below.

USB typing keyboards have a limit on how fast they can be polled, and that
limits how responsive to keystrokes anything controlled by a USB typing
keyboard can be.  The limit depends on the specific keyboard model (the
keyboard decides how fast it will respond) but the default is usually
10ms intervals, for 100 polls per second.  There can also be a further delay
of a few milliseconds internal to the keyboard as the keyboard hardware
scans the array of switches.  Human beings are not capable of perceiving this
amount of so-called latency in keystroke response but many believe
they can, and users who hold that belief might prefer to use a USB MIDI
keyboard instead of a typing keyboard.  The USB MIDI protocol as implemented
by the Gracious Host allows for sub-millisecond polling.

The general layout of a USB typing keyboard is reasonably standardized, but
there are many small details that vary from one to another.  The biggest
distinction is between ``US-style'' keyboards, with a large L-shaped Enter
key, and ``ISO-style'' keyboards with a narrower vertical Enter key and a
few more small keys to accommodate the additional letters in non-English
languages.  The precise details of those keys, which letters are shown on
which keycaps, the locations of some punctuation marks like backslash, and
so on, vary a lot.  Keyboards do not report the details of their layout to a
USB host, so the host has to guess, maintain a database of specific keyboard
models, or be configured for it out-of-band.

The Gracious Host attempts to support all relevant keys that exist in most
popular keyboard layouts used around the world, with reasonable guesses as
to where they will be located; but be aware that it's possible your keyboard
will not actually have all the keys shown in the diagrams in this manual, or
that a few of them (especially around the edges of the main keyboard area)
will not be located exactly where they are shown.

The keys labelled ``GUI'' (and not implemented to do anything, in
the current firmware) are called that here because that's what they are
called in the relevant USB standard.  On most keyboards they are actually
labelled with the logo of an operating system vendor, like Microsoft or
Apple.

%%%%%%%%%%%%%%%%%%%%%%%%%%%%%%%%%%%%%%%%%%%%%%%%%%%%%%%%%%%%%%%%%%%%%%%%

\section{Piano-style keyboard layout}

Figure~\ref{fig:qwerty-layouts} shows two layouts.  The upper one is the
default layout active when a keyboard is first plugged into the Gracious
Host and the Num Lock LED is off.  Note names and MIDI note numbers
(assuming no octave shift) are as shown in the diagram.

\begin{sidewaysfigure*}
{\centering
\begin{tikzpicture}[scale=0.234]
  % first row
  \foreach \xa/\xb/\tb in
    {0/4/16,
     8/12/1,12/16/2,16/20/3,20/24/4,
     26/30/5,30/34/6,34/38/7,38/42/8,
     44/48/9,48/52/10,52/56/11,56/60/12,
     62/66/13,66/70/14,70/74/15} {
    \draw[rounded corners={1mm}]
      ($(\xa,22)+(0.2,0.2)$) rectangle ($(\xb,26)+(-0.2,-0.2)$);
    \node[red!50!black] at ($(\xa,26)+(2,-1)$)
      {\tiny\textsf{Channel}};
    \node[red!50!black] at ($(\xa,22)+(2,1.5)$) {\large\bf\textsf{\tb}};
  }
  % LEDs
  \foreach \x/\t in {76/Num,80/Caps,84/Scroll} {
    \draw (\x+0.6,24.75) rectangle (\x+2.1,25.25);
    \node[anchor=west] at ($(\x,26)+(0,-2)$) {\tiny\textsf{\t}};
    \node[anchor=west] at ($(\x,26)+(0,-3)$) {\tiny\textsf{Lock}};
  }
  % second row, black note keys
  \foreach \xa/\xb/\ta/\tb in
    {0/4/58/{A\mySh},8/12/61/{C\mySh},12/16/63/{D\mySh},
     20/24/66/{F\mySh},24/28/68/{G\mySh},28/32/70/{A\mySh},
     36/40/73/{C\mySh},40/44/75/{D\mySh},48/52/78/{F\mySh},
     52/56/80/{G\mySh},56/60/82/{A\mySh}} {
    \draw[rounded corners={1mm},fill=black]
      ($(\xa,16)+(0.2,0.2)$) rectangle ($(\xb,20)+(-0.2,-0.2)$);
    \node[anchor=west,white] at ($(\xa,20)+(0,-1)$) {\tiny\bf\textsf{\ta}};
    \node[white] at ($(\xa,16)+(2,1.5)$) {\large\bf\textsf{\tb}};
  }
  % second row, skipped keys
  \foreach \xa/\xb/\ta/\tb in
    {4/8/60/{B\mySh},16/20/65/{E\mySh},32/36/72/{B\mySh},44/48/77/{E\mySh}} {
    \draw[rounded corners={1mm},fill=black!20!white]
      ($(\xa,16)+(0.2,0.2)$) rectangle ($(\xb,20)+(-0.2,-0.2)$);
    \node[anchor=west] at ($(\xa,20)+(0,-1)$) {\tiny\bf\textsf{\ta}};
    \node at ($(\xa,16)+(2,1.5)$) {\large\bf\textsf{\tb}};
  }
  % second row, non-note keys
  \foreach \xa/\xb/\ta/\tb in
    {62/66/Insert/{},66/70/Home/{},70/74/Page/Up,
     76/80/{}/{},80/84/{/}/{},84/88/*/{},88/92/-/{}} {
    \draw[rounded corners={1mm}]
      ($(\xa,16)+(0.2,0.2)$) rectangle ($(\xb,20)+(-0.2,-0.2)$);
    \node[anchor=west] at ($(\xa,20)+(0,-1)$) {\tiny\textsf{\ta}};
    \node[anchor=west] at ($(\xa,20)+(0,-2)$) {\tiny\textsf{\tb}};
  }
  % third row, white note keys
  \foreach \xa/\xb/\ta/\tb in
    {0/6/59/B,6/10/60/C,10/14/62/D,14/18/64/E,
     18/22/65/F,22/26/67/G,26/30/69/A,30/34/71/B,
     34/38/72/C,38/42/74/D,42/46/76/E,46/50/77/F,
     50/54/79/G} {
    \draw[rounded corners={1mm}]
      ($(\xa,12)+(0.2,0.2)$) rectangle ($(\xb,16)+(-0.2,-0.2)$);
    \node[anchor=west] at ($(\xa,16)+(0,-1)$) {\tiny\textsf{\ta}};
    \node at ($(\xa,12+1.5)!0.5!(\xb,12+1.5)$) {\large\bf\textsf{\tb}};
  }
  % third row, non-note keys
  \foreach \xa/\xb/\ta/\tb in
    {62/66/Delete/{},66/70/End/{},70/74/Page/Down} {
    \draw[rounded corners={1mm}]
      ($(\xa,12)+(0.2,0.2)$) rectangle ($(\xb,16)+(-0.2,-0.2)$);
    \node[anchor=west] at ($(\xa,16)+(0,-1)$) {\tiny\textsf{\ta}};
    \node[anchor=west] at ($(\xa,16)+(0,-2)$) {\tiny\textsf{\tb}};
  }
  % keys that span third and fourth rows
  \draw[rounded corners={1mm}]
    ($(55,8)+(0.2,0.2)$) -- ($(55,12)+(0.2,0.2)$) --
    ($(54,12)+(0.2,0.2)$) -- ($(54,16)+(0.2,-0.2)$) --
    ($(60,16)+(-0.2,-0.2)$) -- ($(60,8)+(-0.2,0.2)$) --cycle;
  \node[anchor=west] at ($(54,16)+(0,-1)$) {\tiny\textsf{Enter}};
  \draw[rounded corners={1mm}]
    ($(88,8)+(0.2,0.2)$) rectangle ($(92,16)+(-0.2,-0.2)$);
  \node[anchor=west] at ($(88,16)+(0,-1)$) {\tiny\textsf{+}};
  % fourth row, black note keys
  \foreach \xa/\xb/\ta/\tb in
    {11/15/49/{C\mySh},15/19/51/{D\mySh},
     23/27/54/{F\mySh},27/31/56/{G\mySh},31/35/58/{A\mySh},
     39/43/61/{C\mySh},43/47/63/{D\mySh},
     51/55/66/{F\mySh}} {
    \draw[rounded corners={1mm}, fill=black]
      ($(\xa,8)+(0.2,0.2)$) rectangle ($(\xb,12)+(-0.2,-0.2)$);
    \node[anchor=west,white] at ($(\xa,12)+(0,-1)$) {\tiny\bf\textsf{\ta}};
    \node[white] at ($(\xa,8)+(2,1.5)$) {\large\bf\textsf{\tb}};
  }
  % fourth row, skipped keys
  \foreach \xa/\xb/\ta/\tb in
    {7/11/48/{B\mySh},
     19/23/53/{E\mySh},
     35/39/60/{B\mySh},47/51/65/{E\mySh}} {
    \draw[rounded corners={1mm},fill=black!20!white]
      ($(\xa,8)+(0.2,0.2)$) rectangle ($(\xb,12)+(-0.2,-0.2)$);
    \node[anchor=west] at ($(\xa,12)+(0,-1)$) {\tiny\bf\textsf{\ta}};
    \node at ($(\xa,8)+(2,1.5)$) {\large\bf\textsf{\tb}};
  }
  % fourth row, non-note keys
  \foreach \xa/\xb/\ta/\tb in
    {0/7/{}/{}} {
    \draw[rounded corners={1mm}]
      ($(\xa,8)+(0.2,0.2)$) rectangle ($(\xb,12)+(-0.2,-0.2)$);
    \node[anchor=west] at ($(\xa,12)+(0,-1)$) {\tiny\textsf{\ta}};
    \node[anchor=west] at ($(\xa,12)+(0,-2)$) {\tiny\textsf{\tb}};
  }
  % fifth row, white note keys
  \foreach \xa/\xb/\ta/\tb in
    {5/9/47/B,9/13/48/C,13/17/50/D,17/21/52/E,
     21/25/53/F,25/29/55/G,29/33/57/A,33/37/59/B,
     37/41/60/C,41/45/62/D,45/49/64/E} {
    \draw[rounded corners={1mm}]
      ($(\xa,4)+(0.2,0.2)$) rectangle ($(\xb,8)+(-0.2,-0.2)$);
    \node[anchor=west] at ($(\xa,8)+(0,-1)$) {\tiny\textsf{\ta}};
    \node at ($(\xa,4+1.5)!0.5!(\xb,4+1.5)$) {\large\bf\textsf{\tb}};
  }
  % fifth row, non-note keys
  \foreach \xa/\xb/\ta/\tb in
    {0/5/{}/{},49/60/{}/{},
     66/70/{$\uparrow$}/{}} {
    \draw[rounded corners={1mm}]
      ($(\xa,4)+(0.2,0.2)$) rectangle ($(\xb,8)+(-0.2,-0.2)$);
    \node[anchor=west] at ($(\xa,8)+(0,-1)$) {\tiny\textsf{\ta}};
    \node[anchor=west] at ($(\xa,8)+(0,-2)$) {\tiny\textsf{\tb}};
  }
  % keys that span fifth and sixth rows
  \draw[rounded corners={1mm}]
    ($(88,0)+(0.2,0.2)$) rectangle ($(92,8)+(-0.2,-0.2)$);
  \node[anchor=west] at ($(88,8)+(0,-1)$) {\tiny\textsf{Enter}};
  % sixth row
  \foreach \xa/\xb/\ta/\tb in
    {0/6/{}/{},6/11/GUI/{},11/16/Alt/{},16/39/{}/{},
     39/44/Alt/{},44/49/GUI/{},49/54/Menu/{},54/60/{}/{},
     62/66/{$\leftarrow$}/{},66/70/{$\downarrow$}/{},70/74/{$\rightarrow$}/{},
     76/84/{}/{},84/88/./Del} {
    \draw[rounded corners={1mm}]
      ($(\xa,0)+(0.2,0.2)$) rectangle ($(\xb,4)+(-0.2,-0.2)$);
    \node[anchor=west] at ($(\xa,4)+(0,-1)$) {\tiny\textsf{\ta}};
    \node[anchor=west] at ($(\xa,4)+(0,-2)$) {\tiny\textsf{\tb}};
  }
  % keypad numerals
  \foreach \x/\y/\t in {76/4/14,80/4/28,84/4/42,76/8/56,80/8/70,
    84/8/84,76/12/98,80/12/112,84/12/126} {
    \draw[rounded corners={1mm}]
      ($(\x,\y)+(0.2,0.2)$) rectangle ($(\x+4,\y+4)+(-0.2,-0.2)$);
    \node[green!50!black] at ($(\x,\y)+(2,3)$)
      {\tiny\textsf{Velocity}};
    \node[green!50!black] at ($(\x,\y)+(2,1.5)$) {\large\bf\textsf{\t}};
  }
  % additional key labels
  \node[blue!50!black] at (3.5,10) {\bf\textsf{sustain}};
  \node[blue!50!black] at (2.5,6) {\small\bf\textsf{$-$bend}};
  \node[blue!50!black] at (3,2) {\bf\textsf{$-$8ve}};
  \node[blue!50!black] at (54,6) {\bf\textsf{$+$bend}};
  \node[blue!50!black] at (57,2) {\bf\textsf{$+$8ve}};
  \node[blue!50!black] at (78,18) {\scriptsize\bf\textsf{layout}};
  \node[blue!50!black] at (80,2) {\small\bf\textsf{tap tempo}};
\end{tikzpicture}\par
\vspace*{60pt}
\begin{tikzpicture}[scale=0.234]
  % first row
  \foreach \xa/\xb/\tb in
    {0/4/16,
     8/12/1,12/16/2,16/20/3,20/24/4,
     26/30/5,30/34/6,34/38/7,38/42/8,
     44/48/9,48/52/10,52/56/11,56/60/12,
     62/66/13,66/70/14,70/74/15} {
    \draw[rounded corners={1mm}]
      ($(\xa,22)+(0.2,0.2)$) rectangle ($(\xb,26)+(-0.2,-0.2)$);
    \node[red!50!black] at ($(\xa,26)+(2,-1)$)
      {\tiny\textsf{Channel}};
    \node[red!50!black] at ($(\xa,22)+(2,1.5)$) {\large\bf\textsf{\tb}};
  }
  % LEDs
  \foreach \x/\t in {80/Caps,84/Scroll} {
    \draw (\x+0.6,24.75) rectangle (\x+2.1,25.25);
    \node[anchor=west] at ($(\x,26)+(0,-2)$) {\tiny\textsf{\t}};
    \node[anchor=west] at ($(\x,26)+(0,-3)$) {\tiny\textsf{Lock}};
  }
  \draw[fill=red] (76+0.6,24.75) rectangle (76+2.1,25.25);
  \node[anchor=west] at ($(76,26)+(0,-2)$) {\tiny\textsf{Num}};
  \node[anchor=west] at ($(76,26)+(0,-3)$) {\tiny\textsf{Lock}};
  % second row, purple note keys
  \foreach \xa/\xb/\ta/\tb in
    {0/4/67/{A\myFlFl},4/8/69/{B\myFlFl},48/52/91/{F\myShSh},
     52/56/93/{G\myShSh},56/60/95/{A\myShSh}} {
    \draw[rounded corners={1mm},fill=red!50!blue!80!black]
      ($(\xa,16)+(0.2,0.2)$) rectangle ($(\xb,20)+(-0.2,-0.2)$);
    \node[anchor=west,white] at ($(\xa,20)+(0,-1)$) {\tiny\bf\textsf{\ta}};
    \node[white] at ($(\xa,16)+(2,1.5)$) {\large\bf\textsf{\tb}};
  }
  % second row, black note keys
  \foreach \xa/\xb/\ta/\tb in
    {8/12/71/{C\myFl},12/16/73/{D\myFl},16/20/75/{E\myFl},
     36/40/85/{C\mySh},40/44/87/{D\mySh},44/48/89/{E\mySh}} {
    \draw[rounded corners={1mm},fill=black]
      ($(\xa,16)+(0.2,0.2)$) rectangle ($(\xb,20)+(-0.2,-0.2)$);
    \node[anchor=west,white] at ($(\xa,20)+(0,-1)$) {\tiny\bf\textsf{\ta}};
    \node[white] at ($(\xa,16)+(2,1.5)$) {\large\bf\textsf{\tb}};
  }
  % second row, white note keys
  \foreach \xa/\xb/\ta/\tb in
    {20/24/77/F,24/28/79/G,28/32/81/A,32/36/83/B} {
    \draw[rounded corners={1mm}]
      ($(\xa,16)+(0.2,0.2)$) rectangle ($(\xb,20)+(-0.2,-0.2)$);
    \node[anchor=west] at ($(\xa,20)+(0,-1)$) {\tiny\textsf{\ta}};
    \node at ($(\xa,16)+(2,1.5)$) {\large\bf\textsf{\tb}};
  }
  % second row, non-note keys
  \foreach \xa/\xb/\ta/\tb in
    {62/66/Insert/{},66/70/Home/{},70/74/Page/Up,
     76/80/{}/{},80/84/{/}/{},84/88/*/{},88/92/-/{}} {
    \draw[rounded corners={1mm}]
      ($(\xa,16)+(0.2,0.2)$) rectangle ($(\xb,20)+(-0.2,-0.2)$);
    \node[anchor=west] at ($(\xa,20)+(0,-1)$) {\tiny\textsf{\ta}};
    \node[anchor=west] at ($(\xa,20)+(0,-2)$) {\tiny\textsf{\tb}};
  }
  % third row, purple note keys
  \foreach \xa/\xb/\ta/\tb in
    {0/6/62/{E\myFlFl},50/54/86/{C\myShSh}} {
    \draw[rounded corners={1mm},fill=red!50!blue!80!black]
      ($(\xa,12)+(0.2,0.2)$) rectangle ($(\xb,16)+(-0.2,-0.2)$);
    \node[anchor=west,white] at ($(\xa,16)+(0,-1)$) {\tiny\bf\textsf{\ta}};
    \node[white] at ($(\xa,12+1.5)!0.5!(\xb,12+1.5)$) {\large\bf\textsf{\tb}};
  }
  % third row, black note keys
  \foreach \xa/\xb/\ta/\tb in
    {6/10/64/{F\myFl},10/14/66/{G\myFl},14/18/68/{A\myFl},18/22/70/{B\myFl},
     34/38/78/{F\mySh},38/42/80/{G\mySh},42/46/82/{A\mySh},46/50/84/{B\mySh}} {
    \draw[rounded corners={1mm},fill=black]
      ($(\xa,12)+(0.2,0.2)$) rectangle ($(\xb,16)+(-0.2,-0.2)$);
    \node[anchor=west,white] at ($(\xa,16)+(0,-1)$) {\tiny\bf\textsf{\ta}};
    \node[white] at ($(\xa,12+1.5)!0.5!(\xb,12+1.5)$) {\large\bf\textsf{\tb}};
  }
  % third row, white note keys
  \foreach \xa/\xb/\ta/\tb in
    {22/26/72/C,26/30/74/D,30/34/76/E} {
    \draw[rounded corners={1mm}]
      ($(\xa,12)+(0.2,0.2)$) rectangle ($(\xb,16)+(-0.2,-0.2)$);
    \node[anchor=west] at ($(\xa,16)+(0,-1)$) {\tiny\textsf{\ta}};
    \node at ($(\xa,12+1.5)!0.5!(\xb,12+1.5)$) {\large\bf\textsf{\tb}};
  }
  % third row, non-note keys
  \foreach \xa/\xb/\ta/\tb in
    {62/66/Delete/{},66/70/End/{},70/74/Page/Down} {
    \draw[rounded corners={1mm}]
      ($(\xa,12)+(0.2,0.2)$) rectangle ($(\xb,16)+(-0.2,-0.2)$);
    \node[anchor=west] at ($(\xa,16)+(0,-1)$) {\tiny\textsf{\ta}};
    \node[anchor=west] at ($(\xa,16)+(0,-2)$) {\tiny\textsf{\tb}};
  }
  % keys that span third and fourth rows
  \draw[rounded corners={1mm}]
    ($(55,8)+(0.2,0.2)$) -- ($(55,12)+(0.2,0.2)$) --
    ($(54,12)+(0.2,0.2)$) -- ($(54,16)+(0.2,-0.2)$) --
    ($(60,16)+(-0.2,-0.2)$) -- ($(60,8)+(-0.2,0.2)$) --cycle;
  \node[anchor=west] at ($(54,16)+(0,-1)$) {\tiny\textsf{Enter}};
  \draw[rounded corners={1mm}]
    ($(88,8)+(0.2,0.2)$) rectangle ($(92,16)+(-0.2,-0.2)$);
  \node[anchor=west] at ($(88,16)+(0,-1)$) {\tiny\textsf{+}};
  % fourth row, purple note keys
  \foreach \xa/\xb/\ta/\tb in
    {47/51/79/{F\myShSh},51/55/81/{G\myShSh}} {
    \draw[rounded corners={1mm},fill=red!50!blue!80!black]
      ($(\xa,8)+(0.2,0.2)$) rectangle ($(\xb,12)+(-0.2,-0.2)$);
    \node[anchor=west,white] at ($(\xa,12)+(0,-1)$) {\tiny\bf\textsf{\ta}};
    \node[white] at ($(\xa,8+1.5)!0.5!(\xb,8+1.5)$) {\large\bf\textsf{\tb}};
  }
  % fourth row, black note keys
  \foreach \xa/\xb/\ta/\tb in
    {7/11/59/{C\myFl},11/15/61/{D\myFl},15/19/63/{E\myFl},
     35/39/73/{C\mySh},39/43/75/{D\mySh},43/47/77/{E\mySh}} {
    \draw[rounded corners={1mm},fill=black]
      ($(\xa,8)+(0.2,0.2)$) rectangle ($(\xb,12)+(-0.2,-0.2)$);
    \node[anchor=west,white] at ($(\xa,12)+(0,-1)$) {\tiny\bf\textsf{\ta}};
    \node[white] at ($(\xa,8+1.5)!0.5!(\xb,8+1.5)$) {\large\bf\textsf{\tb}};
  }
  % fourth row, white note keys
  \foreach \xa/\xb/\ta/\tb in
    {0/7/{}/{},19/23/65/F,23/27/67/G,27/31/69/A,31/35/71/B} {
    \draw[rounded corners={1mm}]
      ($(\xa,8)+(0.2,0.2)$) rectangle ($(\xb,12)+(-0.2,-0.2)$);
    \node[anchor=west] at ($(\xa,12)+(0,-1)$) {\tiny\textsf{\ta}};
    \node at ($(\xa,8+1.5)!0.5!(\xb,8+1.5)$) {\large\bf\textsf{\tb}};
  }
  % fifth row, black note keys
  \foreach \xa/\xb/\ta/\tb in
    {5/9/52/{F\myFl},9/13/54/{G\myFl},13/17/56/{A\myFl},17/21/58/{B\myFl},
     33/37/66/{F\mySh},37/41/68/{G\mySh},41/45/70/{A\mySh},45/49/72/{B\mySh}} {
    \draw[rounded corners={1mm},fill=black]
      ($(\xa,4)+(0.2,0.2)$) rectangle ($(\xb,8)+(-0.2,-0.2)$);
    \node[anchor=west,white] at ($(\xa,8)+(0,-1)$) {\tiny\bf\textsf{\ta}};
    \node[white] at ($(\xa,4+1.5)!0.5!(\xb,4+1.5)$) {\large\bf\textsf{\tb}};
  }
  % fifth row, white note keys
  \foreach \xa/\xb/\ta/\tb in
    {21/25/60/C,25/29/62/D,29/33/64/E} {
    \draw[rounded corners={1mm}]
      ($(\xa,4)+(0.2,0.2)$) rectangle ($(\xb,8)+(-0.2,-0.2)$);
    \node[anchor=west] at ($(\xa,8)+(0,-1)$) {\tiny\textsf{\ta}};
    \node at ($(\xa,4+1.5)!0.5!(\xb,4+1.5)$) {\large\bf\textsf{\tb}};
  }
  % fifth row, non-note keys
  \foreach \xa/\xb/\ta/\tb in
    {0/5/{}/{},49/60/{}/{},
     66/70/{$\uparrow$}/{}} {
    \draw[rounded corners={1mm}]
      ($(\xa,4)+(0.2,0.2)$) rectangle ($(\xb,8)+(-0.2,-0.2)$);
    \node[anchor=west] at ($(\xa,8)+(0,-1)$) {\tiny\textsf{\ta}};
    \node[anchor=west] at ($(\xa,8)+(0,-2)$) {\tiny\textsf{\tb}};
  }
  % keys that span fifth and sixth rows
  \draw[rounded corners={1mm}]
    ($(88,0)+(0.2,0.2)$) rectangle ($(92,8)+(-0.2,-0.2)$);
  \node[anchor=west] at ($(88,8)+(0,-1)$) {\tiny\textsf{Enter}};
  % sixth row
  \foreach \xa/\xb/\ta/\tb in
    {0/6/{}/{},6/11/GUI/{},11/16/Alt/{},16/39/{}/{},
     39/44/Alt/{},44/49/GUI/{},49/54/Menu/{},54/60/{}/{},
     62/66/{$\leftarrow$}/{},66/70/{$\downarrow$}/{},70/74/{$\rightarrow$}/{},
     76/84/{}/{},84/88/./Del} {
    \draw[rounded corners={1mm}]
      ($(\xa,0)+(0.2,0.2)$) rectangle ($(\xb,4)+(-0.2,-0.2)$);
    \node[anchor=west] at ($(\xa,4)+(0,-1)$) {\tiny\textsf{\ta}};
    \node[anchor=west] at ($(\xa,4)+(0,-2)$) {\tiny\textsf{\tb}};
  }
  % keypad numerals
  \foreach \x/\y/\t in {76/4/14,80/4/28,84/4/42,76/8/56,80/8/70,
    84/8/84,76/12/98,80/12/112,84/12/126} {
    \draw[rounded corners={1mm}]
      ($(\x,\y)+(0.2,0.2)$) rectangle ($(\x+4,\y+4)+(-0.2,-0.2)$);
    \node[green!50!black] at ($(\x,\y)+(2,3)$)
      {\tiny\textsf{Velocity}};
    \node[green!50!black] at ($(\x,\y)+(2,1.5)$) {\large\bf\textsf{\t}};
  }
  % additional key labels
  \node[blue!50!black] at (3.5,10) {\bf\textsf{sustain}};
  \node[blue!50!black] at (2.5,6) {\small\bf\textsf{$-$bend}};
  \node[blue!50!black] at (3,2) {\bf\textsf{$-$8ve}};
  \node[blue!50!black] at (54,6) {\bf\textsf{$+$bend}};
  \node[blue!50!black] at (57,2) {\bf\textsf{$+$8ve}};
  \node[blue!50!black] at (78,18) {\scriptsize\bf\textsf{layout}};
  \node[blue!50!black] at (80,2) {\small\bf\textsf{tap tempo}};
\end{tikzpicture}\par
\vspace*{24pt}}
\caption{Keyboard layouts.}\label{fig:qwerty-layouts}
\end{sidewaysfigure*}

This layout is meant to imitate a piano's keyboard layout, with a little
over 2$\tfrac{1}{2}$ octaves of coverage.  Keys that fall into gaps of the
piano layout (shown in grey) are assigned to B$\musSharp$ and E$\musSharp$
(enharmonic to C and F) to make a consistent pattern.  The upper left of the
main letter area (Q on a QWERTY layout) is Middle~C, and the F key in a
QWERTY layout, which often has a tactile guide bump on it, is effectively an
F (shown as E$\musSharp$, but the same MIDI number).

%%%%%%%%%%%%%%%%%%%%%%%%%%%%%%%%%%%%%%%%%%%%%%%%%%%%%%%%%%%%%%%%%%%%%%%%

\section{Wicki-Hayden isomorphic layout}

Press Num Lock to switch layouts.  With the Num Lock LED glowing, the
Gracious Host implements a Wicki-Hayden isomorphic keyboard layout, as shown
in the lower half of Figure~\ref{fig:qwerty-layouts}.  This layout is named
after Kaspar Wicki and Brian Hayden, who independently invented and patented
it in 1896 and 1986 respectively.  It is similar to the layout commonly used
for concertina keyboards.

The Wicki-Hayden layout treats the keys as an hexagonal grid. 
Horizontally adjacent keys are a major second (two semitones) apart, pitch
going up from left to right across the keyboard.  From any note the
diagonally adjacent notes to the right represent the fifth of the note, in
the higher or lower octave according to whether they are diagonally up
and right or down and right.  Similarly, the diagonal notes to the left are
the fourth of the current note.  And going directly up or down two rows
corresponds to going up or down by a whole octave.

{\center\begin{tikzpicture}[scale=0.234]
  \foreach \xa/\xb/\ta/\tb in
    {22/26/72/C,26/30/74/D} {
    \draw[rounded corners={1mm}]
      ($(\xa,12)+(0.2,0.2)$) rectangle ($(\xb,16)+(-0.2,-0.2)$);
    \node[anchor=west] at ($(\xa,16)+(0,-1)$) {\tiny\textsf{\ta}};
    \node at ($(\xa,12+1.5)!0.5!(\xb,12+1.5)$) {\large\bf\textsf{\tb}};
  }
  \foreach \xa/\xb/\ta/\tb in
    {19/23/65/F,23/27/67/G,27/31/69/A} {
    \draw[rounded corners={1mm}]
      ($(\xa,8)+(0.2,0.2)$) rectangle ($(\xb,12)+(-0.2,-0.2)$);
    \node[anchor=west] at ($(\xa,12)+(0,-1)$) {\tiny\textsf{\ta}};
    \node at ($(\xa,8+1.5)!0.5!(\xb,8+1.5)$) {\large\bf\textsf{\tb}};
  }
  \foreach \xa/\xb/\ta/\tb in
    {21/25/60/C,25/29/62/D} {
    \draw[rounded corners={1mm}]
      ($(\xa,4)+(0.2,0.2)$) rectangle ($(\xb,8)+(-0.2,-0.2)$);
    \node[anchor=west] at ($(\xa,8)+(0,-1)$) {\tiny\textsf{\ta}};
    \node at ($(\xa,4+1.5)!0.5!(\xb,4+1.5)$) {\large\bf\textsf{\tb}};
  }
  \fill (22,11) circle[radius=0.4];
  \fill (30,11) circle[radius=0.4];
  \fill (28,7) circle[radius=0.4];
\end{tikzpicture}\par}

The defining feature of an \emph{isomorphic}\footnote{This word means ``same
shape.''} music keyboard is that the harmonic relationships are the same
everywhere on the keyboard.  If you memorize the shape for a given chord
such as D~minor, shown by the dots above, you can play any chord of the same
quality, such as C~minor, by shifting the same shape somewhere else on the
keyboard.  These chords would have to be learned separately on a piano,
where D~minor is played entirely on white keys and C~minor involves a black
key.  Melodies can be transposed on an isomorphic keyboard just by moving to
a different physical location without changing the fingering.

With no octave shift, the Gracious Host's Wicki-Hayden layout puts the F
and G above Middle~C on the F and G keys of a standard QWERTY layout.

%%%%%%%%%%%%%%%%%%%%%%%%%%%%%%%%%%%%%%%%%%%%%%%%%%%%%%%%%%%%%%%%%%%%%%%%

\section{Sustain}

The Caps Lock key activates the \emph{sustain} feature.  Modes like
quantizer and arpeggiator benefit from being able to play many MIDI notes at
once; but both because people usually have at most ten fingers, and because
USB typing keyboards are limited in how many simultaneous keypresses they
can handle, actually pressing many note keys at once may be a problem.  The
basic sustain function is that when Caps Lock is pressed, the associated LED
goes on, and then all note keys pressed at that time or while Caps Lock
remains pressed, are locked.  These notes remain in effect even if the note
keys are released.  When Caps Lock is pressed a second time and the LED goes
off, all the locked notes are considered released at once.

The usual performace practice would be to either hold a chord and tap
Caps Lock to lock it; or to press and hold Caps Lock, and while holding it
build up the chord one or a few notes at a time before releasing Caps Lock.

In more detail:  Notes become locked if they are playing at the moment the
Caps Lock key is first pressed to activate sustain.  They also become locked
if they are newly pressed \emph{during} that first press of Caps Lock, so
the sequence ``press Caps Lock and hold; press note key; release Caps Lock''
results in the note being locked, regardless of exactly when the note key is
released.  Once locked, notes remain locked, and no additional Note On
events will be sent, until the second Caps Lock keypress, which will turn
off the LED and send Note Off events all at once for all the locked notes
except those that may actually be pressed at the moment of the second Caps
Lock keypress (which instead will be released when the note keys are
actually released).  After the first time Caps Lock has been released, with
the LED on and some notes held, other notes can be played normally. 
Re-playing a note already held will have no additional effect -- no second
Note On will be sent -- except that a note whose key is actually pressed
when releasing sustain will not get a Note Off at the release of sustain,
but only when its key is also released.

The sustain feature remembers the current channel when it is activated.  If
you change the channel while sustain is active, notes played in the new
channel are completely separate.  New Note On messages may be sent in the
new channel, and the eventual Note Offs when sustain is released will be
sent in the original channel.  Also, sustain is associated with \emph{note
numbers}, not with \emph{physical keyboard keys}.  By changing the octave
shift or the piano/isomorphic layout selection at different points in the
use of the sustain feature, it is quite possible for the same note key to
end up causing more than one note to be locked.

%%%%%%%%%%%%%%%%%%%%%%%%%%%%%%%%%%%%%%%%%%%%%%%%%%%%%%%%%%%%%%%%%%%%%%%%

\section{Octave shift}

The two Ctrl keys on the USB keyboard control octave shift.  Press the left
Ctrl key (just press like a normal key; it is not necessary to hold it down,
or press other keys along with it) to shift down one octave.  The typing
keys that send MIDI notes will send notes one octave lower than their
default values.  Press the right Ctrl to shift one octave up.  The Scroll
Lock LED will glow (as well as possibly blinking with the beat, see ``tap
tempo'' below) whenever an octave shift in either direction is active.

Pressing the shift keys again, shifts further.  Shifting up to five octaves
down or four octaves up is allowed, those limits being chosen to allow
covering the range of MIDI notes 1 to 127 in either keyboard layout, even on
a US-style keyboard with relatively few keys.  However, in practice it will
seldom be useful to play notes outside the range 24\ldots 96, which
corresponds to the range of the control voltage output DACs.

%%%%%%%%%%%%%%%%%%%%%%%%%%%%%%%%%%%%%%%%%%%%%%%%%%%%%%%%%%%%%%%%%%%%%%%%

\section{Pitch bend}

The right and left Shift keys send MIDI pitch bend messages.  The pitch will
bend up or down while you hold either Shift key (cancelling out, if both) up
to its default limit of two semitones, then will return once the Shift key
is released.  The speed of bend and return is fixed in the firmware; to
achieve finer control of pitch bend you need a proper MIDI controller
with this feature.

%%%%%%%%%%%%%%%%%%%%%%%%%%%%%%%%%%%%%%%%%%%%%%%%%%%%%%%%%%%%%%%%%%%%%%%%

\section{Channel selection}

Each function key (F1, F2, and so on) corresponds to the MIDI channel with
the same number.  Press one of these keys to set the channel on which the
typing keyboard will send subsequent MIDI events.  See the previous chapter
of this manual for descriptions of the different channels.  You can, for
instance, press F2 to switch to duophonic mode, or F10 to send drum
triggers.  Channel~1 is the default when the typing keyboard is first
connected, before any function key has been pressed.

The remaining keys in the function-key row select the remaining channels:
Print Screen, Scroll Lock, and Pause/Break (at the right of the row)
correspond to Channels~13, 14, and~15 as if they were F13, F14, and F15; and
Esc (usually at the left of the row, though it appears elsewhere on some
keyboards) to Channel~16 as if it were F16.  However, in the current
firmware as of this writing, channels beyond~12 do not actually do
anything.

Note that the MIDI subsystem determines its mode from \emph{the channel of
the most recent Note On message}.  Just pressing a function key to change
channels will not change the MIDI subsystem's mode; you must press a note
key to send a Note On before the MIDI subsystem will change modes.  The
concept here is that the module implements a MIDI to CV interface.  When you
plug in a typing keyboard, the typing keyboard is just a funny-looking MIDI
master keyboard plugged into the interface.  The function keys are
configuration commands to the keyboard regarding what channel it should send
MIDI messages on in the future; they are not MIDI messages in themselves. 
The MIDI subsystem, to the extent possible, responds to MIDI messages from a
typing keyboard just the same way it would respond to MIDI messages from a
music keyboard.

%%%%%%%%%%%%%%%%%%%%%%%%%%%%%%%%%%%%%%%%%%%%%%%%%%%%%%%%%%%%%%%%%%%%%%%%

\section{Velocity}

The keypad numerals 1 through 9 set the velocity for MIDI Note On events. 
Press any of these to choose the velocity for any subsequent notes.  The
values are as shown in the keyboard layout diagram, equal to the keycap
numeral value times 14, to provide equally spaced values throughout the MIDI
range.  Velocity is only relevant to Channel~1 (where it will be a control
voltage appearing on the right analog output) with the default firmware, but
some customized or future firmware might use velocity in other channels in
some way.

%%%%%%%%%%%%%%%%%%%%%%%%%%%%%%%%%%%%%%%%%%%%%%%%%%%%%%%%%%%%%%%%%%%%%%%%

\section{Tap tempo}

The keypad Insert (0) key functions as \emph{tap tempo}.  This allows the
performer to establish a clock for the arpeggiator and sync modes without
needing to patch in a clock control voltage signal.  In some channel modes
the resulting clock appears as a control voltage output and can be used to
control other modules.  Although there are some differences in the internal
implementation, the tap tempo feature basically serves the same purpose as
MIDI Timing Clock messages (24 of those per tap).

Press the tap tempo key at least twice, on the desired beat, to start the
tempo clock.  The Scroll Lock LED will blink on the beat (overlaid on its
solid glow, if octave shift is active).  If you enter three or more taps, in
a reasonably consistent straight rhythm, the tempo will be determined by an
exponential moving average of the most recent timings; that allows for a
more precise fit than would be possible by using only the two most recent,
given the limited precision of USB keyboard timing.  A tap that is not close
to the established timing, or that seems to have skipped at least one beat,
will be treated as the start of a new tap tempo sequence, stopping the old
clock.

%%%%%%%%%%%%%%%%%%%%%%%%%%%%%%%%%%%%%%%%%%%%%%%%%%%%%%%%%%%%%%%%%%%%%%%%

\section{Maintenance codes}

It is possible to activate special features, mostly intended for firmware
testing, by entering a four-digit decimal number through the typing
keyboard driver.  The only maintenance code that most module owners will
find really useful is 5833, to run the calibration procedure.  But for
completeness, here is a list of all the codes supported by production
firmware.

\begin{description}
\item[5833] Run the calibration procedure.
\item[1240] Simulate USB hub insertion (blinks ``H'' in Morse code).
\item[3627] Simulate insertion of an unsupported USB device (blinks ``D'' in
Morse code).
\item[4935] Perform the ``success'' display (normally done as the result of a
completed calibration).
\item[6697] Perform the ``failure'' display (normally the result of an aborted
calibration).
\item[8189] Throw a driver exception (flashes lights and waits for USB
disconnect).
\item[8605] Simulate a power-on reset.
\end{description}

Special firmware assembled with test routines included will also accept a
few other codes to run the test routines, but those will have no effect on
standard production firmware.  See the \emph{Gracious Host Programmer's
Manual} for details on compiling test firmware, and how to
create your own maintenance codes.

To enter a maintenance code:

\begin{itemize}
\item Attach a typing keyboard to the module.
\item Press and hold one (either) of the Ctrl keys and one of the Alt keys.
\item While holding Ctrl and Alt, type out the four digits of the
maintenance code, on the numeric keypad.
\end{itemize}
