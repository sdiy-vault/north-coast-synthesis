% $Id: midi.tex 9713 2021-12-18 19:22:35Z mskala $

%
% MIDI interface
% Copyright (C) 2022  Matthew Skala
%
% This program is free software: you can redistribute it and/or modify
% it under the terms of the GNU General Public License as published by
% the Free Software Foundation, version 3.
%
% This program is distributed in the hope that it will be useful,
% but WITHOUT ANY WARRANTY; without even the implied warranty of
% MERCHANTABILITY or FITNESS FOR A PARTICULAR PURPOSE.  See the
% GNU General Public License for more details.
%
% You should have received a copy of the GNU General Public License
% along with this program.  If not, see <http://www.gnu.org/licenses/>.
%
% Matthew Skala
% https://northcoastsynthesis.com/
% mskala@northcoastsynthesis.com
%

\chapter{MIDI interface}

The Gracious Host's MIDI subsystem handles connections to USB MIDI devices
like (musical) keyboards, DIN-to-USB cables, and synthesizers.  The same
backend also processes MIDI events generated by the typing (QWERTY) keyboard
driver.

%%%%%%%%%%%%%%%%%%%%%%%%%%%%%%%%%%%%%%%%%%%%%%%%%%%%%%%%%%%%%%%%%%%%%%%%

\section{USB compatibility}

The usual pattern with USB standards is that the governing organization
defines a complicated protocol intended to capture every possible strange
feature that an unusual device might support; and a simplified version
representing the lowest common denominator.  Then most implementors just use
the simplified version, and the few who are building unusual devices that
really require a complicated protocol, ignore the official one anyway and
make up their own proprietary interfaces to use instead, which they don't
document, so you can only access those devices through reverse engineering
or with the manufacturers' proprietary drivers and on the operating systems
they support.  USB MIDI is no exception.

The Gracious Host will work with most \emph{class compliant} USB MIDI
devices in the real world.  Technically, that means USB devices which expose
an interface of class 1 subclass 3 and one BULK IN endpoint for MIDI stream
data.  Things like DIN MIDI to USB cables, controller keyboards, and so on,
are likely to work.  The Gracious Host is only intended to receive, not
send, MIDI data from the USB port, so USB devices like synthesizers will
probably work to the extent they can be used as MIDI controllers, but not as
synthesizers per se.  Devices that combine multiple
synthesizers, audio interfaces, MIDI ports, and other things in a single
product, are iffy.

Devices that are not \emph{class compliant} will not work.  In particular, a
device that requires a special software driver of its own to work with a
personal computer and cannot be used with a PC operating system's generic
driver for that category of device, is likely not to work with the Gracious
Host.

The Roland UM-ONE DIN to USB MIDI cable has a switch on it for ``TAB'' or
``COMP.'' The vendor documents this switch as selecting whether you want to
plug it into a ``tablet'' or a ``computer,'' but what it really does is it
selects class compliance:  in ``TAB'' mode the cable is class compliant and
will work with the Gracious Host, and in ``COMP'' mode it isn't and won't. 
So you should always use it in ``TAB'' mode with the Gracious Host.  I think
requiring a manual selection between the two interfaces (instead of exposing
them both for automatic selection through the USB auto-configuration
mechanism) is certainly a violation of the spirit, and maybe also the
letter, of the USB standards on Roland's part.  Other devices probably have
similar issues, so if you are trying to connect some device similar in
nature to a UM-ONE and find it doesn't work with the Gracious Host, try
looking for any kind of compatibility mode switch or similar configuration
setting, and change it.

I have an Akai MPK Mini keyboard that is basically class compliant, except
it sends 64 bytes of apparently random garbage through the MIDI interface
every time it boots up.  The Gracious Host will ignore that burst, as well
as any similar behaviour from other devices.  I'm not sure Akai still sells
this product anymore, and if they do, it may well be that the current
version has been updated not to send bursts of garbage.

The Gracious Host does not support the new features introduced in USB MIDI
2.0.  Since all USB MIDI 2.0 class compliant devices are required to also
support USB MIDI 1.0 connections, the new version should not create any
compatibility problems with the Gracious Host.  The USB MIDI 2.0 standard
does remove some of the special features of USB MIDI 1.0 for complicated
devices that the Gracious Host did not support anyway (and, it's clear,
almost nobody supported).

%%%%%%%%%%%%%%%%%%%%%%%%%%%%%%%%%%%%%%%%%%%%%%%%%%%%%%%%%%%%%%%%%%%%%%%%

\section{General comments on MIDI}

The Gracious Host's function is usually determined by the channel of the
most recent MIDI event it has received.  The channels are not fully
independent and, with the exception of pairs designed to work together like
Channels~8 and~9, it usually will not work well to send messages to multiple
channels at once.  Some data structures in the firmware are shared, so you
may find for instance that changing the pitch bend on Channel~1 also has the
effect of changing it on Channel~3.

The Gracious Host supports only those MIDI messages that are directly
relevant to its functions as described in this documentation.  Some features
of MIDI not relevant to the Gracious Host, like Omni Mode, System Exclusive
messages, and so on, will be ignored even if the MIDI organization has
declared them mandatory.

Running Status is supported.

Pitch Bend messages are supported, and the range is $\pm$2 semitones by
default.  Support for adjustable pitch bend range (MIDI Registered Parameter
Number 0) is planned but not yet implemented.

New features of MIDI~2.0 are not supported.

The Gracious Host (regardless of channel mode) will process the MIDI Timing
Clock message, which defines a 24 PPQN clock, in order to set the tempo used
for things like the arpeggiator channels.  This clock is also available on
an output jack, or similar clocks are accepted on input jacks, in some
channel modes; and when using a typing keyboard (described in the next
chapter) the Scroll Lock LED blinks in time with the MIDI clock and the
keypad Insert key can be used as a tap tempo input.  The module will process
the MIDI Start message as a reset to the start of the current quarter note,
but it ignores other MIDI sequencer-control messages (Stop, Continue, Song
Position Pointer, and Song Select).

%%%%%%%%%%%%%%%%%%%%%%%%%%%%%%%%%%%%%%%%%%%%%%%%%%%%%%%%%%%%%%%%%%%%%%%%

\section{Channel 1:  mono CV/gate with velocity and square wave}

In this mode the Gracious Host can control a single channel of a CV/gate
synthesizer.  Note On and Off messages translate to pitch and velocity
control voltages with a nominal 0V to 5V range representing MIDI notes 36 to
96 with any relevant pitch bend, and velocity values 0 to 127.  Both LEDs
glow green when a note is active.  The righthand gate output jack
generates a square wave (low value 0V, high value 9V nominal) at the
frequency corresponding to the selected MIDI note.  Note that while in this
mode (that is, when the most recent note played was on Channel 1), the
oscillator continues running indefinitely at the frequency of the most
recent note even after the note has ended and taken the gate low.

{\centering
\begin{tikzpicture}[scale=0.8]
  {\setmainfont[Path={../../tsukurimashou/otf/},
BoldFont={TsukurimashouBokukkoExtraBoldPS}]{TsukurimashouBokukkoDemiboldPS}%
% start out defining coordinates
\coordinate (lbh) at (4.91mm,31.23mm) {};
\coordinate (j1) at (8.72mm,59.17mm) {};
\coordinate (j2) at (27.77mm,59.17mm) {};
\coordinate (j3) at (8.72mm,40.12mm) {};
\coordinate (j4) at (27.77mm,40.12mm) {};
\coordinate (j5) at (8.72mm,21.07mm) {};
\coordinate (j6) at (27.77mm,21.07mm) {};
\coordinate (d5) at (8.72mm,50.28mm) {};
\coordinate (d6) at (27.77mm,50.28mm) {};
%
\draw (0,11.07mm) -- (0,69.07mm);
\draw (40.30mm,11.07mm) -- (40.30mm,69.07mm);
%
\draw[very thick,black] (j1) -- (j3);
\draw[very thick,black] (j2) -- (j4);
\draw[very thick,black,fill=blue!30!black!20!white,rounded corners=3.0mm]
  (40.30mm,47.12mm) -- (3.72mm,47.12mm) --
  (3.72mm,14.07mm) -- (40.33mm,14.07mm);
\node[black!30!white] at ($(j1)!0.5!(j2)$) {\large IN};
\node[white] at ($(j3)!0.5!(j4)$) {\large\textbf{CV}};
\node[white] at ($(j5)!0.5!(j6)$) {\large\textbf{GT}};
%
% board-to-panel mounting holes, to clear M3 machine screw
\draw (lbh) circle[radius=1.60mm];
% six jacks with M6 threads, 6.3mm holes
\draw (j1) circle[radius=3.15mm];
\draw (j2) circle[radius=3.15mm];
\draw (j3) circle[radius=3.15mm];
\draw (j4) circle[radius=3.15mm];
\draw (j5) circle[radius=3.15mm];
\draw (j6) circle[radius=3.15mm];
% two LEDs, 5.20mm holes
\draw (d5) circle[radius=2.60mm];
\draw (d6) circle[radius=2.60mm];
%
% mock up screw heads, knobs
% machine screws with 6mm heads
\foreach \screwname in {lbh} {
  \path[draw=black,shading=ball,
    left color=black!10!white,right color=black!60!white]
    (\screwname) circle[radius=3mm];
  \draw ($(\screwname)+(50:3.0mm)$)--($(\screwname)+(220:3.0mm)$);
  \draw ($(\screwname)+(40:3.0mm)$)--($(\screwname)+(230:3.0mm)$);
}
% jacks with knurled nuts
\foreach \jackname in {j1,j2,j3,j4,j5,j6} {
  \path[draw=black,fill=black!20!white,
    decorate,decoration={snake,amplitude=0.6,segment length=2.5}]
    ($(\jackname)+(0,-0.15mm)$) circle[radius=4mm];
  \path[draw=black,fill=black!15!white] (\jackname) circle[radius=3mm];
  \path[draw=black,fill=black] (\jackname) circle[radius=2mm];
}
}

%
  \path[fill=green!60!white] (d5) circle[radius=2.50mm];
  \path[fill=green!60!white] (d6) circle[radius=2.50mm];
  \draw[very thick,-{Stealth[scale=1.2]}]
    ($(j1)+(-15mm,0)$) -- ($(j1)+(-5mm,0)$);
  \draw[very thick,-{Stealth[scale=1.2]}]
    ($(j2)+(18mm,0)$) -- ($(j2)+(5mm,0)$);
  \draw[very thick,{Stealth[scale=1.2]}-]
    ($(d5)+(-15mm,0)$) -- ($(d5)+(-4mm,0)$);
%  \draw[very thick,{Stealth[scale=1.2]}-]
%    ($(d6)+(18mm,0)$) -- ($(d6)+(4mm,0)$);
  \draw[rounded corners=4mm,very thick,dashed]
    ($(d5)+(-4mm,-4mm)$) rectangle ($(d6)+(4mm,4mm)$);
  \draw[very thick,{Stealth[scale=1.2]}-]
    ($(j3)+(-15mm,0)$) -- ($(j3)+(-5mm,0)$);
  \draw[very thick,{Stealth[scale=1.2]}-]
    ($(j4)+(18mm,0)$) -- ($(j4)+(5mm,0)$);
  \draw[very thick,{Stealth[scale=1.2]}-]
    ($(j5)+(-15mm,0)$) -- ($(j5)+(-5mm,0)$);
  \draw[very thick,{Stealth[scale=1.2]}-]
    ($(j6)+(18mm,0)$) -- ($(j6)+(5mm,0)$);
%
  \node[anchor=east] at ($(j1)+(-15mm,0)$) {24 PPQN};
  \node[anchor=west] at ($(j2)+(18mm,0)$) {1 PPQN};
  \node[anchor=east] at ($(d5)+(-15mm,0)$) {gate};
  \node[anchor=east] at ($(j3)+(-15mm,0)$) {pitch CV};
  \node[anchor=west] at ($(j4)+(18mm,0)$) {velocity};
  \node[anchor=east] at ($(j5)+(-15mm,0)$) {gate};
  \node[anchor=west] at ($(j6)+(18mm,0)$)
    {\parbox{12mm}{square wave}};
\end{tikzpicture}\par}

Channel 1 implements monophonic note stealing.  If a new note starts while
an old one is in progress, then the old note immediately ends and is
replaced by the new one.  In such a case the gate output goes low for
approximately 1ms to signal the start of the new note.

A stolen note ends when it is stolen.  For instance, if you play Note On C,
Note On G, Note Off G, then the gate goes low and the oscillator continues
playing G.  The C does not come back when the G ends, even if you still
have not yet played Note Off C.

Although in this mode the module does not \emph{use} its tempo clock, it can
\emph{accept} clock signals on the input jacks.  The left input jack accepts
a standard 24~PPQN MIDI clock.  The right input jack will function as a
1~PPQN or tap tempo input if there is nothing received on the left
input, or as a synchronization or reset input (marking the start of a
quarter note) when there is also a 24~PPQN clock being received on the left.

%%%%%%%%%%%%%%%%%%%%%%%%%%%%%%%%%%%%%%%%%%%%%%%%%%%%%%%%%%%%%%%%%%%%%%%%

\section{Channel 2:  duophonic CV/gate}

This mode is intended for controlling a synthesizer with two more or less
identical voices.  Each voice has a pitch CV and a gate CV, which respond to
Note On and Off messages for up to two simultaneous notes in the same MIDI
channel.

{\centering
\begin{tikzpicture}[scale=0.8]
  {\setmainfont[Path={../../tsukurimashou/otf/},
BoldFont={TsukurimashouBokukkoExtraBoldPS}]{TsukurimashouBokukkoDemiboldPS}%
% start out defining coordinates
\coordinate (lbh) at (4.91mm,31.23mm) {};
\coordinate (j1) at (8.72mm,59.17mm) {};
\coordinate (j2) at (27.77mm,59.17mm) {};
\coordinate (j3) at (8.72mm,40.12mm) {};
\coordinate (j4) at (27.77mm,40.12mm) {};
\coordinate (j5) at (8.72mm,21.07mm) {};
\coordinate (j6) at (27.77mm,21.07mm) {};
\coordinate (d5) at (8.72mm,50.28mm) {};
\coordinate (d6) at (27.77mm,50.28mm) {};
%
\draw (0,11.07mm) -- (0,69.07mm);
\draw (40.30mm,11.07mm) -- (40.30mm,69.07mm);
%
\draw[very thick,black] (j1) -- (j3);
\draw[very thick,black] (j2) -- (j4);
\draw[very thick,black,fill=blue!30!black!20!white,rounded corners=3.0mm]
  (40.30mm,47.12mm) -- (3.72mm,47.12mm) --
  (3.72mm,14.07mm) -- (40.33mm,14.07mm);
\node[black!30!white] at ($(j1)!0.5!(j2)$) {\large IN};
\node[white] at ($(j3)!0.5!(j4)$) {\large\textbf{CV}};
\node[white] at ($(j5)!0.5!(j6)$) {\large\textbf{GT}};
%
% board-to-panel mounting holes, to clear M3 machine screw
\draw (lbh) circle[radius=1.60mm];
% six jacks with M6 threads, 6.3mm holes
\draw (j1) circle[radius=3.15mm];
\draw (j2) circle[radius=3.15mm];
\draw (j3) circle[radius=3.15mm];
\draw (j4) circle[radius=3.15mm];
\draw (j5) circle[radius=3.15mm];
\draw (j6) circle[radius=3.15mm];
% two LEDs, 5.20mm holes
\draw (d5) circle[radius=2.60mm];
\draw (d6) circle[radius=2.60mm];
%
% mock up screw heads, knobs
% machine screws with 6mm heads
\foreach \screwname in {lbh} {
  \path[draw=black,shading=ball,
    left color=black!10!white,right color=black!60!white]
    (\screwname) circle[radius=3mm];
  \draw ($(\screwname)+(50:3.0mm)$)--($(\screwname)+(220:3.0mm)$);
  \draw ($(\screwname)+(40:3.0mm)$)--($(\screwname)+(230:3.0mm)$);
}
% jacks with knurled nuts
\foreach \jackname in {j1,j2,j3,j4,j5,j6} {
  \path[draw=black,fill=black!20!white,
    decorate,decoration={snake,amplitude=0.6,segment length=2.5}]
    ($(\jackname)+(0,-0.15mm)$) circle[radius=4mm];
  \path[draw=black,fill=black!15!white] (\jackname) circle[radius=3mm];
  \path[draw=black,fill=black] (\jackname) circle[radius=2mm];
}
}

%
  \path[fill=red!90!black] (d5) circle[radius=2.50mm];
  \path[fill=red!90!black] (d6) circle[radius=2.50mm];
%  \draw[very thick,-{Stealth[scale=1.2]}]
%    ($(j1)+(-15mm,0)$) -- ($(j1)+(-5mm,0)$);
%  \draw[very thick,-{Stealth[scale=1.2]}]
%    ($(j2)+(18mm,0)$) -- ($(j2)+(5mm,0)$);
  \draw[very thick,{Stealth[scale=1.2]}-]
    ($(d5)+(-15mm,0)$) -- ($(d5)+(-4mm,0)$);
  \draw[very thick,{Stealth[scale=1.2]}-]
    ($(d6)+(18mm,0)$) -- ($(d6)+(4mm,0)$);
  \draw[very thick,{Stealth[scale=1.2]}-]
    ($(j3)+(-15mm,0)$) -- ($(j3)+(-5mm,0)$);
  \draw[very thick,{Stealth[scale=1.2]}-]
    ($(j4)+(18mm,0)$) -- ($(j4)+(5mm,0)$);
  \draw[very thick,{Stealth[scale=1.2]}-]
    ($(j5)+(-15mm,0)$) -- ($(j5)+(-5mm,0)$);
  \draw[very thick,{Stealth[scale=1.2]}-]
    ($(j6)+(18mm,0)$) -- ($(j6)+(5mm,0)$);
%
  \node[anchor=east] at ($(d5)+(-15mm,0)$) {gate};
  \node[anchor=west] at ($(d6)+(18mm,0)$) {gate};
  \node[anchor=east] at ($(j3)+(-15mm,0)$) {pitch CV};
  \node[anchor=west] at ($(j4)+(18mm,0)$) {pitch CV};
  \node[anchor=east] at ($(j5)+(-15mm,0)$) {gate};
  \node[anchor=west] at ($(j6)+(18mm,0)$) {gate};
\end{tikzpicture}\par}

Channel~2 implements note stealing according to these rules:
\begin{itemize}
\item The first note goes to the left.
\item If exactly one side is free, a new note goes there.
\item Otherwise (both sides free, or neither), a new note goes to the side
that least recently had a new note.
\end{itemize}

As with Channel~1, the gate CV drops for about 1ms when a new note replaces
an old one, so as to give some chance of retriggering whatever module is
using that gate signal.  Also as in Channel~1, stolen notes end when they
are stolen.

Signals on the input jacks are ignored in this mode, and the LED on each
side glows red when there is a note active on the side in question.  Pitch
bend in this channel affects both sides.

%%%%%%%%%%%%%%%%%%%%%%%%%%%%%%%%%%%%%%%%%%%%%%%%%%%%%%%%%%%%%%%%%%%%%%%%

\section{Channels 3 and 4:  quantize to MIDI}

When it receives MIDI notes in either of these two channels, the Gracious
Host acts as a dual quantizer.  The input voltage on each side is compared
to whichever MIDI notes are currently held, and the output pitch control
voltages are set to the voltages for the notes closest to the inputs.

{\centering
\begin{tikzpicture}[scale=0.8]
  {\setmainfont[Path={../../tsukurimashou/otf/},
BoldFont={TsukurimashouBokukkoExtraBoldPS}]{TsukurimashouBokukkoDemiboldPS}%
% start out defining coordinates
\coordinate (lbh) at (4.91mm,31.23mm) {};
\coordinate (j1) at (8.72mm,59.17mm) {};
\coordinate (j2) at (27.77mm,59.17mm) {};
\coordinate (j3) at (8.72mm,40.12mm) {};
\coordinate (j4) at (27.77mm,40.12mm) {};
\coordinate (j5) at (8.72mm,21.07mm) {};
\coordinate (j6) at (27.77mm,21.07mm) {};
\coordinate (d5) at (8.72mm,50.28mm) {};
\coordinate (d6) at (27.77mm,50.28mm) {};
%
\draw (0,11.07mm) -- (0,69.07mm);
\draw (40.30mm,11.07mm) -- (40.30mm,69.07mm);
%
\draw[very thick,black] (j1) -- (j3);
\draw[very thick,black] (j2) -- (j4);
\draw[very thick,black,fill=blue!30!black!20!white,rounded corners=3.0mm]
  (40.30mm,47.12mm) -- (3.72mm,47.12mm) --
  (3.72mm,14.07mm) -- (40.33mm,14.07mm);
\node[black!30!white] at ($(j1)!0.5!(j2)$) {\large IN};
\node[white] at ($(j3)!0.5!(j4)$) {\large\textbf{CV}};
\node[white] at ($(j5)!0.5!(j6)$) {\large\textbf{GT}};
%
% board-to-panel mounting holes, to clear M3 machine screw
\draw (lbh) circle[radius=1.60mm];
% six jacks with M6 threads, 6.3mm holes
\draw (j1) circle[radius=3.15mm];
\draw (j2) circle[radius=3.15mm];
\draw (j3) circle[radius=3.15mm];
\draw (j4) circle[radius=3.15mm];
\draw (j5) circle[radius=3.15mm];
\draw (j6) circle[radius=3.15mm];
% two LEDs, 5.20mm holes
\draw (d5) circle[radius=2.60mm];
\draw (d6) circle[radius=2.60mm];
%
% mock up screw heads, knobs
% machine screws with 6mm heads
\foreach \screwname in {lbh} {
  \path[draw=black,shading=ball,
    left color=black!10!white,right color=black!60!white]
    (\screwname) circle[radius=3mm];
  \draw ($(\screwname)+(50:3.0mm)$)--($(\screwname)+(220:3.0mm)$);
  \draw ($(\screwname)+(40:3.0mm)$)--($(\screwname)+(230:3.0mm)$);
}
% jacks with knurled nuts
\foreach \jackname in {j1,j2,j3,j4,j5,j6} {
  \path[draw=black,fill=black!20!white,
    decorate,decoration={snake,amplitude=0.6,segment length=2.5}]
    ($(\jackname)+(0,-0.15mm)$) circle[radius=4mm];
  \path[draw=black,fill=black!15!white] (\jackname) circle[radius=3mm];
  \path[draw=black,fill=black] (\jackname) circle[radius=2mm];
}
}

%
  \path[fill=green!60!white] (d5) circle[radius=2.50mm];
  \path[fill=green!60!white] (d6) circle[radius=2.50mm];
  \path[fill=red!90!black] (d5) circle[radius=1.0mm];
  \path[fill=red!90!black] (d6) circle[radius=1.0mm];
  \draw[very thick,-{Stealth[scale=1.2]}]
    ($(j1)+(-15mm,0)$) -- ($(j1)+(-5mm,0)$);
  \draw[very thick,-{Stealth[scale=1.2]}]
    ($(j2)+(18mm,0)$) -- ($(j2)+(5mm,0)$);
  \draw[very thick,{Stealth[scale=1.2]}-]
    ($(d5)+(-15mm,0)$) -- ($(d5)+(-4mm,0)$);
  \draw[very thick,{Stealth[scale=1.2]}-]
    ($(d6)+(18mm,0)$) -- ($(d6)+(4mm,0)$);
  \draw[very thick,{Stealth[scale=1.2]}-]
    ($(j3)+(-15mm,0)$) -- ($(j3)+(-5mm,0)$);
  \draw[very thick,{Stealth[scale=1.2]}-]
    ($(j4)+(18mm,0)$) -- ($(j4)+(5mm,0)$);
  \draw[very thick,{Stealth[scale=1.2]}-]
    ($(j5)+(-15mm,0)$) -- ($(j5)+(-5mm,0)$);
  \draw[very thick,{Stealth[scale=1.2]}-]
    ($(j6)+(18mm,0)$) -- ($(j6)+(5mm,0)$);
%
  \node[anchor=east] at ($(j1)+(-15mm,0)$) {pitch CV};
  \node[anchor=west] at ($(j2)+(18mm,0)$) {pitch CV};
  \node[anchor=east] at ($(d5)+(-15mm,0)$)
    {\parbox{12mm}{active/ match}};
  \node[anchor=west] at ($(d6)+(18mm,0)$)
    {\parbox{12mm}{active/ match}};
  \node[anchor=east] at ($(j3)+(-15mm,0)$) {pitch CV};
  \node[anchor=west] at ($(j4)+(18mm,0)$) {pitch CV};
  \node[anchor=east] at ($(j5)+(-15mm,0)$) {gate};
  \node[anchor=west] at ($(j6)+(18mm,0)$) {gate};
\end{tikzpicture}\par}

If, once in this mode, there happen to be no MIDI notes held, then all four
output voltages go to zero and the LEDs go dark.  Otherwise, the LED on each
side is red when the input happens to hit a quantized output value exactly
(that is, within the same semitone), and green otherwise.  Each gate output
is normally high when any notes are held, but it drops low for approximately
1ms when the associated pitch output goes to a new note, in order to allow
for retriggering of a synthesizer voice.

The difference between the two channels is that with Channel~3, quantization
is just to the received MIDI notes and no others.  With Channel~4, each
MIDI note counts as if you played it \emph{in every octave}.  So if you
play MIDI notes 60 and 69 (Middle C and the A above it, equivalent to output
voltages 2.00V and 2.75V) in Channel~3, then an input voltage of 0.50V
(equivalent to MIDI note 42) will quantize to 2.00V (MIDI note 60); but with
the same notes played in Channel~4, an input voltage of 0.50V will quantize
to 0.75V (MIDI note 45) because playing any C and A activates every C and A
as a possible quantization value.

Pitch bend, if nonzero, is applied to both outputs equally after
quantization; it does not affect the input quantization boundaries.  Some
future version of the firmware may attempt to do something more intelligent
with pitch bend, such as using it to support microtonal quantization.

%%%%%%%%%%%%%%%%%%%%%%%%%%%%%%%%%%%%%%%%%%%%%%%%%%%%%%%%%%%%%%%%%%%%%%%%

\section{Channels 5--7:  arpeggiator modes}

MIDI notes on these three channels will be arpeggiated to the CV and gate
outputs, in different styles depending on the channel.

{\centering
\begin{tikzpicture}[scale=0.8]
  {\setmainfont[Path={../../tsukurimashou/otf/},
BoldFont={TsukurimashouBokukkoExtraBoldPS}]{TsukurimashouBokukkoDemiboldPS}%
% start out defining coordinates
\coordinate (lbh) at (4.91mm,31.23mm) {};
\coordinate (j1) at (8.72mm,59.17mm) {};
\coordinate (j2) at (27.77mm,59.17mm) {};
\coordinate (j3) at (8.72mm,40.12mm) {};
\coordinate (j4) at (27.77mm,40.12mm) {};
\coordinate (j5) at (8.72mm,21.07mm) {};
\coordinate (j6) at (27.77mm,21.07mm) {};
\coordinate (d5) at (8.72mm,50.28mm) {};
\coordinate (d6) at (27.77mm,50.28mm) {};
%
\draw (0,11.07mm) -- (0,69.07mm);
\draw (40.30mm,11.07mm) -- (40.30mm,69.07mm);
%
\draw[very thick,black] (j1) -- (j3);
\draw[very thick,black] (j2) -- (j4);
\draw[very thick,black,fill=blue!30!black!20!white,rounded corners=3.0mm]
  (40.30mm,47.12mm) -- (3.72mm,47.12mm) --
  (3.72mm,14.07mm) -- (40.33mm,14.07mm);
\node[black!30!white] at ($(j1)!0.5!(j2)$) {\large IN};
\node[white] at ($(j3)!0.5!(j4)$) {\large\textbf{CV}};
\node[white] at ($(j5)!0.5!(j6)$) {\large\textbf{GT}};
%
% board-to-panel mounting holes, to clear M3 machine screw
\draw (lbh) circle[radius=1.60mm];
% six jacks with M6 threads, 6.3mm holes
\draw (j1) circle[radius=3.15mm];
\draw (j2) circle[radius=3.15mm];
\draw (j3) circle[radius=3.15mm];
\draw (j4) circle[radius=3.15mm];
\draw (j5) circle[radius=3.15mm];
\draw (j6) circle[radius=3.15mm];
% two LEDs, 5.20mm holes
\draw (d5) circle[radius=2.60mm];
\draw (d6) circle[radius=2.60mm];
%
% mock up screw heads, knobs
% machine screws with 6mm heads
\foreach \screwname in {lbh} {
  \path[draw=black,shading=ball,
    left color=black!10!white,right color=black!60!white]
    (\screwname) circle[radius=3mm];
  \draw ($(\screwname)+(50:3.0mm)$)--($(\screwname)+(220:3.0mm)$);
  \draw ($(\screwname)+(40:3.0mm)$)--($(\screwname)+(230:3.0mm)$);
}
% jacks with knurled nuts
\foreach \jackname in {j1,j2,j3,j4,j5,j6} {
  \path[draw=black,fill=black!20!white,
    decorate,decoration={snake,amplitude=0.6,segment length=2.5}]
    ($(\jackname)+(0,-0.15mm)$) circle[radius=4mm];
  \path[draw=black,fill=black!15!white] (\jackname) circle[radius=3mm];
  \path[draw=black,fill=black] (\jackname) circle[radius=2mm];
}
}

%
  \path[fill=green!60!white] (d5) circle[radius=2.50mm];
  \path[fill=red!90!black] (d6) circle[radius=2.50mm];
  \draw[very thick,-{Stealth[scale=1.2]}]
    ($(j1)+(-15mm,0)$) -- ($(j1)+(-5mm,0)$);
  \draw[very thick,-{Stealth[scale=1.2]}]
    ($(j2)+(18mm,0)$) -- ($(j2)+(5mm,0)$);
  \draw[very thick,{Stealth[scale=1.2]}-]
    ($(d5)+(-15mm,0)$) -- ($(d5)+(-4mm,0)$);
  \draw[very thick,{Stealth[scale=1.2]}-]
    ($(d6)+(18mm,0)$) -- ($(d6)+(4mm,0)$);
  \draw[very thick,{Stealth[scale=1.2]}-]
    ($(j3)+(-15mm,0)$) -- ($(j3)+(-5mm,0)$);
  \draw[very thick,{Stealth[scale=1.2]}-]
    ($(j4)+(18mm,0)$) -- ($(j4)+(5mm,0)$);
  \draw[very thick,{Stealth[scale=1.2]}-]
    ($(j5)+(-15mm,0)$) -- ($(j5)+(-5mm,0)$);
  \draw[very thick,{Stealth[scale=1.2]}-]
    ($(j6)+(18mm,0)$) -- ($(j6)+(5mm,0)$);
%
  \node[anchor=east] at ($(j1)+(-15mm,0)$) {24 PPQN};
  \node[anchor=west] at ($(j2)+(18mm,0)$) {1 PPQN};
  \node[anchor=east] at ($(d5)+(-15mm,0)$) {active};
  \node[anchor=west] at ($(d6)+(18mm,0)$) {gate};
  \node[anchor=east] at ($(j3)+(-15mm,0)$) {arp. out 1};
  \node[anchor=west] at ($(j4)+(18mm,0)$) {arp. out 2};
  \node[anchor=east] at ($(j5)+(-15mm,0)$) {gate};
  \node[anchor=west] at ($(j6)+(18mm,0)$) {trigger};
\end{tikzpicture}\par}

These channels use the global MIDI clock.  MIDI Timing Clock messages, or
pulses received on the left input jack, define the tempo at a rate of
24~PPQN.  MIDI Start messages reset to the start of the quarter note. 
Pulses received on the right input jack, and the tap tempo key on a typing
keyboard, reset to the start of the quarter note and also define the tempo
when there is no 24~PPQN clock.

The left digital output is the gate; it goes high (9V nominal) and the right
LED glows red, for the first 7/8 of each quarter note time.  The right
digital output produces a 960$\mu$s trigger at the start of each quarter
note time; this can also function as a 1~PPQN clock for other modules.  The
left LED glows green when there are any held notes, that is, when the
arpeggiator is active.  As for the CV outputs, they cycle through the held
MIDI notes in a way that depends on the channel.

Channel~5 arpeggiates up and down.  The left output cycles through the notes
upward (in order of increasing pitch) while the right output cycles
downward.

Channel~6 arpeggiates the held notes in the order they were entered, with
the right output one step ahead of the left.

Channel~7 arpeggiates in random order.  In more detail, the right output
selects one of the held notes uniformly at random on each beat.  That means
\emph{every} held note is available; the right output can repeat notes.  The
left output selects a note at random, but if there are at least two held
notes then it will avoid choosing the same note as the right output, and if
there are at least three held notes then it will also avoid its own previous
value; other than those exceptions, it selects uniformly.  So the left
output will not repeat notes, given sufficient held notes to choose from. 
The random numbers for Channel~7 come from a hashed entropy pool
continuously reseeded by I/O timings, and they should be random enough for
rock'n'roll even if not cryptographically certifiable.

All three arpeggiation modes start their new notes \emph{on the beat}.  If
you start playing notes into these channels when the tempo is slow, there
may be a noticeable delay before the output starts playing; but this design
feature is intended to allow playing more precisely \emph{with} the beat in
typical modular performance styles.  Note that the tempo clock needs to be
running in order for the arpeggiator to be useful.  To use these channels
you must provide the module with MIDI clock messages, a clock on the input
jacks, or tap tempo from a typing keyboard.

%%%%%%%%%%%%%%%%%%%%%%%%%%%%%%%%%%%%%%%%%%%%%%%%%%%%%%%%%%%%%%%%%%%%%%%%

\section{Channels 8 and 9:  dual mono}

These two channels work together to allow independent control of two GV/gate
synthesizer voices.  Unlike Channel~2, which automatically assigns incoming
notes to either channel, you can use this pair of channels to specify on
which side each note will play.  Channel~8 controls the left and Channel~9
the right.

{\centering
\begin{tikzpicture}[scale=0.8]
  {\setmainfont[Path={../../tsukurimashou/otf/},
BoldFont={TsukurimashouBokukkoExtraBoldPS}]{TsukurimashouBokukkoDemiboldPS}%
% start out defining coordinates
\coordinate (lbh) at (4.91mm,31.23mm) {};
\coordinate (j1) at (8.72mm,59.17mm) {};
\coordinate (j2) at (27.77mm,59.17mm) {};
\coordinate (j3) at (8.72mm,40.12mm) {};
\coordinate (j4) at (27.77mm,40.12mm) {};
\coordinate (j5) at (8.72mm,21.07mm) {};
\coordinate (j6) at (27.77mm,21.07mm) {};
\coordinate (d5) at (8.72mm,50.28mm) {};
\coordinate (d6) at (27.77mm,50.28mm) {};
%
\draw (0,11.07mm) -- (0,69.07mm);
\draw (40.30mm,11.07mm) -- (40.30mm,69.07mm);
%
\draw[very thick,black] (j1) -- (j3);
\draw[very thick,black] (j2) -- (j4);
\draw[very thick,black,fill=blue!30!black!20!white,rounded corners=3.0mm]
  (40.30mm,47.12mm) -- (3.72mm,47.12mm) --
  (3.72mm,14.07mm) -- (40.33mm,14.07mm);
\node[black!30!white] at ($(j1)!0.5!(j2)$) {\large IN};
\node[white] at ($(j3)!0.5!(j4)$) {\large\textbf{CV}};
\node[white] at ($(j5)!0.5!(j6)$) {\large\textbf{GT}};
%
% board-to-panel mounting holes, to clear M3 machine screw
\draw (lbh) circle[radius=1.60mm];
% six jacks with M6 threads, 6.3mm holes
\draw (j1) circle[radius=3.15mm];
\draw (j2) circle[radius=3.15mm];
\draw (j3) circle[radius=3.15mm];
\draw (j4) circle[radius=3.15mm];
\draw (j5) circle[radius=3.15mm];
\draw (j6) circle[radius=3.15mm];
% two LEDs, 5.20mm holes
\draw (d5) circle[radius=2.60mm];
\draw (d6) circle[radius=2.60mm];
%
% mock up screw heads, knobs
% machine screws with 6mm heads
\foreach \screwname in {lbh} {
  \path[draw=black,shading=ball,
    left color=black!10!white,right color=black!60!white]
    (\screwname) circle[radius=3mm];
  \draw ($(\screwname)+(50:3.0mm)$)--($(\screwname)+(220:3.0mm)$);
  \draw ($(\screwname)+(40:3.0mm)$)--($(\screwname)+(230:3.0mm)$);
}
% jacks with knurled nuts
\foreach \jackname in {j1,j2,j3,j4,j5,j6} {
  \path[draw=black,fill=black!20!white,
    decorate,decoration={snake,amplitude=0.6,segment length=2.5}]
    ($(\jackname)+(0,-0.15mm)$) circle[radius=4mm];
  \path[draw=black,fill=black!15!white] (\jackname) circle[radius=3mm];
  \path[draw=black,fill=black] (\jackname) circle[radius=2mm];
}
}

%
  \path[fill=green!60!white] (d5) circle[radius=2.50mm];
  \path[fill=green!60!white] (d6) circle[radius=2.50mm];
  \draw[very thick,{Stealth[scale=1.2]}-]
    ($(d5)+(-15mm,0)$) -- ($(d5)+(-4mm,0)$);
  \draw[very thick,{Stealth[scale=1.2]}-]
    ($(d6)+(18mm,0)$) -- ($(d6)+(4mm,0)$);
  \draw[very thick,{Stealth[scale=1.2]}-]
    ($(j3)+(-15mm,0)$) -- ($(j3)+(-5mm,0)$);
  \draw[very thick,{Stealth[scale=1.2]}-]
    ($(j4)+(18mm,0)$) -- ($(j4)+(5mm,0)$);
  \draw[very thick,{Stealth[scale=1.2]}-]
    ($(j5)+(-15mm,0)$) -- ($(j5)+(-5mm,0)$);
  \draw[very thick,{Stealth[scale=1.2]}-]
    ($(j6)+(18mm,0)$) -- ($(j6)+(5mm,0)$);
%
  \node[anchor=east] at ($(d5)+(-15mm,0)$) {Ch8 gate};
  \node[anchor=west] at ($(d6)+(18mm,0)$) {Ch9 gate};
  \node[anchor=east] at ($(j3)+(-15mm,0)$) {Ch8 pitch};
  \node[anchor=west] at ($(j4)+(18mm,0)$) {Ch9 pitch};
  \node[anchor=east] at ($(j5)+(-15mm,0)$) {Ch8 gate};
  \node[anchor=west] at ($(j6)+(18mm,0)$) {Ch9 gate};
\end{tikzpicture}\par}

Much like Channel~1, these channels do monophonic note stealing.  The
LED for each channel glows green when a note is playing.  The input jacks
are ignored.  Channels~8 and~9 have independent pitch bend.

%%%%%%%%%%%%%%%%%%%%%%%%%%%%%%%%%%%%%%%%%%%%%%%%%%%%%%%%%%%%%%%%%%%%%%%%

\section{Channels 10 and 11:  drum notes}

These channels map note numbers into pulses on the four output jacks. 
Channel~10 produces a trigger of approximately 1ms at the start of each
note, whereas Channel~11 produces a gate pulse, high as long as the note
lasts.

Note numbers translate to output jacks according to a somewhat complicated
mapping that is designed to work without needing special configuration on
many common MIDI controllers.  As a simple starting point, this
assignment of General MIDI drum notes to output jacks is one that works:
\begin{itemize}
\item MIDI note 36 (Bass Drum 1): lower right
\item MIDI note 38 (Acoustic Snare): upper left
\item MIDI note 40 (Electric Snare): upper right
\item MIDI note 42 (Closed Hi Hat): lower left
\end{itemize}

{\centering
\begin{tikzpicture}[scale=0.8]
  {\setmainfont[Path={../../tsukurimashou/otf/},
BoldFont={TsukurimashouBokukkoExtraBoldPS}]{TsukurimashouBokukkoDemiboldPS}%
% start out defining coordinates
\coordinate (lbh) at (4.91mm,31.23mm) {};
\coordinate (j1) at (8.72mm,59.17mm) {};
\coordinate (j2) at (27.77mm,59.17mm) {};
\coordinate (j3) at (8.72mm,40.12mm) {};
\coordinate (j4) at (27.77mm,40.12mm) {};
\coordinate (j5) at (8.72mm,21.07mm) {};
\coordinate (j6) at (27.77mm,21.07mm) {};
\coordinate (d5) at (8.72mm,50.28mm) {};
\coordinate (d6) at (27.77mm,50.28mm) {};
%
\draw (0,11.07mm) -- (0,69.07mm);
\draw (40.30mm,11.07mm) -- (40.30mm,69.07mm);
%
\draw[very thick,black] (j1) -- (j3);
\draw[very thick,black] (j2) -- (j4);
\draw[very thick,black,fill=blue!30!black!20!white,rounded corners=3.0mm]
  (40.30mm,47.12mm) -- (3.72mm,47.12mm) --
  (3.72mm,14.07mm) -- (40.33mm,14.07mm);
\node[black!30!white] at ($(j1)!0.5!(j2)$) {\large IN};
\node[white] at ($(j3)!0.5!(j4)$) {\large\textbf{CV}};
\node[white] at ($(j5)!0.5!(j6)$) {\large\textbf{GT}};
%
% board-to-panel mounting holes, to clear M3 machine screw
\draw (lbh) circle[radius=1.60mm];
% six jacks with M6 threads, 6.3mm holes
\draw (j1) circle[radius=3.15mm];
\draw (j2) circle[radius=3.15mm];
\draw (j3) circle[radius=3.15mm];
\draw (j4) circle[radius=3.15mm];
\draw (j5) circle[radius=3.15mm];
\draw (j6) circle[radius=3.15mm];
% two LEDs, 5.20mm holes
\draw (d5) circle[radius=2.60mm];
\draw (d6) circle[radius=2.60mm];
%
% mock up screw heads, knobs
% machine screws with 6mm heads
\foreach \screwname in {lbh} {
  \path[draw=black,shading=ball,
    left color=black!10!white,right color=black!60!white]
    (\screwname) circle[radius=3mm];
  \draw ($(\screwname)+(50:3.0mm)$)--($(\screwname)+(220:3.0mm)$);
  \draw ($(\screwname)+(40:3.0mm)$)--($(\screwname)+(230:3.0mm)$);
}
% jacks with knurled nuts
\foreach \jackname in {j1,j2,j3,j4,j5,j6} {
  \path[draw=black,fill=black!20!white,
    decorate,decoration={snake,amplitude=0.6,segment length=2.5}]
    ($(\jackname)+(0,-0.15mm)$) circle[radius=4mm];
  \path[draw=black,fill=black!15!white] (\jackname) circle[radius=3mm];
  \path[draw=black,fill=black] (\jackname) circle[radius=2mm];
}
}

%
  \path[fill=green!60!white] (d5) circle[radius=2.50mm];
  \path[fill=red!90!black] ($(d5)+(225:2.5mm)$)
    arc[start angle=225,end angle=45,radius=2.50mm] --cycle;
  \path[fill=green!60!white] (d6) circle[radius=2.50mm];
  \path[fill=red!90!black] ($(d6)+(225:2.5mm)$)
    arc[start angle=225,end angle=45,radius=2.50mm] --cycle;
  \draw[very thick,{Stealth[scale=1.2]}-]
    ($(d5)+(-15mm,0)$) -- ($(d5)+(-4mm,0)$);
  \draw[very thick,{Stealth[scale=1.2]}-]
    ($(d6)+(18mm,0)$) -- ($(d6)+(4mm,0)$);
  \draw[very thick,{Stealth[scale=1.2]}-]
    ($(j3)+(-15mm,0)$) -- ($(j3)+(-5mm,0)$);
  \draw[very thick,{Stealth[scale=1.2]}-]
    ($(j4)+(18mm,0)$) -- ($(j4)+(5mm,0)$);
  \draw[very thick,{Stealth[scale=1.2]}-]
    ($(j5)+(-15mm,0)$) -- ($(j5)+(-5mm,0)$);
  \draw[very thick,{Stealth[scale=1.2]}-]
    ($(j6)+(18mm,0)$) -- ($(j6)+(5mm,0)$);
%
  \node[anchor=east] at ($(d5)+(-15mm,0)$) {38/42};
  \node[anchor=west] at ($(d6)+(18mm,0)$) {40/36};
  \node[anchor=east] at ($(j3)+(-15mm,0)$) {38 AS};
  \node[anchor=west] at ($(j4)+(18mm,0)$) {40 ES};
  \node[anchor=east] at ($(j5)+(-15mm,0)$) {42 CHH};
  \node[anchor=west] at ($(j6)+(18mm,0)$) {36 BD1};
\end{tikzpicture}\par}

Signals on the input jacks are ignored.  Each LED glows if any notes on
the corresponding side are active, red if the upper note (analog CV
output) is active and green if the lower but not the upper note is
active.

The upper analog outputs in drum mode go to their maximum voltage when high,
which is 5.5V nominal (uncalibrated).  The lower digital outputs, as usual,
go to about 9V.  The trigger pulse width is nominally 960$\mu$s.

Now, some more detail on how note mapping works.  Every MIDI note maps to one
of the four output jacks, so you can make up a mapping by choosing any four
notes that happen to map to different jacks.  If you have a controller like
a keyboard with many notes on it, you can probably find a usable mapping
quickly just through trial and error, but the scheme is designed to
guarantee that:
\begin{itemize}
\item any four consecutive MIDI notes starting with a multiple of four, such
as $\{0,1,2,3\}$ or $\{60,61,62,63\}$, will map to distinct jacks; and
\item any four MIDI notes spaced two apart, such as $\{0,2,4,6\}$ or
$\{65,67,69,71\}$, will map to distinct jacks.
\end{itemize}

That means pad controllers which typically assign the pads to consecutive
notes will often have four conveniently-arranged pads which control the four
output jacks.  It also means that the notes F, G, A, B (which are two
semitones apart) in any octave on a piano-style keyboard, or any four
horizontally adjacent keys on a Wicki-Hayden isomorphic keyboard (as
discussed in the typing-keyboard chapter of this manual), will work.

In even more detail:  each note number $N$ maps to a jack number $j=(\lfloor
N/4 \rfloor+N) \bmod 4$.  In words, that formula says to start with the note
number $N$, divide it by four and throw away the remainder, then add the
original $N$, divide the result by four again, and this time throw away the
quotient and look at only the remainder, which we'll call $j$.  Then:
\begin{itemize}
\item if $j=0$, the note activates the lower left jack;
\item if $j=1$, it activates the lower right jack;
\item if $j=2$, the upper left jack; and
\item if $j=3$, the upper right.
\end{itemize}

That splits the range of MIDI note numbers into four sets with the desired
properties of being hit by many convenient controller assignments.

{\centering
\begin{tikzpicture}[scale=0.8]
  {\setmainfont[Path={../../tsukurimashou/otf/},
BoldFont={TsukurimashouBokukkoExtraBoldPS}]{TsukurimashouBokukkoDemiboldPS}%
% start out defining coordinates
\coordinate (lbh) at (4.91mm,31.23mm) {};
\coordinate (j1) at (8.72mm,59.17mm) {};
\coordinate (j2) at (27.77mm,59.17mm) {};
\coordinate (j3) at (8.72mm,40.12mm) {};
\coordinate (j4) at (27.77mm,40.12mm) {};
\coordinate (j5) at (8.72mm,21.07mm) {};
\coordinate (j6) at (27.77mm,21.07mm) {};
\coordinate (d5) at (8.72mm,50.28mm) {};
\coordinate (d6) at (27.77mm,50.28mm) {};
%
\draw (0,11.07mm) -- (0,69.07mm);
\draw (40.30mm,11.07mm) -- (40.30mm,69.07mm);
%
\draw[very thick,black] (j1) -- (j3);
\draw[very thick,black] (j2) -- (j4);
\draw[very thick,black,fill=blue!30!black!20!white,rounded corners=3.0mm]
  (40.30mm,47.12mm) -- (3.72mm,47.12mm) --
  (3.72mm,14.07mm) -- (40.33mm,14.07mm);
\node[black!30!white] at ($(j1)!0.5!(j2)$) {\large IN};
\node[white] at ($(j3)!0.5!(j4)$) {\large\textbf{CV}};
\node[white] at ($(j5)!0.5!(j6)$) {\large\textbf{GT}};
%
% board-to-panel mounting holes, to clear M3 machine screw
\draw (lbh) circle[radius=1.60mm];
% six jacks with M6 threads, 6.3mm holes
\draw (j1) circle[radius=3.15mm];
\draw (j2) circle[radius=3.15mm];
\draw (j3) circle[radius=3.15mm];
\draw (j4) circle[radius=3.15mm];
\draw (j5) circle[radius=3.15mm];
\draw (j6) circle[radius=3.15mm];
% two LEDs, 5.20mm holes
\draw (d5) circle[radius=2.60mm];
\draw (d6) circle[radius=2.60mm];
%
% mock up screw heads, knobs
% machine screws with 6mm heads
\foreach \screwname in {lbh} {
  \path[draw=black,shading=ball,
    left color=black!10!white,right color=black!60!white]
    (\screwname) circle[radius=3mm];
  \draw ($(\screwname)+(50:3.0mm)$)--($(\screwname)+(220:3.0mm)$);
  \draw ($(\screwname)+(40:3.0mm)$)--($(\screwname)+(230:3.0mm)$);
}
% jacks with knurled nuts
\foreach \jackname in {j1,j2,j3,j4,j5,j6} {
  \path[draw=black,fill=black!20!white,
    decorate,decoration={snake,amplitude=0.6,segment length=2.5}]
    ($(\jackname)+(0,-0.15mm)$) circle[radius=4mm];
  \path[draw=black,fill=black!15!white] (\jackname) circle[radius=3mm];
  \path[draw=black,fill=black] (\jackname) circle[radius=2mm];
}
}

%
  \path[fill=green!60!white] (d5) circle[radius=2.50mm];
  \path[fill=red!90!black] ($(d5)+(225:2.5mm)$)
    arc[start angle=225,end angle=45,radius=2.50mm] --cycle;
  \path[fill=green!60!white] (d6) circle[radius=2.50mm];
  \path[fill=red!90!black] ($(d6)+(225:2.5mm)$)
    arc[start angle=225,end angle=45,radius=2.50mm] --cycle;
  \draw[very thick,{Stealth[scale=1.2]}-]
    ($(d5)+(-15mm,0)$) -- ($(d5)+(-4mm,0)$);
  \draw[very thick,{Stealth[scale=1.2]}-]
    ($(d6)+(18mm,0)$) -- ($(d6)+(4mm,0)$);
  \draw[very thick,{Stealth[scale=1.2]}-]
    ($(j3)+(-15mm,0)$) -- ($(j3)+(-5mm,0)$);
  \draw[very thick,{Stealth[scale=1.2]}-]
    ($(j4)+(18mm,0)$) -- ($(j4)+(5mm,0)$);
  \draw[very thick,{Stealth[scale=1.2]}-]
    ($(j5)+(-15mm,0)$) -- ($(j5)+(-5mm,0)$);
  \draw[very thick,{Stealth[scale=1.2]}-]
    ($(j6)+(18mm,0)$) -- ($(j6)+(5mm,0)$);
%
  \node[anchor=east] at ($(d5)+(-15mm,0)$) {left};
  \node[anchor=west] at ($(d6)+(18mm,0)$) {right};
  \node[anchor=east] at ($(j3)+(-15mm,0)$)
    {\parbox{14mm}{2, 5, 8, 15, 18\ldots}};
  \node[anchor=west] at ($(j4)+(18mm,0)$)
    {\parbox{14mm}{3, 6, 9, 12, 19\ldots}};
  \node[anchor=east] at ($(j5)+(-15mm,0)$)
    {\parbox{14mm}{0, 7, 10, 13, 16\ldots}};
  \node[anchor=west] at ($(j6)+(18mm,0)$)
    {\parbox{14mm}{1, 4, 11, 14, 17\ldots}};
  \node[draw,fill=white] at ($(j5)+(4mm,4mm)$) {\small $j=0$};
  \node[draw,fill=white] at ($(j6)+(4mm,4mm)$) {\small $j=1$};
  \node[draw,fill=white] at ($(j3)+(4mm,4mm)$) {\small $j=2$};
  \node[draw,fill=white] at ($(j4)+(4mm,4mm)$) {\small $j=3$};
\end{tikzpicture}\par}

%%%%%%%%%%%%%%%%%%%%%%%%%%%%%%%%%%%%%%%%%%%%%%%%%%%%%%%%%%%%%%%%%%%%%%%%

\section{Channel 12:  mono with clock out}

This monophonic mode provides CV/gate to control a synthesizer voice, with
output of the MIDI clock as Eurorack trigger signals.  The gate output is
sent through an analog output jack, meaning that its high level will be
approximately 5.5V.  The clock outputs are 960$\mu$s triggers with a high
level of approximately 9V, at 24~PPQN and 1~PPQN.  The 1~PPQN pulse is
scheduled about 1ms before the first 24~PPQN pulse of the quarter note, so
that it can be used as a ``reset''; these pulses are meant to express the
same semantics as the MIDI Timing Clock and MIDI Start messages.

The source for timing can be the input jacks, MIDI timing messages, or
the tap tempo feature of the typing keyboard driver.

{\centering
\begin{tikzpicture}[scale=0.8]
  {\setmainfont[Path={../../tsukurimashou/otf/},
BoldFont={TsukurimashouBokukkoExtraBoldPS}]{TsukurimashouBokukkoDemiboldPS}%
% start out defining coordinates
\coordinate (lbh) at (4.91mm,31.23mm) {};
\coordinate (j1) at (8.72mm,59.17mm) {};
\coordinate (j2) at (27.77mm,59.17mm) {};
\coordinate (j3) at (8.72mm,40.12mm) {};
\coordinate (j4) at (27.77mm,40.12mm) {};
\coordinate (j5) at (8.72mm,21.07mm) {};
\coordinate (j6) at (27.77mm,21.07mm) {};
\coordinate (d5) at (8.72mm,50.28mm) {};
\coordinate (d6) at (27.77mm,50.28mm) {};
%
\draw (0,11.07mm) -- (0,69.07mm);
\draw (40.30mm,11.07mm) -- (40.30mm,69.07mm);
%
\draw[very thick,black] (j1) -- (j3);
\draw[very thick,black] (j2) -- (j4);
\draw[very thick,black,fill=blue!30!black!20!white,rounded corners=3.0mm]
  (40.30mm,47.12mm) -- (3.72mm,47.12mm) --
  (3.72mm,14.07mm) -- (40.33mm,14.07mm);
\node[black!30!white] at ($(j1)!0.5!(j2)$) {\large IN};
\node[white] at ($(j3)!0.5!(j4)$) {\large\textbf{CV}};
\node[white] at ($(j5)!0.5!(j6)$) {\large\textbf{GT}};
%
% board-to-panel mounting holes, to clear M3 machine screw
\draw (lbh) circle[radius=1.60mm];
% six jacks with M6 threads, 6.3mm holes
\draw (j1) circle[radius=3.15mm];
\draw (j2) circle[radius=3.15mm];
\draw (j3) circle[radius=3.15mm];
\draw (j4) circle[radius=3.15mm];
\draw (j5) circle[radius=3.15mm];
\draw (j6) circle[radius=3.15mm];
% two LEDs, 5.20mm holes
\draw (d5) circle[radius=2.60mm];
\draw (d6) circle[radius=2.60mm];
%
% mock up screw heads, knobs
% machine screws with 6mm heads
\foreach \screwname in {lbh} {
  \path[draw=black,shading=ball,
    left color=black!10!white,right color=black!60!white]
    (\screwname) circle[radius=3mm];
  \draw ($(\screwname)+(50:3.0mm)$)--($(\screwname)+(220:3.0mm)$);
  \draw ($(\screwname)+(40:3.0mm)$)--($(\screwname)+(230:3.0mm)$);
}
% jacks with knurled nuts
\foreach \jackname in {j1,j2,j3,j4,j5,j6} {
  \path[draw=black,fill=black!20!white,
    decorate,decoration={snake,amplitude=0.6,segment length=2.5}]
    ($(\jackname)+(0,-0.15mm)$) circle[radius=4mm];
  \path[draw=black,fill=black!15!white] (\jackname) circle[radius=3mm];
  \path[draw=black,fill=black] (\jackname) circle[radius=2mm];
}
}

%
  \path[fill=green!60!white] (d5) circle[radius=2.50mm];
  \path[fill=green!60!white] (d6) circle[radius=2.50mm];
  \path[fill=red!90!black] (d6) circle[radius=1.0mm];
  \draw[very thick,-{Stealth[scale=1.2]}]
    ($(j1)+(-15mm,0)$) -- ($(j1)+(-5mm,0)$);
  \draw[very thick,-{Stealth[scale=1.2]}]
    ($(j2)+(18mm,0)$) -- ($(j2)+(5mm,0)$);
  \draw[very thick,{Stealth[scale=1.2]}-]
    ($(d5)+(-15mm,0)$) -- ($(d5)+(-4mm,0)$);
  \draw[very thick,{Stealth[scale=1.2]}-]
    ($(d6)+(18mm,0)$) -- ($(d6)+(4mm,0)$);
  \draw[very thick,{Stealth[scale=1.2]}-]
    ($(j3)+(-15mm,0)$) -- ($(j3)+(-5mm,0)$);
  \draw[very thick,{Stealth[scale=1.2]}-]
    ($(j4)+(18mm,0)$) -- ($(j4)+(5mm,0)$);
  \draw[very thick,{Stealth[scale=1.2]}-]
    ($(j5)+(-15mm,0)$) -- ($(j5)+(-5mm,0)$);
  \draw[very thick,{Stealth[scale=1.2]}-]
    ($(j6)+(18mm,0)$) -- ($(j6)+(5mm,0)$);
%
  \node[anchor=east] at ($(j1)+(-15mm,0)$) {24 PPQN};
  \node[anchor=west] at ($(j2)+(18mm,0)$) {1 PPQN};
  \node[anchor=east] at ($(d5)+(-15mm,0)$) {gate};
  \node[anchor=west] at ($(d6)+(18mm,0)$) {gate/beat};
  \node[anchor=east] at ($(j3)+(-15mm,0)$) {pitch CV};
  \node[anchor=west] at ($(j4)+(18mm,0)$) {gate};
  \node[anchor=east] at ($(j5)+(-15mm,0)$) {24 PPQN};
  \node[anchor=west] at ($(j6)+(18mm,0)$) {1 PPQN};
\end{tikzpicture}\par}

This channel implements monophonic note stealing, the same as Channel~1. 
Both LEDs glow green when a note is playing, but the right LED also flashes
red on the beat, overriding the green.

%%%%%%%%%%%%%%%%%%%%%%%%%%%%%%%%%%%%%%%%%%%%%%%%%%%%%%%%%%%%%%%%%%%%%%%%

\section{Channels 13--16:  reserved}

The last four channels are not currently implemented.  They are available
for use by future versions of the official firmware, or possibly by
user-defined firmware.
