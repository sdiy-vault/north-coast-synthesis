% $Id: nonusb.tex 9713 2021-12-18 19:22:35Z mskala $

%
% Non-USB operation
% Copyright (C) 2022  Matthew Skala
%
% This program is free software: you can redistribute it and/or modify
% it under the terms of the GNU General Public License as published by
% the Free Software Foundation, version 3.
%
% This program is distributed in the hope that it will be useful,
% but WITHOUT ANY WARRANTY; without even the implied warranty of
% MERCHANTABILITY or FITNESS FOR A PARTICULAR PURPOSE.  See the
% GNU General Public License for more details.
%
% You should have received a copy of the GNU General Public License
% along with this program.  If not, see <http://www.gnu.org/licenses/>.
%
% Matthew Skala
% https://northcoastsynthesis.com/
% mskala@northcoastsynthesis.com
%

\chapter{Non-USB operation}

\begin{figure*}[b]
{\centering
\begin{tikzpicture}
  {\setmainfont[Path={../../tsukurimashou/otf/},
BoldFont={TsukurimashouBokukkoExtraBoldPS}]{TsukurimashouBokukkoDemiboldPS}%
% start out defining coordinates
\coordinate (lbh) at (4.91mm,31.23mm) {};
\coordinate (j1) at (8.72mm,59.17mm) {};
\coordinate (j2) at (27.77mm,59.17mm) {};
\coordinate (j3) at (8.72mm,40.12mm) {};
\coordinate (j4) at (27.77mm,40.12mm) {};
\coordinate (j5) at (8.72mm,21.07mm) {};
\coordinate (j6) at (27.77mm,21.07mm) {};
\coordinate (d5) at (8.72mm,50.28mm) {};
\coordinate (d6) at (27.77mm,50.28mm) {};
%
\draw (0,11.07mm) -- (0,69.07mm);
\draw (40.30mm,11.07mm) -- (40.30mm,69.07mm);
%
\draw[very thick,black] (j1) -- (j3);
\draw[very thick,black] (j2) -- (j4);
\draw[very thick,black,fill=blue!30!black!20!white,rounded corners=3.0mm]
  (40.30mm,47.12mm) -- (3.72mm,47.12mm) --
  (3.72mm,14.07mm) -- (40.33mm,14.07mm);
\node[black!30!white] at ($(j1)!0.5!(j2)$) {\large IN};
\node[white] at ($(j3)!0.5!(j4)$) {\large\textbf{CV}};
\node[white] at ($(j5)!0.5!(j6)$) {\large\textbf{GT}};
%
% board-to-panel mounting holes, to clear M3 machine screw
\draw (lbh) circle[radius=1.60mm];
% six jacks with M6 threads, 6.3mm holes
\draw (j1) circle[radius=3.15mm];
\draw (j2) circle[radius=3.15mm];
\draw (j3) circle[radius=3.15mm];
\draw (j4) circle[radius=3.15mm];
\draw (j5) circle[radius=3.15mm];
\draw (j6) circle[radius=3.15mm];
% two LEDs, 5.20mm holes
\draw (d5) circle[radius=2.60mm];
\draw (d6) circle[radius=2.60mm];
%
% mock up screw heads, knobs
% machine screws with 6mm heads
\foreach \screwname in {lbh} {
  \path[draw=black,shading=ball,
    left color=black!10!white,right color=black!60!white]
    (\screwname) circle[radius=3mm];
  \draw ($(\screwname)+(50:3.0mm)$)--($(\screwname)+(220:3.0mm)$);
  \draw ($(\screwname)+(40:3.0mm)$)--($(\screwname)+(230:3.0mm)$);
}
% jacks with knurled nuts
\foreach \jackname in {j1,j2,j3,j4,j5,j6} {
  \path[draw=black,fill=black!20!white,
    decorate,decoration={snake,amplitude=0.6,segment length=2.5}]
    ($(\jackname)+(0,-0.15mm)$) circle[radius=4mm];
  \path[draw=black,fill=black!15!white] (\jackname) circle[radius=3mm];
  \path[draw=black,fill=black] (\jackname) circle[radius=2mm];
}
}

%
  \path[fill=red!90!black] (d5) circle[radius=2.50mm];
  \path[fill=red!90!black] (d6) circle[radius=2.50mm];
  \draw[very thick,-{Stealth[scale=1.2]}]
    ($(j1)+(-15mm,0)$) -- ($(j1)+(-5mm,0)$);
  \draw[very thick,-{Stealth[scale=1.2]}]
    ($(j2)+(18mm,0)$) -- ($(j2)+(5mm,0)$);
  \draw[very thick,{Stealth[scale=1.2]}-]
    ($(d5)+(-15mm,0)$) -- ($(d5)+(-4mm,0)$);
%  \draw[very thick,{Stealth[scale=1.2]}-]
%    ($(d6)+(18mm,0)$) -- ($(d6)+(4mm,0)$);
  \draw[rounded corners=4mm,very thick,dashed]
    ($(d5)+(-4mm,-4mm)$) rectangle ($(d6)+(4mm,4mm)$);
  \draw[very thick,{Stealth[scale=1.2]}-]
    ($(j3)+(-15mm,0)$) -- ($(j3)+(-5mm,0)$);
  \draw[very thick,{Stealth[scale=1.2]}-]
    ($(j4)+(18mm,0)$) -- ($(j4)+(5mm,0)$);
  \draw[very thick,{Stealth[scale=1.2]}-]
    ($(j5)+(-15mm,0)$) -- ($(j5)+(-5mm,0)$);
  \draw[very thick,{Stealth[scale=1.2]}-]
    ($(j6)+(18mm,0)$) -- ($(j6)+(5mm,0)$);
%
  \node[anchor=east] at ($(j1)+(-15mm,0)$) {V/oct CV};
  \node[anchor=west] at ($(j2)+(18mm,0)$) {gate in};
  \node[anchor=east] at ($(d5)+(-15mm,0)$) {red on gate};
  \node[anchor=east] at ($(j3)+(-15mm,0)$) {quantized CV};
  \node[anchor=west] at ($(j4)+(18mm,0)$) {envelope};
  \node[anchor=east] at ($(j5)+(-15mm,0)$)
    {\parbox{19mm}{unquantized oscillator}};
  \node[anchor=west] at ($(j6)+(18mm,0)$)
    {\parbox{15mm}{quantized oscillator}};
\end{tikzpicture}\par}
\caption{Jack/LED functions for non-USB mode.}\label{fig:non-usb-jacks}
\end{figure*}

The Gracious Host is meant to work with a USB device like a MIDI controller,
but as a default, with no USB device connected, the standard firmware will
function as a baby synth voice.  See Figure~\ref{fig:non-usb-jacks} for the
panel jack assignments in this mode.

The left CV input is volt per octave pitch, with a nominal range from 0V for
MIDI note 36 (C two octaves below Middle C, 65.406Hz) to 5V for MIDI note
96 (C three octaves above Middle C, 2093.004Hz).  The right CV input is for
a gate signal (threshold approximately 1.62V).  Both LEDs will light, in red,
when the gate is high.

The left CV output is a semitone quantized version of the CV input.  The
right CV output is an ADSR envelope with fixed parameters.  And the two
digital outputs are square wave oscillators:  unquantized pitch on the left,
quantized pitch on the right.

\newpage

The unquantized oscillator runs only while
the gate is high, whereas the quantized oscillator runs continuously once
started.  The concept here is that you could use the module with an external
VCA and the quantized and envelope outputs to get sound during the decay tail, or just
use the unquantized output directly and have a very basic synth voice in the
Gracious Host alone.

Note that the digital outputs, as usual, produce only the fixed levels of 0V
for low and approximately 9V for high; so the square waves on these outputs
include a DC offset of approximately 4.5V.

If you connect a USB device, the module will leave this mode and attempt to
do something appropriate to the device instead.
