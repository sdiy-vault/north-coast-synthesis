% $Id: glossary.tex 9823 2022-02-09 01:07:30Z mskala $

%
% Glossary for MSK 014 programmer's manual
% Copyright (C) 2022  Matthew Skala
%
% This program is free software: you can redistribute it and/or modify
% it under the terms of the GNU General Public License as published by
% the Free Software Foundation, version 3.
%
% This program is distributed in the hope that it will be useful,
% but WITHOUT ANY WARRANTY; without even the implied warranty of
% MERCHANTABILITY or FITNESS FOR A PARTICULAR PURPOSE.  See the
% GNU General Public License for more details.
%
% You should have received a copy of the GNU General Public License
% along with this program.  If not, see <http://www.gnu.org/licenses/>.
%
% Matthew Skala
% https://northcoastsynthesis.com/
% mskala@northcoastsynthesis.com
%

\chapter{Glossary}

\begin{description}

\item[absolute section] to the disappointment of tsundere fans, this is a
reserved section in object code files used for storing symbols whose values
are just unrelocatable numbers and not offsets into memory sections that the
linker should relocate.  The\ .struct assembler directive is documented
(sketchily) as being for defining symbols in the absolute section that can
then be used as offsets into data structures; but it either was broken by
Microchip when they did their chop job on the GNU assembler, or it may have
never even worked in the original GNU assembler to begin with.

\item[ADC] Analog to Digital Converter, a device that converts voltage
measurements to digital numbers.  The Gracious Host uses a 10-bit ADC that
is built into the microcontroller.

\item[API] Application Programming Interface, the interface through which
application software can make use of a library or driver.

\item[BCD] Binary Coded Decimal:  decimal translated to binary by
translating each digit to four bits independently, rather than using the
integer value of the entire number at once.  Basically the same
thing you would get by pretending the decimal number is actually hexadecimal.

\item[big endian] handling the most significant bytes of a multi-byte number
first, or storing them in smaller-indexed addresses.  Some computers, and
almost all Internet-associated protocols, prefer big endian, hence the
alternate name \emph{network byte order}.  See also \emph{little endian}. 
The USB Mass Storage specification, in its most popular variation, requires
both big endian and little endian numbers in different parts of the same data
structures.

\item[bit bang] serial communication by means of software on the CPU
controlling the timing of individual bits on one or more GPIO pins, instead
of using a specialized peripheral that implements the serial protocol in
hardware.

\item[boot keyboard] the USB Human Interface Device Specification's name for
the ordinary kind of typing or QWERTY keyboard typically used with PCs, and
the protocol for talking to it; called \emph{boot} because of their idea
that this protocol would only be used during the boot sequence and then
would be replaced by something more complicated once the full operating
system finished loading.

\item[boot mouse] much like \emph{boot keyboard}, the basic low-feature and
low-complexity USB mouse protocol.

\item[CABA] Cluster And Block Address, a data structure specific to the
Gracious Host FAT filesystem driver, used for referring to blocks within
chains.

\item[CBD] Technically this is the abbreviation for cannabidiol, a drug
found in cannabis and purported to have health benefits while producing
little or none of the ``high'' produced by tetrahydrocannabinol (THC); but
in the context of USB Mass Storage, more likely a typo for CDB.

\item[CBW] Command Block Wrapper.  A 31-byte structure sent from host to
device to initiate a USB Mass Storage bulk-only SCSI command.  Contains an
unaligned 16-byte field for the SCSI CDB.

\item[CDB] Command Data Block.  The header of a SCSI command, likely
to be followed by a data transfer in one direction or the other.  Tends to
contain big-endian fields with 16-bit alignment.  SCSI commands are often
named with the length of the CDB, with more than one length and CDB layout
possible for what would otherwise be the same command, such as READ (10) or
READ (12).

\item[CSW] Command Status Wrapper.  A 13-byte structure sent from device to
host after completion of a USB Mass Storage bulk-only SCSI command to report
success or failure of the command and maintain synchronization.

\item[cluster] in a FAT filesystem, the unit of allocation.  A cluster may
be from 1 to 64 \emph{FAT blocks}; standard FAT filesystems have the
limitations that the cluster size must be a power of 2 and no more than 64K
bytes total (no more than 16K in many implementations), but the Gracious
Host can actually read some filesystems that break those rules.

\item[common data] a toolchain feature by which copies of a symbol can be
defined in more than one assembly-language file and will then be merged to
all appear at the same address; not perfectly supported by the
PIC24 toolchain, but we use it to implement a substitute for the even more
broken \emph{data overlay} feature.

\item[CN] Change Notification, a microcontroller feature that allows a GPIO
pin to serve as an interrupt request.

\item[CPU] Central Processing Unit.

\item[CRC] Cyclic Redundancy Check, a kind of checksum based on finite field
polynomial division; our microcontroller has a dedicated hardware module for
computing these efficiently.

\item[CRC32] One of:  the PIC24F peripheral for calculating CRCs; a specific
very popular 32-bit CRC algorithm used by the Gracious Host for checking the
integrity of firmware update images and in the PRNG code; or just any 32-bit
CRC.

\item[CRC5] a specific 5-bit CRC algorithm used in USB for low-level bit
error detection and affected by a silicon erratum in the PIC24F USB
hardware.

\item[CTMU] Charge Time Measurement Unit, a microcontroller feature for
implementing capacitive touch controls, not usable in the Gracious Host
hardware.

\item[CV] Control Voltage, as in a Eurorack synthesizer patch.

\item[DAC] Digital to Analog Converter; the Gracious Host contains a
separate 12-bit two-output DAC chip connected to the microcontroller
by the SPI port.

\item[data overlay] a toolchain feature by which multiple software modules
that will not run at the same time can reuse the same addresses for their
RAM data, saving overall memory consumption; desired by the Gracious Host
firmware but unusable in the PIC24F toolchain because of linker bugs.

\item[descriptor] in USB, a data structure that the device returns to the
host during a configuration phase after enumeration.  Devices usually have
many descriptors, containing information about the device's manufacturer and
model number, which USB standards it supports, its capabilities and
limitations (such as number of buttons or ports), and so on.

\item[dirent] Directory Entry.  A 32-byte record in a FAT filesystem's
directory (root directory or subdirectory).  A basic dirent stores the
8.3-format short filename, file size, starting cluster number, timestamp,
and so on.  Files with longer names have extra dirents each
storing a chunk of the long name.

\item[drive block] my name for a block of data in the native size of a USB
Mass Storage device as reported by the SCSI READ CAPACITY (10) transaction,
used to organize subsequent data transfers.  In practice this is
expected to always be 512 bytes, but the Gracious Host can also handle a few
larger sizes.

\item[DMA] Direct Memory Access, the act of a peripheral reading or writing
general RAM directly instead of going through the CPU and special registers
or memory mapping, often mediated by a \emph{DMA controller}.  The PIC24F
USB module uses DMA with a built-in dedicated controller.

\item[DS] Data Sheet, specifically the \emph{PIC24FJ64GB004 Family Data Sheet}
published by Microchip.

\item[ECPLL] External Clock Phase Locked Loop, the operating mode for the
microcontroller clock used in the Gracious Host.  An external crystal
oscillator module supplies a digital reference at an accurate frequency that
gets multiplied and then divided to generate the different clock frequencies
needed internally by the microcontroller.

\item[enumeration] something that is supposed to happen when a USB device is
attached to a USB host: the host assigns the device an ID number, so
that different devices on the same bus can be addressed separately.  The
Gracious Host tells every device to be number 1, because there can only
be one device attached at a time anyway.

\item[EP] ``endpoint''; a data structure used by the USB driver to represent
the local end of a ``pipe'' between software on the host and a ``function''
on the device.

\item[Fast RC] one of the built-in oscillators on the microcontroller,
capable of running the chip at full speed without needing any external
components, but not accurate enough for USB operation unless possibly it may
be \emph{trimmed} for the variations of individual chips by a fiddly and
poorly-documented procedure.

\item[FAT] File Allocation Table; see \emph{FAT filesystem} and \emph{FAT
per se}.

\item[FAT filesystem] a data structure for storing files and directories on
a disk or a disk-like medium, made popular by MS-DOS, subsequently used by
Windows, and popular for USB Mass Storage devices even when they are read
and written by non-Microsoft systems.  The Gracious Host includes a
low-featured FAT filesystem driver for reading firmware update images from
USB flash drives.  FAT filesystems are described as FAT12, FAT16, or FAT32
depending on the bit width of the entries in the \emph{FAT per se}.

\item[FAT per se] the part of a \emph{FAT filesystem} that is literally
named the ``file allocation table.'' It is an array of entries that are 12,
16, or 32 bits long, recording for each cluster in the filesystem whether
that cluster is in use, bad, or free, and if in use, which other cluster is
next in the chain.

\item[FAT block] my name for one of the blocks used to organize a FAT
filesystem, officially (but confusingly) called a \emph{logical sector} and
not necessarily matching the \emph{drive block} size.

\item[FIFO] First In First Out, describing a type of buffer commonly used
between the CPU and a peripheral, in either direction, so that they will
less often need to wait for each other.

\item[firmware] software that is built into hardware, effectively becoming
part of it.

\item[foreground] the code running on the microcontroller under ordinary
circumstances, when it is \emph{not} processing an interrupt.

\item[FRM] Family Reference Manual, specifically the \emph{PIC24F Family
Reference Manual} published (a chapter at a time, not as a single document)
by Microchip.

\item[GPIO] General Purpose Input/Output, the common digital interface pins
on many microcontrollers.  GPIO pins can usually be configured one at a time
as input or output, and often have some extra features like being
configurable for open-drain or tri-state output modes, or to generate
interrupts in input mode.

\item[GPL] the General Public License, a set of copyright licensing terms
applicable to the Gracious Host hardware and firmware, as well as to the GNU
toolchain.  It means you're allowed to distribute and make modifications to
the things in question, but you're not allowed to prevent others from doing
the same.

\item[hardware breakpoint] in ICD, breakpoints mediated by undocumented
hardware features.  They are efficient and do not wear out the flash, but
you can only have up to four code and four data hardware breakpoints at a
time, and if you use more than three, you lose some single-stepping
capability and the Microchip tools will encourage you to switch to software
breakpoints.

\item[Harvard architecture, modified] a computer architecture that puts code
and data in separate address spaces which may have significantly different
rules, common in microcontrollers too small to run operating systems; used
in PIC24.

\item[HID] Human Interface Device, the USB term for a general category of
devices used by humans to directly communicate with computers.  Includes
mice, typing keyboards, and some things like joysticks, arcade buttons, and
VR controllers, but notably does \emph{not} usually include music keyboards
(which tend to be USB-MIDI instead) nor sex toys (which tend to have
proprietary vendor-only protocols).  The USB HID standard includes
simplified protocols for typing keyboards and mice, which are called the
``boot'' protocols, and a much more complicated generic protocol intended to
work with all types of HIDs including all of their unique features.

\item[ICD] In-Circuit Debugging, with a special hardware device plugged into
reserved pins on the microcontroller to allow stepping through the code,
setting breakpoints, and so on.

\item[ICSP] In-Circuit Serial Programming, loading the microcontroller with
its firmware through basically the same interface as ICD.

\item[I$^2$C] Inter-Integrated Circuit (bus), a serial bus similar in nature
and typical application to SPI, and supported by the microcontroller, but
not actually used in the Gracious Host.

\item[IRP] I/O Request Packet, the USB standards' term for a data structure
passed into the driver when requesting data to be transferred over the bus
in either direction.

\item[ISR] Interrupt Service Routine, the subroutine that handles an
interrupt.  In PIC24 these need to return with the special \insn{retfie}
instruction instead of an ordinary \insn{return}.

\item[keep-alive] a pulse sent from the host to the device at 1\,ms
intervals on a low-speed USB connection.  The device disconnects if it
misses three consecutive keep-alives.

\item[LDO] Low Drop Out, describing a voltage regulator that can operate
with significantly less than the minimum 3V difference between input and
output that is required by traditional 78xx-style regulators.

\item[level-triggered] Microchip's description of the USB attach and detach
interrupts, which will keep reoccurring until fully disabled as long as the
relevant state persists, even if the individual interrupts are acknowledged
in the way required by other PIC24F interrupts.

\item[linker script]  A file that gives the linker instructions on how to
process fragments of code and data into the memory image of the complete
program.  The Gracious Host uses a customized linker script to put knowledge
about drivers into the higher-level code that uses it; but even the default
PIC24 linker script does a lot of complicated processing to support features
like automatic creation of interrupt vector tables, and initialization of
high-level language variables.

\item[little endian] handling the least significant bytes of a multi-byte
number first, or storing them in smaller-indexed addresses.  PIC24
microcontrollers have a preference for little endian organization.  See also
\emph{big endian}.

\item[LSB] Least Significant Bit; the bit with least numerical value,
normally written on the right.

\item[microcontroller] in this manual, specifically the Microchip
PIC24FJ64GB002-I/SP microcontroller.

\item[Microchip Corporation] vendors of PIC24 chips, some other chips used
in the Gracious Host, and hardware and software development tools. 
Distributors of a version of the GNU multi-platform toolchain modified to
work with PIC24, which means they have certain obligations under the GPL. 
See also \emph{Sirius Cybernetics Corporation}.

\item[MIDI] Musical Instruments Digital Interface.

\item[MSB] Most Significant Bit; the one with greatest numerical value,
normally written on the left.

\item[mutatis mutandis] Medieval Latin for ``with the necessary changes.''

\item[OTG] USB On The Go, a specification explaining how a USB host or
device can be confused about which one of those it is.

\item[page] in the context of the PIC24F flash program memory, an aligned
block of 512 words or 1536 bytes; this is the unit of flash erase
operations.

\item[PIC24] Microchip Corporation's 16/24-bit microcontroller architecture.

\item[PIC24F] a specific family of microcontrollers, subset of the broader
PIC24 architecture.

\item[PID] Packet Identifier.  A four-bit code attached to USB packets which
specifies their role in the protocol.

\item[PMP] Parallel Master Port, a feature of some larger PIC24F
microcontrollers that theoretically exists in the silicon of ours too, but
cannot actually be used because of packaging limitations, let alone the
conflicting Gracious Host board design.

\item[polyfuse] a component in the Gracious Host hardware, technically a
special temperature-sensitive resistor, that functions like a fuse to
temporarily shut off power to the connected USB device if the device tries
to draw a dangerously large amount of power.

\item[PPQN] Pulse Per Quarter Note.  Describes the ratio between synthesizer
tempo clock signals and musical notes.  MIDI uses a 24~PPQN clock, meaning
that there are 24 clock pulses for each quarter note.  At a tempo of 120
BPM, the 24~PPQN clock has $120\times 24 = 2880$ pulses per minute, or 48
pulses per second.

\item[PPS] Peripheral Pin Select, a microcontroller feature allowing the
mapping between on-chip digital peripherals and pins of the 28-pin SPDIP
package to be changed under software control.

\item[PRNG] Pseudo-Random Number Generator.  In the Gracious Host this is a
firmware feature implemented in utils.s and used by the MIDI backend for
random arpeggiation and by the USB mass storage driver to generate
transaction-recognition tokens.  It makes use of the CRC32 module.

\item[PSV] Program Space Visibility, a feature of the PIC24 architecture by
which part of program memory can be made to appear in a read-only window of
the data memory space.

\item[row] in the context of the PIC24F flash program memory, an aligned
block of 64 instruction words, equal to 192 bytes; there are 8 rows to
each page, and some flash write operations write a whole row at a time. 
Each set of Gracious Host calibration data uses, though it does not entirely
fill, one row.

\item[SCSI] Small Computer System Interface, originally a parallel bus used
for connecting disk drives and other peripherals to computers; basically
obsolete in its original form, but some more recent standards, including the
common variant of USB Mass Storage supported by the Gracious Host, work by
sending SCSI commands on top of some other protocol.  SCSI is \emph{big
endian}.

\item[SFR] Special Function Register(s); the hardware registers used for
communication with the microcontroller's on-chip peripherals and for CPU
control, mapped between addresses 0x0000 and 0x07FF in data memory.

\item[skinny DIP] a Dual Inline Package (DIP) for a through-hole IC with
0.300$''$ row spacing despite having 24 pins or more.  Traditional DIPs use
0.300$''$ spacing only for packages with fewer than 24 pins and 0.600$''$ at
higher pin counts.  The microcontroller in the Gracious Host comes in a
28-pin skinny DIP.

\item[soft timer] a timer implemented by having the CPU periodically update
a variable in RAM, instead of by using a hardware counter that runs
independently of the CPU.

\item[software breakpoint] in ICD, breakpoints implemented by rewriting the
program memory instead of using the hardware debugging features.  You can
use an unlimited number of software breakpoints, but they are slower, and
they cause significant wear on the flash memory.

\item[SOF] Start Of Frame, a packet the host sends to the device at
1\,ms intervals on a full-speed USB connection, much like the low-speed
\emph{keep-alive}.  The PIC24F hardware can
be set to generate an interrupt linked to the SOF.

\item[SPDIP] Skinny Plastic Dual Inline Package; see \emph{skinny DIP}.

\item[SPI] Serial Peripheral Interface, a serial bus used in
the Gracious Host to control the SRAM and DAC chips.

\item[SRAM] either Serial Random Access Memory or Static Random Access
Memory, which are not the same thing but the 23LC1024 SRAM chip in the
Gracious Host happens to be both.  This chip provides 128K bytes of memory
accessible to the microcontroller via SPI transactions, and is used as
a buffer for the incoming firmware image when updating firmware from USB
mass storage.

\item[star section] a section (in the assembly-language sense) with
its name declared as just an asterisk.  Then the assembler will automatically
give it a unique name, separating it from other sections and allowing
the linker some flexibility in locating it.

\item[tail call]  When a subroutine ends by branching to another subroutine. 
The return from the second subroutine has the effect of returning to the
caller of the first.  If the last instruction of the first subroutine before
returning would have been a regular call to the second subroutine, then
doing a tail call instead saves some time and space.

\item[TDL] Targeted Device List, the Gracious Host-specific term for the
part of the TPL executable data structure that recognizes entire devices
based on the information in their device descriptors.

\item[TIL] Targeted Interface List, the Gracious Host-specific term for the
part of the TPL executable data structure that recognizes interface
descriptors within a device, if no TDL entry has claimed the device first.

\item[TLA] usually read as Three-Letter Acronym, though many including TLA
itself are more properly called Three-Letter \emph{Abbreviations} because
they are not pronounced as words.

\item[toolchain] the sequence of software tools that turn source code into a
loadable binary image.  For PIC24 assembly language programs like the
Gracious Host firmware, the toolchain as such is basically just the
assembler and linker, although side utilities for dealing with object and
archive files are often counted as part of the broader toolchain entity.

\item[TPL] Targeted Peripherals List, the list of USB devices with which a
USB host is intended to work -- both a listing in the documentation and a
data structure likely to exist in its driver software.

\item[trap] basically an interrupt that happens when something really bad
has occurred, like an unaligned memory access.  They cannot be ignored or
masked, and normally lead to a CPU reset.

\item[UART] Universal Asynchronous Receiver Transmitter, a serial interface
traditionally used for connecting to things like modems and terminals.

\item[UBM] the \emph{MSK~014 Gracious Host User/Build Manual}, companion to
this volume.

\item[USB] Universal Serial Bus.

\item[VCO] Voltage Controlled Oscillator.  One of the basic modules in an
analog synthesizer.  Eurorack VCOs in particular usually have their
frequencies controlled by a voltage that shifts the frequency \emph{one volt
per octave} (V/oct); VCOs in non-Eurorack systems may have different control
voltage standards.  One of the main applications for the Gracious Host is in
controlling a VCO, and a V/oct VCO is required for the calibration process.

\item[von Neumann architecture] a computer architecture in which code and
data are in the same address space, accessed in substantially the same way,
typical of general-purpose computers and usually taken for granted by
operating systems and programming language tools.

\item[WDT] Watch Dog Timer, which resets the CPU if the firmware does not
clear the timer occasionally; intended to break out of situations where the
firmware has gone into an uncontrolled infinite loop.  There is also a
\emph{deep-sleep WDT}, not used on the Gracious Host, which has a similar
function in longer-lasting power-saving modes.

\item[ZIF socket] Zero Insertion Force socket, a type of IC socket designed
to withstand many cycles of inserting and removing ICs without damage to
either party, such as in IC testing equipment or for the CPU chip on
a desktop computer motherboard.

\end{description}
