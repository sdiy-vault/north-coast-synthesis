% $Id: mouse.tex 9718 2021-12-19 19:26:46Z mskala $

%
% Mouse interface
% Copyright (C) 2022  Matthew Skala
%
% This program is free software: you can redistribute it and/or modify
% it under the terms of the GNU General Public License as published by
% the Free Software Foundation, version 3.
%
% This program is distributed in the hope that it will be useful,
% but WITHOUT ANY WARRANTY; without even the implied warranty of
% MERCHANTABILITY or FITNESS FOR A PARTICULAR PURPOSE.  See the
% GNU General Public License for more details.
%
% You should have received a copy of the GNU General Public License
% along with this program.  If not, see <http://www.gnu.org/licenses/>.
%
% Matthew Skala
% https://northcoastsynthesis.com/
% mskala@northcoastsynthesis.com
%

\chapter{Mouse interface}

An ordinary USB mouse plugged into the Gracious Host makes a simple CV/gate
controller.  The connections and basic functions for this mode are as shown in
Figure~\ref{fig:mouse-conn}.

\begin{figure*}
{\centering\begin{tikzpicture}
  \begin{scope}
    \setmainfont[Path={../../tsukurimashou/otf/},
      BoldFont={TsukurimashouBokukkoExtraBoldPS}]{TsukurimashouBokukkoDemiboldPS}%
    % start out defining coordinates
    \coordinate (o) at (0,0);
    \coordinate (llrh) at ($(o)+(7.50mm,3.00mm)$) {};
    \coordinate (ulrh) at ($(o)+(7.50mm,125.50mm)$) {};
    \coordinate (lrrh) at ($(o)+(32.90mm,3.00mm)$) {};
    \coordinate (urrh) at ($(o)+(32.90mm,125.50mm)$) {};
    \coordinate (lbh) at ($(o)+(4.91mm,34.23mm)$) {};
    \coordinate (ubh) at ($(o)+(35.39mm,112.97mm)$) {};
    \coordinate (j1) at ($(o)+(8.72mm,62.17mm)$) {};
    \coordinate (j2) at ($(o)+(27.77mm,62.17mm)$) {};
    \coordinate (j3) at ($(o)+(8.72mm,43.12mm)$) {};
    \coordinate (j4) at ($(o)+(27.77mm,43.12mm)$) {};
    \coordinate (j5) at ($(o)+(8.72mm,24.07mm)$) {};
    \coordinate (j6) at ($(o)+(27.77mm,24.07mm)$) {};
    \coordinate (d5) at ($(o)+(8.72mm,53.28mm)$) {};
    \coordinate (d6) at ($(o)+(27.77mm,53.28mm)$) {};
    \coordinate (j11) at ($(o)+(8.80mm,102.81mm)$) {};
%
    \coordinate (cpn) at ($(o)+(20.15mm,128.50mm)$) {};
    \coordinate (cps) at ($(o)+(20.15mm,0mm)$) {};
    \coordinate (cpe) at ($(o)+(40.30mm,64.25mm)$) {};
    \coordinate (cpw) at ($(o)+(0mm,64.25mm)$) {};
%
    % background
    \draw[fill=white] (o) rectangle (40.30mm,128.50mm);
    \clip (o) rectangle (40.30mm,128.50mm);
  %
    \draw[very thick,black,fill=blue!15!black!30!white,rounded corners=3.0mm]
      ($(j11)+(-15mm,-12mm)$) rectangle ($(j11)+(11mm,12mm)$);
    \draw[very thick,black] (j1) -- (j3);
    \draw[very thick,black] (j2) -- (j4);
    \draw[very thick,black,fill=blue!50!black!30!white,rounded corners=3.0mm]
      ($(j5)+(-5mm,-7mm)$) rectangle ($(j4-|cpe)+(5mm,7mm)$);
    \node at ($(j1)!0.5!(j2)$) {\large IN};
    \node[white] at ($(j3)!0.5!(j4)$) {\large\textbf{CV}};
    \node[white] at ($(j5)!0.5!(j6)$) {\large\textbf{GT}};
  %
    \coordinate (sref) at ($(j1)!0.53!(j11)$) {};
    \coordinate (nref) at ($(sref)+(9.525mm,-20.32mm)$) {};
    \draw[very thick] ($(sref-|cpw)$)
      -- ($(sref-|cpe)$);
    \draw[very thick] ($(sref-|cpw)+(0,-2.5mm)$)
      -- ($(sref-|cpe)+(0,-2.5mm)$);
    \draw[very thick] ($(sref-|cpw)+(0,-5mm)$)
      -- ($(sref-|cpe)+(0,-5mm)$);
    \draw[very thick] ($(sref-|cpw)+(0,-7.5mm)$)
      -- ($(sref-|cpe)+(0,-7.5mm)$);
    \draw[very thick] ($(sref-|cpw)+(0,-10mm)$)
      -- ($(sref-|cpe)+(0,-10mm)$);
    \node at ($(nref)+(-11.0mm,15.62mm)$)
      {\lilyGlyph[scale=2.9]{clefs.G}};
    \node at ($(nref)+(-3.5mm,15.62mm)$)
      {\lilyGlyph[scale=2.9]{rests.2}};
    \foreach \x/\y in {3.00/0.05,10.00/2.55,17.00/5.05} {
      \node at ($(nref)+(\x mm,14mm+\y mm)$)
        {\lilyGlyph[scale=2.9]{noteheads.s2}};
      \draw[very thick]
        ($(nref)+(\x mm+1.4mm,14.1mm+\y mm)$) -- ++(0,8.5mm);
    }
  %
    \node at ($(cpn)+(0.0mm,-4.0mm)$) {\large MSK 014};
    \node at ($(cpn)+(0.0mm,-9.5mm)$) {\large GRACIOUS HOST};
    \node at ($(cps)+(0.0mm,8.5mm)$)
      {\parbox{0.9in}{\linespread{0.75}\selectfont\center NORTH COAST}};
%
    % panel-to-rails mounting holes, 3.2mm holes to clear M3 machine screw
    \draw (llrh) circle[radius=1.60mm];
    \draw (ulrh) circle[radius=1.60mm];
    \draw (lrrh) circle[radius=1.60mm];
    \draw (urrh) circle[radius=1.60mm];
    % board-to-panel mounting holes, to clear M3 machine screw
    \draw (lbh) circle[radius=1.60mm];
    \draw (ubh) circle[radius=1.60mm];
    % six jacks with M6 threads, 6.3mm holes
    \draw (j1) circle[radius=3.15mm];
    \draw (j2) circle[radius=3.15mm];
    \draw (j3) circle[radius=3.15mm];
    \draw (j4) circle[radius=3.15mm];
    \draw (j5) circle[radius=3.15mm];
    \draw (j6) circle[radius=3.15mm];
    % two LEDs, 5.20mm holes
    \draw (d5) circle[radius=2.60mm];
    \draw (d6) circle[radius=2.60mm];
    % rectangular hole for USB A connector
    \draw[rounded corners=1mm]
      ($(j11)+(-3.76mm,-7.45mm)$) rectangle ($(j11)+(3.76mm,7.45mm)$);
%
    % machine screws with 6mm heads
    \foreach \screwname in {llrh,ulrh,lrrh,urrh,lbh,ubh} {
      \path[draw=black,shading=ball,
        left color=black!10!white,right color=black!60!white]
        (\screwname) circle[radius=3mm];
      \draw ($(\screwname)+(50:3.0mm)$)--($(\screwname)+(220:3.0mm)$);
      \draw ($(\screwname)+(40:3.0mm)$)--($(\screwname)+(230:3.0mm)$);
    }
    % jacks with knurled nuts
    \foreach \jackname in {j1,j2,j3,j4,j5,j6} {
      \path[draw=black,fill=black!20!white,
        decorate,decoration={snake,amplitude=0.6,segment length=2.5}]
        ($(\jackname)+(0,-0.15mm)$) circle[radius=4mm];
      \path[draw=black,fill=black!15!white] (\jackname) circle[radius=3mm];
      \path[draw=black,fill=black] (\jackname) circle[radius=2mm];
    }
    % LEDs with red/green gradient
    \foreach \ledname in {d5,d6} {
      \path[shading=ball,left color=green,right color=red]
        (\ledname) circle[radius=2.50mm];
    }
    % mock up USB connector
    \draw[rounded corners=1mm,fill=black!10!white]
      ($(j11)+(-3.76mm,-7.45mm)$) rectangle ($(j11)+(3.76mm,7.45mm)$);
    \fill[rounded corners=0.64mm,fill=black!70!white]
      ($(j11)+(-2.25mm,-6.00mm)$) rectangle ($(j11)+(2.25mm,6.00mm)$);
    \fill[rounded corners=0.64mm,fill=blue!35!black!30!white]
      ($(j11)+(0,-5.50mm)$) rectangle ($(j11)+(1.75mm,5.50mm)$);
  \end{scope}
  \draw (o) rectangle (40.30mm,128.50mm);
%
  \coordinate (mtab) at ($(d5)+(-20mm,27mm)$) {};
  \draw[very thick,{Stealth[scale=1.2]}-,rounded corners=4mm]
    (mtab) -- ($(d5|-mtab)+(-11mm,0mm)$) --
    ($(d5)+(-11mm,0)$) -- ($(d5)+(-4mm,0)$);
  \draw[rounded corners=4mm,very thick,dashed]
    ($(d5)+(-4mm,-4mm)$) rectangle ($(d6)+(4mm,4mm)$);
  \draw[very thick,{Stealth[scale=1.2]}-]
    ($(j3)+(-15mm,0)$) -- ($(j3)+(-5mm,0)$);
  \draw[very thick,{Stealth[scale=1.2]}-]
    ($(j4)+(18mm,0)$) -- ($(j4)+(5mm,0)$);
  \draw[very thick,{Stealth[scale=1.2]}-]
    ($(j5)+(-15mm,0)$) -- ($(j5)+(-5mm,0)$);
  \draw[very thick,{Stealth[scale=1.2]}-]
    ($(j6)+(18mm,0)$) -- ($(j6)+(5mm,0)$);
%
  \node[anchor=east] at (mtab) {quantization mode};
  \node[anchor=east] at ($(j3)+(-15mm,0)$) {X coord};
  \node[anchor=west] at ($(j4)+(18mm,0)$) {Y coord};
  \node[anchor=east] at ($(j5)+(-15mm,0)$)
    {\parbox{12mm}{left button}};
  \node[anchor=west] at ($(j6)+(18mm,0)$)
    {\parbox{12mm}{right button}};
%
  \path[draw,fill=green!60!white] ($(mtab)+(-23mm,-6mm)$)
    circle[radius=1.50mm];
  \path[draw,fill=green!60!white] ($(mtab)+(-19mm,-6mm)$)
    circle[radius=1.50mm];
  \node[anchor=west] at ($(mtab)+(-16mm,-6mm)$)
    {unquantized};
%
  \path[draw,fill=red!90!black] ($(mtab)+(-23mm,-11mm)$)
    circle[radius=1.50mm];
  \path[draw,fill=red!90!black] ($(mtab)+(-19mm,-11mm)$)
    circle[radius=1.50mm];
  \node[anchor=west] at ($(mtab)+(-16mm,-11mm)$)
    {smart};
%
  \path[draw,fill=red!90!black] ($(mtab)+(-23mm,-16mm)$)
    circle[radius=1.50mm];
  \path[draw,fill=green!60!white] ($(mtab)+(-19mm,-16mm)$)
    circle[radius=1.50mm];
  \node[anchor=west] at ($(mtab)+(-16mm,-16mm)$)
    {diatonic};
%
  \path[draw,fill=green!60!white] ($(mtab)+(-23mm,-21mm)$)
    circle[radius=1.50mm];
  \path[draw,fill=red!90!black] ($(mtab)+(-19mm,-21mm)$)
    circle[radius=1.50mm];
  \node[anchor=west] at ($(mtab)+(-16mm,-21mm)$)
    {semitone};
%
  \coordinate (mouse) at ($(o)+(90mm,110mm)$) {};
  \coordinate (arctop) at ($(o)+(50mm,170mm)$) {};
  \fill (j11) circle[radius=1.50mm];
  \draw[very thick] (j11)
    ..controls ($(arctop)+(-10mm,0)$) and ($(arctop)+(10mm,0)$)..
    (mouse);
  \path[thick,draw,fill=white]
    ($(mouse)+(-17mm,-9mm)$)
      arc[start angle=180,end angle=90,radius=9mm] --
    ($(mouse)+(-8mm,0mm)$) -- ($(mouse)+(8mm,0mm)$)
      arc[start angle=90,end angle=0,radius=9mm] --
    ($(mouse)+(17mm,-30mm)$)
      arc[start angle=0,end angle=-180,radius=17mm]
    --cycle;
  \draw[thick] ($(mouse)+(-17mm,-25mm)$) -- ($(mouse)+(17mm,-25mm)$);
  \draw[thick] (mouse) -- ($(mouse)+(0mm,-25mm)$);
  \draw[thick,fill=white,rounded corners=2mm]
    ($(mouse)+(-3mm,-20mm)$) rectangle ($(mouse)+(3mm,-5mm)$);
%
  \draw[very thick,{Stealth[scale=1.2]}-]
    ($(mouse)+(-10mm,-16mm)$) -- ($(mouse)+(-27mm,8mm)$);
  \draw[very thick,{Stealth[scale=1.2]}-]
    ($(mouse)+(10mm,-16mm)$) -- ($(mouse)+(27mm,4mm)$);
  \draw[very thick,{Stealth[scale=1.2]}-]
    ($(mouse)+(0mm,-10mm)$) -- ($(mouse)+(12mm,16mm)$);
%
  \node[anchor=south] at ($(mouse)+(-25mm,8mm)$) {\parbox{12mm}{left gate}};
  \node[anchor=south] at ($(mouse)+(29mm,4mm)$) {\parbox{12mm}{right gate}};
  \node[anchor=south] at ($(mouse)+(15mm,16mm)$) {\parbox{12mm}{change mode}};
\end{tikzpicture}\par}
\caption{Mouse interface functions.}\label{fig:mouse-conn}
\end{figure*}

The standardization of USB mice is much like that of typing keyboards:  the
relevant standard includes a very complicated protocol and a simple one, and
most implementors only use the simple one.  In technical terms, the Gracious
Host supports the ``boot mouse'' protocol, for USB devices that expose an
interface descriptor of class 3, subclass 1, protocol 2.  Nearly all
commonly-available USB mice can operate under this protocol.

The basic function with a mouse attached is that the left and right mouse
buttons send gates (to the Gracious Host's digital outputs) and the X and Y
coordinates of mouse motion control the analog CV outputs.  The middle
button (or wheel, when clicked) cycles between the quantization modes below;
and the colours of the LEDs (which light up on button presses) indicate the
current mode.

\begin{itemize}
  \item Unquantized (both LEDs green; default on startup) -- CV outputs
    directly reflect the current X and Y coordinates.
  \item Smart quantize (both LEDs red) -- CV outputs are quantized to a
    diatonic scale that attempts to automatically follow the notes you play. 
    Play the fourth of the (major) scale three times without playing the
    seventh and it will shift to the subdominant key; play the seventh three
    times without playing the fourth, and it will shift to the dominant.  In
    practice, this allows for free improvisation with the somewhat clumsy
    mouse control, while keeping everything more or less sounding like it is
    in tune.
  \item Quantize to fixed scale (left LED red, right green) -- CV outputs
    are quantized to a fixed diatonic scale (C major or A minor, if 0V
    output is considered to be C).
  \item Quantize to semitones (left LED green, right red) -- CV outputs are
    quantized to V/oct semitones, that is, round multiples of $1/12$ of a
    volt.
\end{itemize}

In the quantization modes, the CVs are quantized only while the button is
held and the gate is high; otherwise they are unquantized.  That way, it is
convenient to build a patch where one voltage is quantized pitch and the
other is something like filter cutoff not meant to be quantized; to control
the patch as intended, just don't press the button on the unquantized side.

The input jacks are not used in mouse mode, and (because the USB boot mouse
protocol does not specify this) the mouse wheel's scrolling
function, other than clicking it, has no consistent interpretation.
