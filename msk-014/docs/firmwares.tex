% $Id: firmwares.tex 9772 2022-01-19 04:17:12Z mskala $

%
% Gracious Host firmware framework
% Copyright (C) 2022  Matthew Skala
%
% This program is free software: you can redistribute it and/or modify
% it under the terms of the GNU General Public License as published by
% the Free Software Foundation, version 3.
%
% This program is distributed in the hope that it will be useful,
% but WITHOUT ANY WARRANTY; without even the implied warranty of
% MERCHANTABILITY or FITNESS FOR A PARTICULAR PURPOSE.  See the
% GNU General Public License for more details.
%
% You should have received a copy of the GNU General Public License
% along with this program.  If not, see <http://www.gnu.org/licenses/>.
%
% Matthew Skala
% https://northcoastsynthesis.com/
% mskala@northcoastsynthesis.com
%

\chapter{Firmware framework (firmware.s)}

The firmware.s file contains what might be called the ``main program'' of
the firmware:  the code that runs at power-on reset, sets up the basic
configuration of the hardware, and then dispatches to
other modules.  It also contains some global infrastructure that simply
needed to be \emph{somewhere}, such as the declaration of the common data
area.

In this chapter I also describe the global include file, global.inc.

\section{Microcontroller configuration}

The configuration registers (``fuses'') and their values are declared at the
start of firmware.s, with some conditional assembly to take into account
configuration settings from config.inc.

At the bottom of this section there is also a quick declaration of the
common data area for use by the \insn{in\_common} macro.

\section{Last page}

The last 1.5K page of program memory cannot be reprogrammed (or,
technically, it cannot be safely \emph{erased}) by the microcontroller under
its own software control; only by in-circuit programming.  So this area can
only be used for data that is never expected to change.  In the Gracious
Host that is an eight-byte hardware identifier (ASCII ``MSK 014'' followed
by byte 0x01) at symbol HARDWARE\_ID, address 0xA800; followed by a table of
divisors at symbol NOTE\_TBL, address 0xA808.  The divisors are intended to
be used with output compare modules to provide musical-note frequencies. 
The firmware Makefile calls a Perl script to generate the file notetbl.inc,
which is imported to firmware.s by an include directive.

Later bytes of the last page, past the end of the note table, include a
human-readable copyright and version control ID; but be aware that if you
read these bytes out on a real-life module, they will represent what was
programmed into the chip the last time it underwent ICSP.  Firmware loaded
later by USB could have rewritten other parts of the program memory, so
might not be the same version identified by the notice on the last page.

\section{Power-on reset}

At reset the microcontroller starts executing code at symbol \_\_reset,
which begins by initializing the stack and the TBLPAG register.  If
FILL\_RAM\_DEAD is selected, it will fill all the general-purpose RAM with
the value 0xDEAD.  Then it opens a TRY/TRIED block with a handler that
points at a \insn{reset} instruction, just to catch stray THROWs executed by
other code, and calls STANDARD\_IO\_CONFIG (defined later this same file) to
set up most of the on-chip peripherals.

Next, the reset handler sets up PPS mappings for the SPI peripheral, and
calls CALIBRATION\_TO\_RAM from calibration.s to extract the curren DAC and
ADC calibration values.

Optionally, if requested by configuration symbols, it branches at this point
to RUN\_TESTS for tests from tests.s, or to \_USB\_MASS\_ENTRY to attempt
reading a simulated filesystem image.  Normal production firmware will
instead fall through into the non-USB behaviour.

\section{Non-USB behaviour}

This section of firmware.s contains the top-level loop that manages the
module's functions when there is no USB device inserted.  Before the loop it
calls USB\_INIT to set up the USB driver for detecting when a device is
inserted, sets up the PPS mapping to send output compare units 1 and 2 to
the left and right digital jacks, and does a couple of other small
initializations like setting the LED colour to red.

This code uses W8 for the current state of the envelope, encoded on a scale
where each count of the register value corresponds to $1/(12\times 256)$ of
a volt and 0x2400 is 0V.  That is the scale (MIDI note number in high byte,
fraction in low byte) used by the calibration API.

The non-USB behaviour (as described in the UBM) is a baby synthesizer voice:
V/oct pitch CV in on the left, gate CV in on the right, and quantized pitch,
envelope, and quantized and unquantized oscillator outputs.  While managing
these functions, the loop is also constantly looking for a USB device to be
inserted.  When one is, it branches off to the general USB driver to handle
the device.

The main loop starts by idling the microcontroller to wait for an interrupt,
which also refreshes the WDT.  It checks whether the USB driver has detected
a device attach, breaking out of the loop if so.  If the module booted up
with a device already attached, that fact will be detected as a device
attach on the first run through the loop.  Then if the loop did not break
for a device attach, it checks whether input 2 is high (gate CV high).

If the gate is low:  it turns off the LEDs and the unquantized oscillator
output, then conditionally on whether a 618\,$\mu$s tick has occurred (from
the ADC subsystem, used here for envelope timing) it updates the envelope
value -- release phase, ready for a new attack, with the envelope voltage
heading for zero.  After an envelope update it branches to
tune\_oscillators, the shared code for setting the oscillator frequencies.

If the gate is high: it turns on both LEDs, then similarly checks for a
618\,$\mu$s tick.  If there has been one since last update, it updates the
envelope, which is a little more complicated because it could be in the
attack, decay, or sustain phase.  Once the new envelope value is in W8, it
falls through to tune\_oscillators.

From the tune\_oscillators label, the code calls ADC1\_TO\_NOTENUM from
calibration.s to get the input voltage from the left channel, and then
CALC\_OSC\_TUNING (later in this same file, global because it may be useful
elsewhere) to find the output compate period value for the unquantized input
note using interpolation between values from NOTE\_TBL.  This is written into
output compare unit 1 conditionally on the gate input being high.

Then the note value, which is in W7, gets rounded to the nearest semitone by
adding 0x80 and zeroing the low byte.  That quantized value is sent to the
left DAC channel and looked up in NOTE\_TBL; no interpolation needed because
it is an exactly tuned note.  The resulting quantized period value is sent
to output compare unit 2, unconditionally.

Finally, the loop sends the current envelope value to DAC~2, and loops back
to wait for another interrupt.

The non-USB loop breaks out to non\_usb\_done when the USB driver tells it a
device has attached.  At that point it changes the LEDs to solid green,
turns off the output compare oscillators, sets up a TRY block with
catch\_usb\_session as the handler, and then calls USB\_HANDLE\_SESSION,
which in the ordinary course should do all the handling of enumerating and
configuring the device, choosing a driver, and running the device driver
until the device is removed.

In case of an exception thrown and not caught during the driver execution:
catch\_usb\_session sets up the LEDs to bring rapidly back and forth in
red, using calls to the LED blinker driver.  It waits for the USB driver to
report device detach, then falls through into the normal-case code run when
the USB driver returns without throwing.

Finally, there is a little bit of cleanup:  the top-level code calls
USB\_DONE and LEDBLINK\_DONE to clean up after the USB and LED blinker
subsystems, particularly by turning off relevant interrupts.  These calls
are harmless if the relevant drivers were in fact already shut down.  Then
it loops back up to the initialization before the main non-USB behaviour
loop.

CALC\_OSC\_TUNING is declared global in case it may be useful elsewhere.  It
calculates a period value to tune an output compare for a MIDI note, with
interpolation for fractional note numbers.  The input note number (high byte
is MIDI note, low byte is fraction) is expected in W0.  It copies that value
to W7, extracts the entries on either side of the fractional note from the
NOTE\_TBL structure, and then interpolates between them, using a tricky
\insn{mov.b}/\insn{swap} combination to divide the 24-bit intermediate value
by 256 in only two instructions.  The address of NOTE\_TBL is left in W6 as
a side effect.

\section{Basic I/O}

This file provides a few basic subroutines for I/O that it needs itself and
might be needed elswhere.  UNLOCK\_PPS does the necessary unlocking sequence
to allow changes to the PPS mapping registers; LOCK\_PPS, similarly, locks
them up again.  Note that there is no automatic handling of nesting for
these operations.  They just set the current state of the registers to
locked or unlocked, regardless of whether it duplicates the previous state,
and they trash W0, W3, and W4.

STANDARD\_IO\_CONFIG sets most of the on-chip peripherals to sane default
values.  It clears the soft interrupt flags (discussed below); sets the data
direction for the GPIO pins; and configures the interrupt priorities.  It
sets up Timers~1, 2, and~3 to their standard configurations, which are
intended to support LED blinking (Timers~1 and~2) and the ADC conversion
schedule (Timer~3).  It turns off Timers~4 and~5.

It sets up all the output
compare units to edge-aligned PWM mode, driven by the Timer~3 prescaler, but
with modulation values that will actually keep their outputs low all the
time.  It sets up the ADC subsystem to its standard configuration, scanning
the two analog input jacks and the USB bus voltage, one conversion per
Timer~3 reset, and one interrupt after each completed cycle of three
conversions (1.618\,kHz interrupt frequency).

Then it sets the comparators to look at the analog input jacks, with a
reference voltage equivalent to $+$1.62V at the jacks, and interrupts enabled
on rising edges through that voltage.  Note that in hardware testing, it was
apparent that interrupts are also sometimes generated on falling edges; this
is not a published erratum but may be related to one, and we work around it
in the ISR, which is in calibration.s.

Finally, STANDARD\_IO\_CONFIG sets up the SPI peripheral for talking to the
SRAM and DAC, and clears the soft interrupt flags a second time to deal
with any spurious interrupts that might have come in while the peripherals
were being reconfigured.

Two more global I/O routines set up PPS mappings used in firmware.s and
possibly of use elsewhere:  PPS\_MAP\_OC\_DOUT sets output compare units~1
and~2 to the two digital output jacks, and PPS\_MAP\_GPIO\_DOUT sets the
pins for these outputs back to plain GPIO mode.  Each of these routines
makes further calls to unlock, and then lock, the mapping registers around
its writes to the registers.

\section{A/D conversion and USB short detect}

The A/D conversion ISR is to some extent a safety feature, so it is expected
to be active pretty much all the time the module is powered up -- possibly
not during the special operations of calibration and firmware reloading.  It
extracts voltage measuerments from the ADC hardware buffer and writes them
to global variables.

Global variables defined here are SOFT\_INT\_FLAGS, INPUT\_ADC1, INPUT\_ADC2,
and USB\_VBUS\_ADC, each one word long.  The other three simply represent
raw ADC readings (10 bits each), updated by the ISR at 618\,$\mu$s
intervals, but SOFT\_INT\_FLAGS is used by several different subsystems that
need to wait for interrupts handled by specific ISRs; the ISR is expected to
set an appropriate bit in the variable, and then the foreground code can
check for that bit after coming out of idle to recognize when the desired
interrupt (as opposed to some other interrupt) has actually occurred. 
Constant values for bit numbers within this variable are defined by
symbols with names starting ``SI\_'' in global.inc.

As well as writing the current ADC readings to variables, the ADC ISR checks
whether the one for USB voltage is too low (indicating that a short circuit
or other problem has caused the polyfuse to trip) and sets the soft
interrupt flag SI\_VBUS\_TRIP, if the voltage is low and has been low
on more than 162 consecutive ADC interrupts, corresponding to 100\,ms.

Before returning, the ISR sets (unconditionally) the soft interrupt flags
for ADC1 and ADC2; separate flags because there could be two different
things in the foreground waiting for them, as for instance during
multi-threaded calibration.

The last thing in the firmware.s file is a subroutine called CHECK\_VBUS,
which is the consumer of the SI\_VBUS\_TRIP flag.  The foreground ought to
call this periodically when a USB device is attached, to shut down the
module in the event of trouble with the USB power.  It checks for whether
the ISR has set the flag, then if so, it shuts down the USB driver, PPS
maps the output jacks to plain GPIO, sets the LEDs to blink in a unique
red/green flicker pattern, and then waits, checking the bus voltage after
every interrupt, for the voltage to remain good for 1618 consecutive ADC
interrupts, corresponding to one second -- the pattern expected if the short
circuit is resolved and the polyfuse cools down.  Then it resets the module.

In practice, a power disruption serious enough to trip the polyfuse is
likely to also disrupt the microcontroller enough that it will be unable to
continue executing instructions until a power cycle, so the semi-graceful
reset contemplated by CHECK\_VBUS is mostly theoretical.  But this code
seems to give it the best chance of recovering under software control.  It
has to be called in the foreground (instead of the ISR just branching
directly into the recovery loop) because logic like the LED blinking also
uses interrupts and trying to keep all that running well without returning
from the ADC interrupt would be inconveniently complicated.

\section{Global include file}

Every .s file includes global.inc, which contains standardized definitions
used everywhere.  This file contains nested includes of Microchip's
p24Fxxxx.inc file, which defines things like register names, and config.inc,
which defines build-time options.  The build-time options are described in the
``Build environment and tools'' chapter of this manual.

These things are defined in global.inc:
\begin{itemize}
  \item the \insn{in\_common} macro for defining file-local static memory
    assignments in the common data area, with its support symbols;
  \item data structure sizes for the USB low-level support;
  \item BUFFER\_SAFETY\_MARGIN, the minimum number of extra bytes added to
    buffers to mitigate overruns by the USB hardware;
  \item STACK\_RESERVATION, the minimum amount of stack space that the mass
    storage buffer allocator will leave unconsumed;
  \item symbols starting SI\_ (``soft interrupt'') representing bit numbers
    in the SOFT\_INT\_FLAGS variable;
  \item symbols starting UF\_ (``USB flag'') representing bit numbers, and
    starting UFM\_ (``USB flag mask'') representing masks ($2^n$), for bits
    in the USB\_FLAGS variable;
  \item symbols starting EPF\_ (``endpoint flag'') representing bit numbers,
    and starting EPFM\_ (``endpoint flag mask'') representing masks, for
    bits in endpoint flags fields;
  \item symbols starting IRPF\_ (``I/O request packet flag'') representing bit
    numbers, and starting IRPFM\_ (``I/O request packet flag mask'')
    representing masks, for bits in IRP flags fields;
  \item symbols starting ERR\_ representing error codes used in IRP errors
    fields; and
  \item symbols starting PID\_ (``packet ID'') representing codes returned
    by the hardware to identify types of USB packets.
\end{itemize}
