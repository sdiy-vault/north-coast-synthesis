% $Id: qwertys.tex 9835 2022-02-13 02:57:36Z mskala $

%
% USB boot keyboard driver
% Copyright (C) 2022  Matthew Skala
%
% This program is free software: you can redistribute it and/or modify
% it under the terms of the GNU General Public License as published by
% the Free Software Foundation, version 3.
%
% This program is distributed in the hope that it will be useful,
% but WITHOUT ANY WARRANTY; without even the implied warranty of
% MERCHANTABILITY or FITNESS FOR A PARTICULAR PURPOSE.  See the
% GNU General Public License for more details.
%
% You should have received a copy of the GNU General Public License
% along with this program.  If not, see <http://www.gnu.org/licenses/>.
%
% Matthew Skala
% https://northcoastsynthesis.com/
% mskala@northcoastsynthesis.com
%

\chapter{USB boot keyboard driver (qwerty.s)}

The USB boot keyboard driver allows a common QWERTY (typing) keyboard to
function like a MIDI (musical) keyboard for controlling the Gracious Host. 
It also supports some extra features like tap tempo for the arpeggiator
channels, and \emph{maintenance codes} used in testing the module.

This driver is called the ``boot keyboard'' driver from the terminology used
by the USB standards.  They define a complicated protocol for \emph{human
interface devices} including keyboards, and then a simplified subset of the
keyboard behaviour that is easier to implement and almost universally used
in preference to the complicated protocol despite the standard's presenting
it as only meant for use during a personal computer's boot process.

\section{TPL entry and key tables}

The first items in the source file are a TPL entry, which recognizes USB
devices that expose an interface of class~3 (``HID''), subclass~1
(``boot''), protocol~1 (``keyboard''); and two tables of information
associated with key codes.

USB defines its own list of key codes, generally starting with the most
common keys that every keyboard has, in the lower code numbers, and then
proceeding toward more obscure keys in higher numbers.  Exactly what a key
code refers to (the legend written on the key; the location of the key on
the keyboard; or the character the key will send when typing) is not clear
and seems to be inconsistent.  Keyboards with unusual layouts may map the
keys to the codes in surprising ways.  Unfortunately, there is nothing the
Gracious Host can do about variations in keyboard layout; there is no
standard way for keyboards to report more detail about their layout than
just sending the key codes.  The Gracious Host just assigns a function to
each key code, with the diagrams in the UBM showing where those key codes
are most likely to appear on a typical consensus keyboard layout.

Key codes 0x00 to 0x03 are reserved and error codes.  Codes 0x04 to 0x1D are
the letters of the Latin alphabet.  Then codes 0x1E to 0x27 are numerals --
the main, typing numerals typically found on the row above the letters.  All
these codes are recognized in the key-handling code by checking the range
boundaries.

Key codes that require individual handling subroutines range from 0x28 to
0x64.  Starting from press\_tbl\_start in program memory is a table of
program memory addresses for the codes from 0x28 to 0x64.  For each code,
the address stored in the table is that of a handler to be called when the
key is pressed.  The instruction immediately before the press handler is the
start of the release handler, so by looking up the press handler and
subtracting two address units, the framework code finds the address of the
release handler.  At the release-handler address might be a \insn{return}
(if there is no processing needed for the key release); a \insn{nop}
(causing execution to fall through into the press handler, if the same code
can handle both); or a branch instruction pointing at a longer handler
elsewhere.

A second table of two-byte entries indexed by key code starts one
instruction before the note\_tbl\_start label and records, for each key that
is meant to play a MIDI note, the MIDI notes that key will play in
piano-style (Num Lock off) and isomorphic (Num Lock on) layout modes.  This
table covers key codes 0x04 through 0x38 for letters, numerals, and a few
punctuation marks, but also key code 0x64 at what is effectively index $-1$,
the instruction word immediately before the note\_tbl\_start label.  The low
byte of each entry is the MIDI note number for pressing the key in
piano-style mode and the high byte is for isomorphic mode.  A few key codes
that fall into the 0x04--0x38 range but are not note keys, such as 0x29 for
Esc, have dummy 0xFFFF entries.  Laying the table out this way, with some
special-case code to check for the exceptional index before the starting
address of the table, ended up saving program memory compared to other ways
the same information might be recorded.

Some assembler directives after the key tables check the length of each
table by subtracting its start and end addresses, and raise an error if a
table length is not as expected.  That is intended to guard against editing
errors which can easily leave the wrong number of entries in a table.

\section{Maintenance codes}

The typing keyboard driver has a general-purpose feature for activating
hidden and special firmware features which might be useful in testing and
debugging.  With a typing keyboard plugged in, the user can hold the Ctrl
and Alt keys and type in a four-digit decimal code on the numeric keypad. 
Then the keyboard driver jumps to an address associated with the code by a
table lookup.  Other parts of the firmware can add entries to the table for
code-accessible entry points they may define.

Two examples of what the table entries look like are in qwerty.s under the
heading ``Maintanance code table.''  The entry format is one word for the
four-digit code (in BCD, first digit should not be zero) followed by one
word for the address to jump to.  In this file, the code 8605 is defined to
jump to RESET\_INSN (simulating a reboot of the module) and 8189 is defined
to jump to THROW (simulating abnormal driver termination).

Other source files should define their maintenance codes in sections named
mtbl or starting with mtbl\_.  The linker will gather such sections in
memory between the \_mtbl\_init and \_mtbl\_done sections defined here; the
order of entries within the table is unimportant.  New code values should be
selected uniformly at random from the set of four-digit numbers not starting
with zero.  I chose mine by rolling dice.

The code that recognizes maintenance codes is described below in the
sections on the Ctrl, Alt, and keypad numeral keys.

\section{RAM data}

Most of the driver's variables defined in the common data area will be
described with the code that uses them, but it is worth noting that
FIND\_IN\_OUT\_ENDPOINTS, defined in this file, expects to find
int\_in\_ep\_ptr and int\_out\_ep\_ptr at the very start of the common area,
immediately followed by the array of endpoints it will search.  Other
drivers that share this code (in particular, the USB Mass Storage and USB
MIDI drivers) must also use this layout for compatibility.

\section{Driver initialization and main loop}

The code in FIND\_IN\_OUT\_ENDPOINTS is separated into a subroutine and
given a global label so that it can be shared by other drivers.  Its general
function is to scan the array of endpoints found by a previous call to
USB\_CONFIGURE\_DEVICE and find the first input and first output endpoint,
if any.  It writes pointers to these endpoint structures to int\_in\_ep\_ptr
and int\_out\_ep\_ptr, and also leaves them in W6 and W7.  The pointers are
null (0x0000) if no endpoint in the corresponding direction was found at
all.  This code will also terminate with a THROW if USB\_CONFIGURE\_DEVICE
found no endpoints at all.

The driver entry point at qwerty\_driver starts by calling
USB\_CONFIGURE\_DEVICE and USB\_SET\_BOOT\_PROTOCOL to select the
configuration and boot protocol.  Then it calls FIND\_IN\_OUT\_ENDPOINTS to
find the interrupt in endpoint, and the interrupt out endpoint if there is
one.  The in endpoint is necessary, and the code will terminate with a
branch to COMPLAIN\_ABOUT\_DEVICE if none was found.  The out endpoint will
be used if present but is not required, so that is not checked at this
point.

Next, it clears the common data variables.  Most of them are cleared to
zero, but the key\_notes array (recording the channel and note number
currently played by each pressed note key) is cleared to 0xFFFF because
0x0000 could in principle be a valid data value there.

After that the driver code sets up the IRP to point at its data buffer; sets
the polling delay to match what was requested by the device, but limited to
the range 2\,ms to 100\,ms (it is expected that most devices will request
10\,ms); initializes a couple more variables that need non-zero init values;
and starts up the MIDI backend.

The main loop starts at driver\_loop with a call to MIDI\_BACKGROUND so that
the MIDI back end can do its ongoing tasks.  It loops on that and a
\insn{pwrsav} instruction until USB\_LOOP\_CHECK returns nonzero, indicating
time for another poll of the interrupt endpoint.  Then it sets up the IRP
for a transfer of eight bytes from the interrupt in endpoint, and does the
transfer with a call to USB\_WAIT\_ON\_IRP.  The use of UF\_MIDI\_BKGND in
USB\_FLAGS means that USB\_WAIT\_ON\_IRP will also be calling MIDI\_BACKGROUND
inside its own loop.

Some USB keyboards will return an empty packet if there is no change from
the last update.  There is a check at this point for whether the packet is
less than eight bytes (which in practice means it'll be zero); in such
cases, it branches to handle\_lr\_shift, skipping most of the key
processing.

\section{Most modifier keys}

The first byte of data returned by the keyboard represents the status of
``modifier'' keys; these are the ones like Shift and Alt that are typically
held down while pressing other keys to change the other keys' effects.  The
eight bits of the byte represent pressed or unpressed status of Shift, Ctrl,
Alt, and what the USB document calls ``GUI,'' which is a modifier key that
usually has a computer or operating system vendor's logo on it (Apple,
Microsoft, etc.).

Code at this point in the driver loop handles Ctrl and Alt.  As of this
writing, GUI is ignored, and the Shift keys are handled later because they
have an effect (pitch bend) on every pass through the loop as long as they
are pressed, regardless of whether we actually received a report of a status
change from the keyboard on this particular pass.

The previous\_modifiers variable stores the value of the modifiers byte from
the last update.  That gets compared against the new value to detect whether
either Ctrl key is newly pressed.  If so, a new press of left Ctrl subtracts
12 from the variable named octave (which represents the current octave
shift, measured in semitones) and a new press of right Ctrl adds 12.  The
variable is limited to four octaves down or five octaves up, representing
the furthest shifts at which the keyboard layout will still be able to hit
at least a few valid MIDI note numbers.

Alt is currently only used in combination with Ctrl, for entering
maintenance codes.  So there is a check for whether both Ctrl and Alt are
pressed (either or both Ctrl keys, and either or both Alt keys).  If not,
the maintenance\_code variable gets cleared.  Digits get shifted into this
variable when the keypad numerals are pressed, but since every valid code
starts with a nonzero digit, the variable will never be able to contain a
valid code and cause something to happen unless Ctrl and Alt are held
throughout the code entry process.

\section{Regular typing keys}

Keys other than modifier keys are reported in the third through eighth bytes
of the keyboard's response packet.  (The second byte is ``reserved for OEM
use'' by the standard.)  The currently-pressed keys are listed in these
bytes, one byte per key, with the remaining bytes filled by zeroes.  It is
because of this data structure that the USB boot keyboard protocol is
limited to a maximum of six simultaneously-pressed keys.

This code makes use of an array called key\_flags, containing a word for
each of the 256 byte values, initialized to zero.  The previous value of the
six pressed-key bytes is kept in previous\_keys, and the logic around the
array runs as follows.  The flag checks are intended not only to detect new
presses and releases between the earlier update and the current one, but
also handle reasonably the case of a key code listed more than once in the
same update.

\begin{itemize}
  \item For each key in previous\_keys, set bit 1 of the corresponding word
    in key\_flags.
  \item For each key in the buffer (new update), set bit 0.  If it and bit 1
    were both previously zero, then we have a new press of this key; handle
    it (as described below).
  \item For each key in previous\_keys, clear bit 1.  If it was set, and bit
    0 is not set, then we have a new release of this key; handle it (as
    described below).
  \item For each key in the buffer, clear bit 0.  As a side effect of this
    loop, write the new value of the byte to previous\_keys.
\end{itemize}

The four loops above share a star section fragment called find\_key\_flags
which indexes into the (current or saved) buffer, and finds the appropriate
word of the key\_flags array.  In that same star section is a label called
second\_return\_insn, which is a \insn{return} instruction immediately after
another \insn{return}, used for no-op entries in the press handler table.

The second and third loops, which detect newly pressed and newly released
keys, share the find\_press\_tbl\_entry subroutine, which looks up the key
code in the press handler table with handling for out-of-range values.  It
returns NZ status if the key code has a press handler, and the address of
the press handler in W4 in that case.  For key codes less than 0x28 it
returns note\_press; others get looked up in the table.

Recall that the instruction word immediately before the press handler is
supposed to be the start of the release handler.  The second loop (handling
presses) calls the address that find\_press\_tbl\_entry returned in W4 on NZ
status; but the third loop (handling releases) decrements it first. 
Subroutines called this way do whatever is needed to handle the press or
release of the key in question.

\section{Left and right shift}

After handling presses and releases of ordinary (non-modifier) keys, the
main driver loop enters the code at handle\_lr\_shift, to process the pitch
bend effect of the Shift keys.  If it received an empty update from the
keyboard, processing skips to this point, because pitch bend should keep
happening on every update, even the empty ones.  The idea is that pitch bend
keeps going up at a fixed rate as long as right Shift is held, down as long as left Shift is
held, then returns toward zero at another fixed rate when neither is held.

This code looks at the previous\_modifiers variable, which at this point
contains the \emph{current} modifiers byte if the keyboard sent one, but
otherwise stores the last modifiers byte that the keyboard did send.  It
builds up the rate of pitch bend to apply in W0.  It sums negative
QWERTY\_PBEND\_RATE (set in config.inc) if the left Shift is pressed and
positive if the right Shift is pressed.  If both, those will cancel to zero. 
Then if W0 is zero and the current pitch bend value is nonzero, it sets W0
to $\pm$QWERTY\_PBEND\_RETURN (another configuration value from config.inc),
with sign opposite to the current pitch bend.

The pitch bend rate is measured in pitch bend units \emph{per millisecond},
so it gets multiplied by the interrupt endpoint's update interval (measured
in milliseconds) to get the adjustment to apply to the pitch bend.  Next,
the absolute values are checked:  if the adjustment would cause the pitch
bend to strictly cross zero (go from negative to positive, or positive to
negative), then it is reduced to only take the pitch bend to zero.

Finally, the new pitch bend value (old value plus adjustment) gets
calculated, with clamping to its 14-bit signed integer range; formatted
into a MIDI message for the current channel; and passed to
MIDI\_READ\_MESSAGE.


\section{Keyboard LED update}

The last step in the main driver loop is to update the keyboard LEDs.  The
variable named leds holds the desired new status of the LEDs, while
previous\_leds holds its value at the last update.  Bit~0 of leds
corresponds to the Num Lock LED, representing isomorphic mode, and is
maintained by the press/release code for the Num Lock key.  Bit~1
corresponds to the Caps Lock LED and is set by code at this point in the
main loop if the sustain variable is nonzero.  Bit~2 corresponds to the
Scroll Lock LED, which lights under circumstances summarized as octave shift
XOR beat flash.  It gets set if the BEAT\_FLASH variable low byte is
nonzero, which is true for 80\,ms at the start of each beat when the tempo
clock is running, and then it gets toggled if the octave variable is
nonzero.

The new value for leds is compared against previous\_leds, and if they
differ, then the keyboard must be told to update its LEDs.

If the keyboard exposes an output endpoint (recognized by int\_out\_ep\_ptr
nonzero, as set by FIND\_IN\_OUT\_ENDPOINTS) then it is preferable to set
the LEDs by writing to that endpoint, and the driver loop code does that,
setting up the IRP accordingly and calling USB\_WAIT\_ON\_IRP.  Without an
output endpoint, it calls USB\_SET\_REPORT instead, to send the LED update
command using the CTRL endpoint.  Either way, this is the end of the main
loop and it branches back to driver\_loop.

\section{Press and release:  note keys}

The rest of the source file consists of per-key handlers,
called by the main loop when non-modifier keys are pressed or released
according to the entries in the press table.  The release handler is
expected to start on the instruction immediately before the start of the
press handler.  These handlers should preserve W11--W15, but may trash
W0--W10.  On entry, the main loop leaves the key code in W3 (low byte, with
the high byte zeroed); \emph{twice} the key code in W0; and a pointer to the
flags word in W1.

The first pair of handlers is for keys that play MIDI notes.  Those include
the main alphabet letters, the numerals in the row above them, and most of
the punctuation keys clustered around the sides of the typical keyboard
layout.

When a note key is pressed, the main loop calls note\_press, which starts by
looking up the key in the note table, with special-case handling for key
code 0x64.  The lookup yields a word representing the key's MIDI note
numbers in piano-style and isomorphic layout modes.  Depending on the
current mode, the code chooses which byte to use and formats it into a word
in W10 with the note number in the low byte and the MIDI channel in the high
byte.

Next, it applies the octave shift by adding the octave variable (measured in
semitones) to the note number, clamping the result to the range 1--127 for
valid note numbers.  It records the current channel and note number for this
key in the key\_notes array, which keeps track of the note currently being
played by each key, if any.  The note played by a key is not necessarily the
same every time because of octave shift and isomorphic/piano mapping
switches, and the channel can vary as well.  In general, once a key starts a
note it will continue playing the same note on the same channel until
released even if the shift, mapping, and channel change while the key
remains pressed.

Some special handling is required for the sustain feature (Caps Lock,
discussed in the next section).  At this point the issue is that if the key
just pressed plays a note \emph{and channel} that is already being sustained
by the sustain feature, then it should not generate a new MIDI note on
message.  So there is a check first against the channel on which sustain is
active (stored in the variable named sustain); then whether the note that
this key is playing is already recorded in the sustained\_notes array.  If
the note is already being sustained, then the code to send the note on is
skipped.  But otherwise, the code sets up the arguments in W1 and W2 and
calls MIDI\_READ\_MESSAGE for a note on.

If the sustain state is 1, which corresponds to Caps Lock actually pressed
at the moment and not only locked on, and the current channel matches the
sustain channel, then new notes should become sustained.  In that case the
current note is recorded in the sustained\_notes array.  And that ends the
key pressed handler.

The key released handler is next in the source file at the note\_release
label, which is referenced by the branch immediately before note\_press so
that the key released code will be able to find it.  This handler starts by
setting the key's entry in the key\_notes array to the null value 0xFFFF,
recording the fact that this key is no longer playing a note.  The old value
is captured for use in the following checks.

The check for sustain is similar to the check in the press handler:  if
sustain is active, the note just released was in the sustained channel, and
it was actually one of the sustained notes recorded in the sustained\_notes
array, then this note should continue after the end of the keypress.  It
will instead get a note off message when sustain ends.  And in that case,
the note off message is skipped, with a branch to RETURN\_INSN that ends the
release handler.  But in other cases, there should be a note off message at
the end of the keypress.  The code sets up W1 and W2 for a note off
(actually zero-velocity note on) and tail calls MIDI\_READ\_MESSAGE.

\section{Press and release:  sustain (Caps Lock)}

The ability to press multiple keys at once and have them all register
correctly is usually called \emph{rollover} in the case of typing keyboards
and \emph{polyphony} in the case of music keyboards.  The USB boot keyboard
protocol can support at most six-key rollover for non-modifier keys
because there are only six bytes for key codes in the report format; and
because of limitations on the wiring and scanning of the switch matrix, most
USB keyboards are unable to really support even six simultaneous keys in at
least some combinations.  Quite often there exist combinations of just
two or three keys that cause the keyboard to either fail to report one of
the pressed keys, or report a ``ghost'' key that was not actually pressed. 
Furthermore, human anatomy limits how many keys the user can accurately
press at once even if the keyboard could register them all, and some musical
applications (like the quantizer modes) may create a demand for holding
notes longer than it is convenient to hold down a keyboard key.

To help resolve these issues the typing keyboard driver has a \emph{sustain}
feature activated by the Caps Lock key, which allows locking an arbitrarily
large set of notes to remain held as long as desired, without needing to
physically press many keys simultaneously.  The basic concept is that note
keys which overlap with a first press of Caps Lock become sustained, and
remain playing until a second press of Caps Lock.  More detail on the use of
the sustain key from the performer's point of view is covered in the UBM.

In more detail from a code perspective, the variable named sustain takes on
the values 0, 1, and 2 to track the current state in the following sequence
of events.
\begin{itemize}
  \item Startup state, sustain state 0:  note keys generate note on messages
    when pressed, note off messages when released.  Caps Lock LED off.
  \item At the moment of the first Caps Lock press:  LED goes on, sustain
    state becomes 1.  Current channel is memorized as the sustain channel,
    all currently-playing notes in that channel become sustained notes.
  \item During the first Caps Lock press:  any newly-played notes in the
    sustain channel generate note on messages and become sustained notes too.
    Keys released that were
    playing sustained notes in the sustain channel, do not generate note
    off messages.  Keys pressed that would play notes already sustained
    in the sustain channel, do not generate additional note on messages.
  \item At the first release of the Caps Lock key:  LED remains on, sustain
    state becomes 2.
  \item Before the next press of Caps Lock, while sustain state remains 2:
    note keys generate note on and off messages normally, except those
    corresponding to sustained notes in the sustain channel have no effect.
  \item At the second press of Caps Lock:  sustain state returns to 0.  Caps
    Lock LED goes off.  Note
    off messages are generated for all sustained notes, except those for
    which there is currently a note key actually pressed.  All sustained
    notes are cleared.
  \item The second release of Caps Lock has no additional effect.
\end{itemize}

Some support for this behaviour was in the note key press and release
handlers described above.  The rest is in the sustain\_press and
sustain\_release handlers, called on press and release of Caps Lock.  The
LED setting is done in the driver main loop, with a check of the current
state value.

The sustain\_press handler starts by checking whether the state is 0 or 2,
which determines whether this is the first or second Caps Lock press of the
sequence.  On the first press, it increments the state to 1, stores the
current channel in sustain\_channel, then scans the key\_notes array to find
currently-playing notes.  All those that are in the current channel (some
might not be, if the channel changed while the key was held) get recorded in
sustained\_notes, and then the handler returns.

On the second press, code at the unlock\_sustain label starts by clearing
the sustain state.  Then it scans key\_notes to find any notes in the
sustain channel that are currently being played by note keys; such notes are
\emph{removed} from sustained\_notes, because their note off messages will
be delayed until the keys are released.  Then it scans sustained\_notes and
sends note off messages to the MIDI backend for all the notes it finds
there, clearing the array as a side effect, before returning.

The sustain\_release handler is very simple:  it just checks whether the
state was~1 (indicating that it's the first Caps Lock keypress ending now)
and if so, increments it to~2.

\section{Press and release:  channel keys (F1--12 etc.)}

The function keys F1--F12, and four more keys (Print Screen, Scroll Lock,
Pause/Break, and Esc), correspond to the 16 MIDI channels; pressing a
function key switches the current channel, used by future note key presses,
to the associated channel.  The key codes for these keys are conveniently
arranged: 0x29 for Esc and 0x3A to 0x48 for the others.  So the press
handler for these keys, at channel\_press, subtracts 0x3A from the key code
to get the channel number (in internal format, where F1 and Channel~1
correspond to value 0), and then if the result is negative indicating the
Esc key was pressed, it substitutes 0x0F for Channel~16.  The result goes
into the variable named channel.

There is no release handler for these keys; the instruction before
channel\_press is a \insn{return}, shared with the end of sustain\_release.

\section{Press and release:  isomorphic mode (Num Lock)}

Pressing Num Lock toggles the isomorphic keyboard layout.  The press handler
is just a \insn{btg} instruction that toggles the Num Lock bit in the
variable named leds, followed by a \insn{return}, and there is no release
handler.  The note key handler looks at the Num Lock bit in leds to
determine which layout to use.

\section{Press and release:  velocity (keypad numerals)}

The keypad numerals 1--9 serve two purposes:  they set the velocity that
will be sent with note on events, and they enter digits of a maintenance
code.

The code at velocity\_press starts by subtracting 0x58 from the key code to
get the digit value.  That is multiplied by 14
and stored in the velocity variable.  Then (at
handle\_maintenance\_code, labelled for reuse by the keypad-zero handler) it
shifts the digit value into the low four bits of the maintenance\_code
variable, moving up whatever bits were already there.

Bearing in mind that maintenance\_code is constantly reset to zero when Ctrl
and Alt are not both held, and all maintenance codes start with a nonzero
digit, if the variable ever contains a complete maintenance code then that
means the user has gone through the full procedure of holding Ctrl and Alt
while typing the four digits.  So a loop at this point scans the maintenance
code table, checking the value of maintenance\_code against the codes in the
table.  If one matches, it jumps to the address associated with that code in
the table.  With no match, the handler returns.

\section{Press and release:  tap tempo (keypad Insert)}

The keypad zero/insert key's main function is to enter tap tempo commands
for the MIDI backend's timer.  It also serves as a zero when entering
maintenance codes.  The release handler is just a \insn{return}.  The press
handler calls MIDI\_TEMPO\_TAP, then tail calls handle\_maintenance\_code
with W3 cleared to enter a zero digit in the maintenance code.
