% $Id: testss.tex 9768 2022-01-17 15:14:02Z mskala $

%
% Test routines
% Copyright (C) 2022  Matthew Skala
%
% This program is free software: you can redistribute it and/or modify
% it under the terms of the GNU General Public License as published by
% the Free Software Foundation, version 3.
%
% This program is distributed in the hope that it will be useful,
% but WITHOUT ANY WARRANTY; without even the implied warranty of
% MERCHANTABILITY or FITNESS FOR A PARTICULAR PURPOSE.  See the
% GNU General Public License for more details.
%
% You should have received a copy of the GNU General Public License
% along with this program.  If not, see <http://www.gnu.org/licenses/>.
%
% Matthew Skala
% https://northcoastsynthesis.com/
% mskala@northcoastsynthesis.com
%

\chapter{Test routines (tests.s)}

It is often useful when debugging firmware to run specialized test code on the
microcontroller, either at full speed in the environment of the real
firmware, on the real chip but single-stepping with a debugging tool, or
on Microchip's simulator without using a real chip.  The tests.s file
collects subroutines written for testing and development.

Each of these is conditionally assembled, under the control of symbols set
in the config.inc file.  A normal ``production'' build of the firmware will
not actually contain any of them.  But during development it may be useful
to build in any or all of them by editing config.inc.  If SKIP\_TESTS is
\emph{not} defined, then the firmware will jump to the test code upon
boot-up instead of running its main loop.  Most of the tests also define
maintenance codes (see qwerty.s), allowing them to be triggered at run time
with a typing keyboard.  Note that if the relevant tests are not assembled
into the firmware, then the maintenance code definitions also will not be
assembled and so will not be available.  But if running under a debugger, it
may be most useful to simply use the debugger to override normal control
flow and jump to a test routine when desired instead of using a maintenance
code anyway.

\section{Calibration routine, code 5833}

The TEST\_CALIBRATION symbol simply assembles a branch to
CALIBRATION\_PROCEDURE in calibration.s, treating the always-included
calibration routine as a test in the same framework as the others.

\section{CRC32 test, code 2540}

Define TEST\_CRC32 to activate the CRC32 test, which uses the PIC24 CRC
hardware to compute the CRC32 value of the ten-byte ASCII string
``1234567890''; that is a standard test vector.  The 32-bit result ought to
be 0xCBF43926.  After doing the calculation, the test routine either colours
the LEDs green (on success) and loops to try again, or colours them red (on
failure) and stops in an infinite loop, repeatedly idling the CPU.  So if
the LEDs stay green while the CPU is running, that means it has not only
gotten the right answer once, but is repeating the test probably thousands
of times per second and getting the right answer every time (at least, until
the WDT possibly times out).  But rather than letting it run at full speed,
it is probably more useful to run this test while single-stepping, possibly
even in the simulator rather than actual hardware, to watch what happens in
the peripheral's registers.

Actually using the PIC24's hardware CRC peripheral is a bit complicated and
non-intuitive.  Microchip's documentation is confusing and the consensus
when I looked in their user support forum was that it is too hard to be
worthwhile, and better to just write a software implementation.  Although
the process is not very visible in the final test code, I found this test
code useful for getting the hardware to actually work -- by doing extensive
trial and error of modifications on this test routine until it succeeded,
then using that knowledge to inform the firmware's real CRC code in
loader.s.

\section{LED blinker test, code 3183}

Define TEST\_LED\_BLINKER for a simple test of the LED blinker driver
(ledblink.s).  It uses that code's API to set the two LEDs to a pattern of
varying long and short red and green blinks, then goes into an idle loop.

\section{MIDI stream test, code 1001}

The MIDI stream test, activated by TEST\_MIDI\_STREAM, sends bytes of data
to the MIDI back end driver as if they had been received from a USB MIDI
device (in single byte mode).  That can be useful for testing the back end
without needing to also have an input device that works.

The data stream to send is defined between
the local symbols midi\_stream\_data and midi\_stream\_end, and it will be
sent infinitely repeating with 150\,ms delay between bytes (as controlled by
USB\_LOOP\_WAIT, for which this routine is also a useful test).

The data stream in the distributed code, which could be modified by editing
the code, is 0x90 0x3C 0x66 0x80 0x3C 0x00, which corresponds to Note On,
Note Off, for Middle~C with velocity~102 in Channel~1.

\section{PRNG test, code 5879}

The PRNG test activated by defining TEST\_PRNG extracts random numbers from
the pseudo-random number generator and sends them to the DACs, producing
white noise on the analog outputs.  It uses the calibrated-voltage code path
and can also be a test of that.  There is a commented-out \insn{pwrsav}~\#1
instruction in the loop.  With that commented out, the loop runs at maximum
speed (allowing a test of how fast the PRNG actually is), but is liable to
be interrupted by the watchdog timer.  With the \insn{pwrsav}~\#1 put in,
the CPU idles waiting for an interrupt once each time around the loop,
which is useful for estimating how often interrupts tend to occur.

Scoping, or listening to, the noise outputs during this test may be of some
help in verifying that the PRNG really produces something like the desired
distribution.

\section{SPI test, code 9485}

SPI transactions can be tested by defining TEST\_SPI.  This test sends four
transactions, in a loop:  first a read from the low moby (addresses 0x00000 to
0x0FFFF) of the SRAM; then a write to DAC1; then a read from the high moby
(0x10000 to 0x1FFFF) of the SRAM; then finally a write to DAC2.  The
addresses, and the values sent to the DACs, increment each time around.  The
test point (P5) also goes high during the write to DAC1, low at other times
during the loop.

The idea is that this loop creates a pattern of signals on the pins of the
microcontroller that can usefully be probed with an oscilloscope.  There is
enough unpatterned traffic on the SPI data lines to create ``eye patterns''
useful for seeing that voltages and rise and fall times on the SPI bus are
as they should be.  By using the test point signal as a trigger, it is
possible to zoom in on particular transactions in the loop and see that the
DAC is getting the right data and the SRAM is responding appropriately to
signals.  And the DAC outputs can also be probed to see whether they are
giving reasonable-looking upward ramps.

\section{USB eye pattern test}

The PIC24 USB hardware is supposed to support an eye pattern test mode
activated by setting bit 15 in the U1CNFG1 register.  I have not been able
to find detailed documentation on what this feature actually does.  There is
a caution in the Microchip documentation saying not to activate it with a
real USB device connected.  It's reasonable to guess that it may drive the
USB data lines into oscillation to display eye patterns on an oscilloscope.

I wrote the USB\_EYE\_PATTERN\_TEST routine to activate the eye pattern test
mode on experimental hardware, but it didn't seem to do anything.  The code
remains for possible future experiments, conditionally assembled on defining
TEST\_USB\_EYE\_PATTERN, but it may not be very useful, so no maintenance
code has been defined.  This code does also toggle the digital output jacks,
and the LEDs between red and green (fast enough to appear yellow).

By putting an oscilloscope on either USB data line while sending data from a
USB device that sends data frequently (possibly a mouse for low speed or a
MIDI keyboard for full speed), with careful triggering and a persistent
display it is possible to display eye patterns without needing any special
testing mode on the host.  I am not even sure that USB eye patterns as such
are particularly useful for this hardware; I did not have any bugs for which
they helped in debugging.  But they are fun to look at, and scoping the USB
data lines was useful for several issues (in particular, related to
attach/detach and SOF/keep-alive generation) that were not debugged using eye
patterns as such.
