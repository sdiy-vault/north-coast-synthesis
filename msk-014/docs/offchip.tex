% $Id: offchip.tex 9810 2022-02-02 20:56:40Z mskala $

%
% Programmer documentation for Gracious Host off-chip peripherals
% Copyright (C) 2022  Matthew Skala
%
% This program is free software: you can redistribute it and/or modify
% it under the terms of the GNU General Public License as published by
% the Free Software Foundation, version 3.
%
% This program is distributed in the hope that it will be useful,
% but WITHOUT ANY WARRANTY; without even the implied warranty of
% MERCHANTABILITY or FITNESS FOR A PARTICULAR PURPOSE.  See the
% GNU General Public License for more details.
%
% You should have received a copy of the GNU General Public License
% along with this program.  If not, see <http://www.gnu.org/licenses/>.
%
% Matthew Skala
% https://northcoastsynthesis.com/
% mskala@northcoastsynthesis.com
%

\chapter{Off-chip hardware}

Here are some notes on the hardware features of the Gracious Host hardware
beyond the peripherals built into the microcontroller chip.  More detail on
the implementation of these is in the circuit explanation chapter of the
UBM; here, the focus is on the programmer's view.

\section{CV inputs}

The Gracious Host has two external inputs, J1 and J2 on the schematic,
intended for Eurorack CV.  Although in some modes of the firmware these are
used for digital (gate or trigger) signals, we normally handle them as
analog voltages with an intended useful range of 0--5V and input impedance
of 100k$\Omega$.  It should be safe
(in the sense of not causing damage to the module) to give them any voltage
in $\pm$12V.

The voltage from the jack goes through an inverting op amp and resistor
network that scale 0V at the jack to about 3.19V at the microcontroller pin
and 5V at the jack to about 0.665V at the microcontroller pin.  These
voltages translate to nominal ADC readings of 989 and 206 respectively,
and the default calibration data for the ADC interpolates between those
values.

This input scaling may seem not to make best use of the microcontroller
ADC's voltage range of 0.0V to 3.3V, but the design is constrained to
guarantee that in all cases of component and power supply tolerances, an
input range of 0V to 5V will translate to something within the ADC's useful
measurement range, and the op amp will be unable to drive the
microcontroller pin outside its absolute maximum bounds of $-$0.3V to $+$3.6V. 
Wide tolerances on the op amp's output voltage capabilities necessitate a
significant safety margin at both ends.

As discussed in the previous chapter, the ADC ISR in firmware.s writes the
raw ADC readings into global variables INPUT\_ADC1 and INPUT\_ADC2 at
1.618\,kHz update rate, and the ADC1\_TO\_NOTENUM and ADC2\_TO\_NOTENUM
subroutines in calibration.s can convert these (using the current
calibration) to a scale from 0x2400 corresponding to 0V to 0x6000
corresponding to 5V.  To read the state of an input as a digital bit, read
bit number COUT from the CM3CON register (for input~1) or CM1CON register
(for input~2).  These bits will be set for high input (higher than about
1.62V) and clear for low input.

\section{Analog outputs}

The upper of the two sets of output jacks, J3 and J4 on the schematic,
is intended for control voltages in the range 0V to 5V (upper end of the
range actually a little higher, but not calibrated).  These jacks are driven
by a Microchip MCP4822 12-bit DAC chip, through non-inverting op amps with a
nominal gain of 2.8.  The DAC chip has an internal reference voltage of
2.048V$\pm$2\%, so the maximum output voltage at the jacks is nominally
5.73V.  The minimum is zero, subject to whatever offset exists in the op
amp.

Note that the two units in the DAC chip are reversed relative to the jacks:
the DAC's V$_\textrm{A}$ is output~2, on the right, and the DAC's
V$_\textrm{B}$ is output~1, on the left.

This chip is controlled through the SPI bus (SPI1 on-chip peripheral,
described in the previous chapter).  To send a transaction, the steps are:
\begin{itemize}
  \item Assert RA1 (microcontroller pin~3) low to tell the DAC to listen to
    the SPI bus (necessary because this bus is shared between it and the
    SRAM), possibly with the instruction \insn{bclr} LATA, \#1.
  \item Send a 16-bit command word to the DAC through the SPI peripheral. 
    Bit 15 is 1 for DAC~B (output channel 1), 0 for DAC~A (output channel 2);
    bit 14 is don't care; bit 13 selects the gain mode, with 1 being
    recommended for 1$\times$ gain; and then the remaining bits are the DAC
    value, 0x000 to 0xFFF.
  \item Retract RA1, as with \insn{bset} LATA, \#1.
  \item If you're changing the output voltage, that doesn't happen just by
    sending the SPI transaction; you need to also strobe the DAC chip's
    $\overline{\textrm{LDAC}}$ input (microcontroller RA4, pin~12) low for
    at least 100\,ns (thus, at least two instruction cycles) and then both
    outputs will update at once.  Sample code looks like this:
\begin{tabbing}
\qquad\=\qquad\qquad\=\kill
\>\insn{bclr}\>LATA, \#4\\
\>\insn{nop}\\
\>\insn{bset}\>LATA, \#4\\
\end{tabbing}
\end{itemize}

Sending data through the SPI peripheral is a little finicky.  Although I
think the hardware is supposed to support 16-bit write as a single
operation, I have only gotten good results sending one byte at a time
(\emph{big endian}) to the SPI1BUF register with the \insn{mov.b} WREG,
SP1BUF instruction specifically.  Referring to WREG by the name W0 makes the
assembler produce a different opcode and I'm not sure they both work, even
though they should have identical effects.  I've also encountered similar
issues with the CRC32 peripheral and it's possible some of these cautions
are more relevant there than with SPI.

Then it's necessary to read back dummy bytes and wait for the SRXMPT bit in
SPI1STAT to detect the end of the transaction.  Instead of talking directly
to the hardware it is probably better for higher-level code to call
WRITE\_DAC1 and WRITE\_DAC2 for writing raw 12-bit DAC values, or
NOTENUM\_TO\_DAC1 and NOTENUM\_TO\_DAC2 for applying calibration data and
writing note numbers on the usual 0x2400 to 0x6000 scale.  All those
subroutines are in calibration.s, and they do the necessary handshaking to
talk to the chip properly.

The stability capacitors in the output amplifiers cut off their frequency
response at about 250\,kHz.  With the SPI bus running at 8\,Mbps, 16 bits per
transaction, some overhead, and the Nyquist limit shaving off another factor
of two, the CPU is unlikely to be able to drive the outputs to frequencies
higher than about 100\,kHz anyway, but that should be plenty fast enough.

It might be possible to get a few extra volts of output range by using the
2$\times$ gain mode of the DAC chip (bit~13 of the command word equal to 0
instead of 1).  In that case the DAC's output code range becomes 4.096V, but
because its power supply is only 3.3V, codes corresponding to higher DAC
voltages than 3.3V will be unusable.  With the 2.8 voltage gain in the op
amp circuit, it might be possible to program the module hardware to generate
control voltages up to about 9V this way, but with lower resolution than in
the standard 5V range.  I have not tested, and do not particularly
recommend, the use of 2$\times$ gain mode.

\section{Gate/trigger outputs}

The lower of the two sets of output jacks, J5 and J6 on the schematic, is
intended for digital~CV outputs (gates and triggers).  These are connected
to the microcontroller's pins 25 (RB14) and 17 (RB8) through non-inverting
op amp circuits with gain of 2.8.  As a result, the 0V and 3.3V logic levels
translate to 0V and 9.2V nominal voltages at the output jacks, which ought
to be plenty for triggering Eurorack modules while still also being low
enough not to damage or confuse reasonably well-designed modules.  As with
the analog outputs, the frequency response of these amplifiers cuts of at
about 250\,kHz, which is just enough that two 960\,$\mu$s trigger pulses
on exact 1\,ms boundaries (thus, separated by a low of 40\,$\mu$s) will be
clearly distinguishable from each other.

These signals can be driven in GPIO mode just by setting and clearing bits
14 and 8, respectively, of the LATB register.  The standard firmware also
sometimes PPS maps them to the output compare peripherals (OC1 for channel
1, OC2 for channel 2) in order to generate pulses or frequencies.  It is
probably never useful to tri-state the outputs; in that case the op amps
will see disconnected inputs, with unpredictable effects.

\section{LEDs}

The two LEDs are connected to microcontroller pins~16 (RB7) and~18 (RB9)
with bidirectional current driver circuits.  Each LED can be red, green, or
off, with the brightness in the two modes intended to be roughly equal
(which ends up meaning about 9.9\,mA in green mode, 2.4\,mA in red mode). 
When the microcontroller pin is tri-stated, the current should be zero (to
within the limits of op amp offset) and the LED will be off.  So the
register bits should be set as follows:

\begin{tabular}{cc|l}
  LATB & TRISB & LED state \\ \hline
  1 & 0 & green \\
  0 & 0 & red \\
  -- & 1 & off
\end{tabular}

Use bit~7 of the LATB and TRISB registers for the left LED and bit~9 for the
right LED.

In principle, you could get yellow output by switching rapidly between red
and green, and you could even use an output compare (possibly OC5, which is
not used for anything else by the current firmware) to accomplish the
switching without CPU intervention.  However, doing that means changing the
load on the power supply by several milliamps at the frequency you're doing
the switching, and if that's an audio frequency it may well end up causing
interference that will be heard in the outputs of other modules in the
synthesizer, given the poorly decoupled power systems many Eurorack users
install.  So, the standard firmware does not attempt to turn the LEDs
yellow, and if you write code for it, please do not do it in such a way that
I will be blamed.

There is a driver in ledblink.s, described in its own chapter of this
manual, for optionally displaying different patterns of on/off and red/green
blinking on the LEDs.

\section{SRAM}

The Gracious Host is equipped with a 128K SRAM chip, a Microchip 23LC1024,
attached to the SPI bus pins (SPI1 on-chip peripheral, described in the
previous chapter).  This chip is used during firmware update: the old
firmware reads the new image into the SRAM, then runs the ``loader,'' which
is a small subroutine free of dependencies that reflashes program memory
using the data in the SRAM.  This way, the large and complicated code needed
to talk to the USB device and decode the FAT filesystem, does not need to
remain in program memory during the re-flash operation, and every
field-programmable byte of program memory can be replaced by the new
firmware.

The SRAM chip is not, at present, used in any other way by the firmware; but
it is available for use in other ways by future or third-party firmware
versions.

The steps to send a command to the SRAM are similar to those for the DAC,
modified by the fact that the connection to the SRAM is two-way.
\begin{itemize}
  \item Assert RA3 (microcontroller pin~10) low to tell the SRAM to connect
    to the SPI bus, possibly with \insn{bclr} LATA, \#3.
  \item Send a command to the SRAM through the SPI peripheral.  Each starts
    with an 8-bit instruction and then possibly has other fields depending
    on the command.  Commands are summarized below and detailed in the
    23LC1024 data sheet.
  \item After transferring all data for the command through SPI,
    retract RA3, as with \insn{bset} LATA, \#3.
\end{itemize}

The usual cautions for SPI apply here.  Every byte written must be matched
by a byte read.  Normally every byte of live instruction or data will be
matched by a dummy byte of discarded garbage, and every byte of live data
returned by the SRAM must be triggered by writing a dummy byte, which will
be ignored.  As discussed in the previous chapter, a silicon erratum
affecting the SPI peripheral means that the SPITBF bit used for detecting a
full transmit buffer may not work properly, and I think it is safest never
to completely fill either buffer.

The SRAM is capable of ``double'' or ``quad'' serial modes, which are
modifications of the SPI protocol using two or four data lines to transmit
more bits per clock cycle.  The Gracious Host's microcontroller does not
support these modes and there are not enough pins available to do it by
bit-banging.  The SRAM also supports three different modes for the READ and
WRITE commands, selected by writing to a mode register with the WRMR
command.  In \emph{sequential} mode, which is the default, you specify a
starting address to read or write and then read or write as many bytes as
desired; they will automatically access consecutive addresses, wrapping
around from 0x01FFFF to 0x000000.  In \emph{byte} mode, you specify a byte
address and then read or write that one byte; longer transactions are not
allowed.  And in \emph{page} mode, you specify a starting address and read
or write a variable number of bytes as with sequential mode, but the
operation will wrap around at the end of the 32-byte aligned region of
memory containing the starting address.

Valid commands for the SRAM chip are as follows.
\begin{itemize}
  \item READ: send opcode 0x03, then a 24-bit address (big endian, top 7
    bits ignored), then read as many data bytes as appropriate.
  \item WRITE: send opcode 0x02, then a 24-bit address (big endian, top 7
    bits ignored), then write as many data bytes as appropriate.
  \item EDIO and EQIO:  opcodes 0x3B and 0x38 enter dual and quad serial
    mode respectively, but this is not a useful thing to do on Gracious Host
    hardware.
  \item RSTIO:  send opcode 0xFF to undo the effect of EDIO or EQIO,
    returning the chip to plain one-data-line SPI mode; should not be
    necessary because we never go into dual or quad mode anyway, but the
    firmware does this before each session of using the SRAM just in case.
    The SRAM chip is designed to recognize this command sent by plain
    one-data-line SPI even if it is in one of the other modes.
  \item WRMR:  ``write mode register.''  Send opcode 0x01, then the new
    value for the mode register.  The useful and legal values are 0x00 for
    byte mode, 0x40 for sequential mode, and 0x80 for page mode.
  \item RDMR:  Send opcode 0x05, then receive the one-byte current value of
    the mode register.
\end{itemize}

Because the current firmware does not use the SRAM much, it does not provide
a complete library for talking to the SRAM.  But there is example code to
imitate in usbmass.s and loader.s, and a globally available subroutine
called SPI1\_READ\_BYTE which may be useful.  That receives one byte from the
SPI peripheral, with appropriate handshaking to wait for the byte to arrive. 
It returns the byte in the low byte of W0, while preserving the former low
byte of W0 by swapping that into the high byte.  Calling this subroutine
twice receives a big-endian 16-bit value.

Remember that every real write should be followed by a dummy read, and every
real read should be preceded by a dummy write.  The total number of bytes
read and written must balance.  Trying to read when there has been no
corresponding write may lock up the firmware until the WDT resets it. 
Writing at least a little bit ahead of reads is necessary, but going too far
may cause bytes to be lost in the full buffer.  As long the excess of writes
over reads is kept in the range zero to seven bytes, there should be no
danger of buffer overflow or deadlock.  And the interface is sufficiently
fast in comparison to the processor's instruction speed, that attempting to
keep the buffers from emptying and maximize throughput, is probably not
worthwhile.

\section{ICD/ICSP header}

There are pads on the back board of the module for a 1$\times$6-pin, 0.1$''$
header connector called P4.  This would not be installed in a standard
build, but someone wishing to use Microchip's in-circuit tools for
programming and debugging Gracious Host firmware could add this header to a
module to have a place to plug in the debugging tool.  Note the index pin
(pin 1) of the header is at the \emph{top} of the board; it is important to
plug in the debugging tool right way round.

The ICD/ICSP header connects to pins 1, 4, and 5 of the microcontroller and
also to the 0V and $+$3.3V power supplies.  A USB debugging tool like a
PICkit will normally draw a fair bit of power and it is probably \emph{not}
advisable to let it draw that power from the Gracious Host; nor will it work
to power the Gracious Host from the debugging tool, because the Gracious
Host also needs $+$5V and $\pm$12V.  Ideally, you should power up both the
Gracious Host and the debugging tool, separately, before plugging them into
each other.

\section{Voltage regulator/bus access}

There are pads on the back board for an optional $+$5V regulator (7805
chip) which regulates the Eurorack $+$12V supply down to $+$5V, to supply
USB bus power and be further regulated down to $+$3.3V (by an external
regulator on the Gracious Host board) and from there to $+$2.55V for the
microcontroller core (by its internal regulator).  Adding the $+$5V
regulator is not recommended for production builds, but it could be useful
in a debugging situation when it is desired to power up the module with a
$\pm$12V bench supply.  The Gracious Host itself consumes at most about
30\,mA of $+$5V power (most when the front-panel LEDs are lit), but because
it also supplies $+$5V from this bus to the USB-connected device, the total
current drawn from the supply may be significantly more.

There is a jumper block on the back of the module (J9) for selecting whether
to connect the internal $+$5V bus to the Eurorack bus (for normal power-up
from a Eurorack $+$5V supply) or to the optional on-board regulator (if
installed).  The same jumper block also selects whether to connect the left
analog and digital output jacks, to the Eurorack CV/gate bus lines, to allow
the Gracious Host to control other modules by default with no front-panel
patching.  The jumpers are not directly readable in firmware; at most their
consequences might be noticed.  But do note that every viable configuration
requires at least \emph{one} jumper to be installed.  If the module is
unable to power up at all, check that there is a jumper installed on the
back.

For more information on the jumper settings and $+$5V power requirements,
see the UBM.
