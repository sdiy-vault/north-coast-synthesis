% $Id: mouses.tex 9801 2022-01-28 23:10:45Z mskala $

%
% USB boot mouse driver
% Copyright (C) 2022  Matthew Skala
%
% This program is free software: you can redistribute it and/or modify
% it under the terms of the GNU General Public License as published by
% the Free Software Foundation, version 3.
%
% This program is distributed in the hope that it will be useful,
% but WITHOUT ANY WARRANTY; without even the implied warranty of
% MERCHANTABILITY or FITNESS FOR A PARTICULAR PURPOSE.  See the
% GNU General Public License for more details.
%
% You should have received a copy of the GNU General Public License
% along with this program.  If not, see <http://www.gnu.org/licenses/>.
%
% Matthew Skala
% https://northcoastsynthesis.com/
% mskala@northcoastsynthesis.com
%

\chapter{USB boot mouse driver (mouse.s)}

The USB boot mouse driver is a relatively simple per-device driver and
serves as an example of the framework code also used by the others.

\section{TPL entry}

The source file starts with a few instructions assembled into a section
named til\_mouse, which the linker will automatically insert in the Targeted
Interface List.  These make a call to the support routine
TPL\_MATCH\_CLASS\_AND\_PROTOCOL with the register values for class~3
(``HID''), subclass~1 (``boot protocol''), protocol~2 (``mouse'').  The
session manager calls the TIL for each interface descriptor the device
exposes, and if it exposes one saying that the attached device is a boot
mouse, then this entry will direct execution to the entry point at label
mouse\_driver.

\section{Data structures}

After declaring some constants for mouse sensitivity, the range of notes
supported by the DACs, and the pattern of semitones in the diatonic scale,
this section declares a bunch of local variables in the common data area
using the in\_common macro.  There is also a small section of constant data
assembled into program memory, which serves as a template for initializing
the local variables.

\section{Driver init and mouse input}

The entry point at mouse\_driver starts by calling USB\_CONFIGURE\_DEVICE
and USB\_SET\_BOOT\_PROTOCOL from usb.s to tell the device to use the current
configuration descriptor and the boot mouse protocol.  These calls have the
side effect of cleaning up the \insn{lnk} stack frame left behind by the
session manager.  The boot mouse protocol uses a single interrupt input
endpoint, which is initialized by the support routines.

Next the code initializes the local common-data variables using the
template.  This includes setting up the IRP for the interrupt transactions
used to read the mouse position.  A few words of zeroes at the start of the
layout are initialized with a \insn{repeat}/\insn{clr} loop instead because
that is cheaper than having extra zeroes at the start of the template.

Next, the driver looks at the poll-frequency recommendation from the device,
which was stashed at offset 12 in the EP data structure by the support
routines.  It clamps this to be in the range 5\,ms to 200\,ms, then saves it
in W11.  The code in the main loop is sufficiently simple that most of the
higher-numbered working registers can be used as local variables, retaining
their values throughout the loop.

Finally before starting the main loop, it calls PPS\_MAP\_GPIO\_DOUT from
firmware.s to make sure that the digital output jacks are in GPIO mode, and
clears the first two bytes of the I/O buffer to make sure that the driver
starts with the buttons recorded as unclicked.  At least some mice send
zero-length reports when there is no change from the previous state, so
without this step the buttons might remain in an uninitialized state until
after the first click.

The main loop begins at label driver\_loop.  This starts with a call to
USB\_LOOP\_WAIT, which keeps the polls happening at the interval in
milliseconds that was stored in W11.  Next, it clears the bytes at offsets 1
and 2 in the buffer (the X and Y offsets), so that if the mouse returns a
short report, the firmware will see zeroes and not keep adding up the
offsets left behind by earlier full-length reports.

Then it calls USB\_WAIT\_ON\_IRP with the proper arguments to request an up
to 3-byte report from the mouse's interrupt in endpoint.

\section{Mouse report decoding}

The three bytes of the mouse report are as follows.
\begin{itemize}
  \item Offset 0:  button states, 1 for pressed and 0 for released; left,
    right, and middle buttons mapped to bits 0, 1, and 2 respectively.
  \item Offset 1: X offset from the previous position, signed 8-bit number.
  \item Offset 2: Y offset from the previous position, signed 8-bit number.
\end{itemize}

Mice often return short or zero-length results, in which case the remaining
bytes will not be changed by USB\_WAIT\_ON\_IRP.  The previous button state
left in byte~0, and the zeroes we explicitly wrote to bytes~1 and~2, will
then show through instead.

The code for this section starts by sign-extending the X byte to 16 bits and
calling a star section subroutine (shared with the Y-axis code below) to
scale it with MOUSE\_SCALE\_MULTIPLIER and MOUSE\_SCALE\_DIVISOR, two
constants that can be tweaked to get different sensitivity.  The result is
added to the abs\_x variable, and then clamped by another star section
subroutine to the range LOW\_NOTE to HIGH\_NOTE.  Then the same logic
operates on the Y coordinate.

The buttons local variable stores the previous two values of the button byte
from the mouse, old in high byte and new in low byte.  This gets updated:
the low byte is swapped into the high byte and the byte we just read from the
mouse is written into the low byte.  Next the code checks whether, between
the last report and the current one, the middle button has gone from
unclicked to clicked.  If so, it increments the quantization mode (initially
3, only the lowest two bits actually used).

The variables abs\_x and abs\_y are stored in the usual
semitone-and-fraction format used by many parts of the Gracious Host
firmware.  The high byte is the MIDI note number and the low byte represents
additional pitch above that, scaled at 256 units equal to one semitone.  The
code somputes semitone-quantized versions of abs\_x and abs\_y and stores
them in W6 and W7 respectively.  The algorithm for semitone quantization is
very simple:  add 0x0080 to the register value (half a semitone) and then
truncate by zeroing the low byte.

Then there is some logic to choose cases according to the current
quantization mode, looking at the bottom two bits of the mode variable.  For
modes~0 and~1 the left LED is red (as a matter of colour; turning it on and
off is handled later), and for modes~2 and~3 it is green.  The low bit of
the mode variable similarly controls the right LED.

\section{Mode 0 (smart quantize)}

Mode~0 is the ``smart'' quantization mode, and the most complicated one:  it
quantizes to a diatonic scale but tries to adjust that scale on the fly to
match the user's note choices.

The way it works is that a diatonic (major or minor) scale covers seven
consecutive positions in the circle of fifths, for instance,
F--C--G--D--A--E--B for C~major or A~minor.  The most naturally ``nearby''
scales are the ones that overlap almost entirely with the current one; for
instance, F~major removes B at the right and adds B\musFlat at the left,
whereas G~major removes F at the left and adds F\musSharp at the right. 
Each differs from C~major by only one note changed.

So if the user seems to be playing notes more at the left end of the
sequence, it suggests they really want to be playing in a flatter key than
this one and we should shift the window of allowed notes to the left,
whereas if they seem to be mostly playing notes toward the right, it
suggests they want to be in a sharper key and we should shift the window of
allowed notes to the right.  The way the driver detects these conditions is
by counting how many times the user has played the note at either end of the
scale (the subdominant and dominant, if we call this a major scale), as a
streak without playing the one at the other end.

The two extreme notes are the fourth and the seventh of the major scale: F
and~B in the case of C~major.  Note that the two notes B and~F are
harmonically as far apart as you can get in the same diatonic scale, and
they form a tritone and are usually thought to sound dissonant when played
together.  Musicians tend to avoid using these two notes -- or more
generally, the two notes at harmonic extremes of \emph{any} diatonic scale
-- in close proximity to each other, except in certain cases like the
dominant seventh chord where there is tension being created on purpose.

Play the fourth three times without playing the seventh, and smart
quantization shifts the scale to the subdominant.  Play the seventh three
times without playing the fourth, and smart quantization shifts the scale to
the dominant.  So starting from C~major, if you play~F three times without
playing~B, then~B is removed from the scale and replaced by B$\musFlat$,
shifting the key to F~major.  You have musically implied that you don't
really want to use~B and would prefer to have more notes harmonically close
to~F.  On the other hand, if you play~B three times without playing~F,
then~F is removed from the scale and replaced by~F$\musSharp$, shifting the
key to G~major; you have implied that you would prefer to have more notes
close to~B.  If you play both~B and~F in close proximity without heavy
emphasis on just one of them, then you are implying that you really do want
to play in C~major after all, or A~minor, which is the same set of notes,
those being the only diatonic keys that can accommodate both~B and~F; and
doing this prevents the smart quantizer from changing the key.  The result
is that the key will shift to accommodate the notes the user is playing, as
comfortably as possible, and other notes offered as quantization choices
will always be notes that harmonize well with the notes the user is
choosing.

I don't know how musically useful this mode really is, but it seems to
produce reasonable results in practical tests, and that description may shed
some light on what it is attempting to achieve.  The basic idea is just that
you can plug in a mouse, mess around with it, and have something come out
that sounds musical, despite the difficulty of making accurate note choices
with the mouse as controller.  Obviously, users who want a more precisely
controllable quantization mode can use some other one; but they are also
likely not to want to use a mouse as a performance controller anyway.

The implementation starts by calling quantize\_to\_key, a chunk of code
shared with mode~1 below.  It finds key-quantized versions of the X and Y
coordinates, that is, the closest notes in the diatonic key instead of just
the closest semitones, using the variable named ``key'' to choose which key. 
The variable named key is kept as a modulo-12 number in the \emph{high}
byte, zero low byte, for compatibility with semitone-and-fraction pitch
values.  But the key value is \emph{added} to the pitch before quantization,
so it has to be the negative, modulo 0x0C00, of the desired root note.  For
example, the key of C$\musSharp$~major is represented by 0x0B00.  Adding
that to a semitone-and-fraction number for any C$\musSharp$ will give a
round multiple of 0x0C00.

The quantized-to-key absolute notes for the X and Y coordinates go
in W6 and W7, but the \emph{relative} notes within the key are saved in W9
and W10.  Those are integers in the range 0 to 11, counting semitones from
the root of the major scale without the octave.  For instance, the note E
will give a value of 4 in the key of C~major/A~minor.  Due to the
quantization to the diatonic scale, the only possible values for W9 and W10
after quantize\_to\_key are those in $\{0,2,4,5,7,9,11\}$.

Then the code checks the left button for whether it is newly pressed.  If
so, we have a new note to count on the X coordinate.  If W9 is 5 (fourth of
the major scale, like~F in C~major) then we add one to the fourths counter
and zero out the sevenths counter.  If W9 is 11 (seventh of the major scale,
like~B in C~major) then we add one to the sevenths counter and zero the
fourths counter.  So, either way, we have a count of the length of the
longest streak of fourths or sevenths without the other.  The same logic is
applied to the right button.  Note they share the same set of fourths and
sevenths counters.

Next the fourths and sevenths counters are checked for whether there has
been a streak of three fourths or three sevenths.  If not, the handling for
mode~0 is over and the code branches to do\_output.  If there has been such a
streak, the key is shifted accordingly, by adding either 0x0700 for a shift
to the subdominant after three fourths, or 0x0500 for a shift to the
dominant after three sevenths.  Either way, the key is reduced modulo 0x0C00
to keep it in the range 0x0000 to 0x0B00.  Then there is a branch to
clr\_fourths\_and\_sevenths, which is code shared with mode~1 that clears the
fourths and sevenths counters before branching to do\_output.

\section{Mode 1 (quantize to C major)}

Mode~1 is essentially just mode~0 with the key forced to stay in C~major and
the fourths and sevenths counts held at zero.  It starts by zeroing the key
variable, then calls quantize\_to\_key, the star section shared with mode~0,
included and described here to reduce the complexity of the mode~0 code and
description.

At the start of quantize\_to\_key, W8 is initialized with the constant bit
mask from DIATONIC\_SCALE, which is 0x0AD5.  That value answers the
question, for each bit number $i$ from 0 to 11, of whether a note $i$
semitones above the root of a major scale is or is not in the scale.

If a note was quantized to a scale note then it can remain there, but if it
was quantized to a non-scale note then it must go up or down an additional
semitone, which guarantees it will land on a scale note because every
non-scale note is surrounded by scale notes one semitone above and below. 
In order to make sure that it ends up as close as possible to the original
unquantized pitch, we want this new semitone shift to be in the opposite
direction from the rounding that was done to get the pitch to an exact
semitone in the first place.

So the code takes the quantized X-coordinate note from W6, adds the key
variable (always zero in mode~1, but possibly nonzero when this same code is
called from mode~0), and then calls mod\_c00, a utility routine used in
several places that computes the remainder of the unsigned 16-bit number in
W0 when divided by 0x0C00.  The significance of 0x0C00 (3072 decimal) is
that it is one octave (twelve semitones) on the scale used throughout the
Gracious Host firmware where one semitone is 0x0100.  So by reducing W0
modulo 0x0C00, we get a number representing where W0 is within its octave,
with the octave itself removed.

The high byte of the result says how many semitones W6 was above the root of
the current key.  That is used to choose a bit in W8 to find whether W6 was
quantized to a scale note.  If so, W6 can be used directly as the
scale-quantized note.  If not, then W6 is compared against abs\_x, which was
the original unquantized pitch value, to determine which way the rounding
went to quantize the note.  It gets adjusted one more semitone in the
opposite direction, and under the circumstances that guarantees it will end
up on a scale note.  The ``round in opposite direction'' operation is done
in another star section subroutine named round\_flip.

Having done all that to quantize W6 onto a scale note, the code goes through
the same logic with W7 for the Y coordinate, and quantize\_to\_key (the code
shared with mode~0) returns.

In mode~1, the last step after quantize\_to\_key is to zero out the fourths
and sevenths counters, so that if the user ends up switching back to mode~0,
they will be starting with a clean slate.

\section{Modes 2 and 3 (semitone and unquantized)}

By the time the code for modes~2 and~3 is reached, most of the work of
choosing notes is already done, so this code is just a few conditionals to
distinguish between the two remaining modes and set the front-panel LED
colours.  The code starting at do\_output expects to find the current notes
in W6 and W7.  For semitone quantization, those registers already contain
the right values and after setting the LED colours we can just branch to
do\_output.  For unquantized notes in mode~3, it first copies abs\_x and
abs\_y back into W6 and W7 and then falls through.

\section{Result output}

Conditionals starting at do\_output handle turning the LEDs on or off (their
colours were already chosen earlier, by the per-mode logic) and raising or
lowering the gate outputs.  Button~1 controls the left gate output directly. 
The left LED goes on if buttons~1 or~3 are pressed.  Also, when the left
button is \emph{not} pressed, the variable abs\_x gets copied into W6 --
pitch output is unquantized when the button is not pressed.  Then the same
logic repeats \emph{mutatis mutandis} for button~2 and the right LED and
gate.

At the bottom of the main loop, W6 and W7 contain the semitone-and-fraction
pitch values for the X and Y coordinates, with the appropriate quantization
if any.  The code calls NOTENUM\_TO\_DAC1 and NOTENUM\_TO\_DAC2 from
calibration.s to send these values to the analog output jacks, and then
loops back to driver\_loop.
