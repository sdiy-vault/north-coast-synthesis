% $Id: ledblinks.tex 9726 2021-12-20 19:14:48Z mskala $

%
% LED blinker utility
% Copyright (C) 2022  Matthew Skala
%
% This program is free software: you can redistribute it and/or modify
% it under the terms of the GNU General Public License as published by
% the Free Software Foundation, version 3.
%
% This program is distributed in the hope that it will be useful,
% but WITHOUT ANY WARRANTY; without even the implied warranty of
% MERCHANTABILITY or FITNESS FOR A PARTICULAR PURPOSE.  See the
% GNU General Public License for more details.
%
% You should have received a copy of the GNU General Public License
% along with this program.  If not, see <http://www.gnu.org/licenses/>.
%
% Matthew Skala
% https://northcoastsynthesis.com/
% mskala@northcoastsynthesis.com
%

\chapter{LED blinker (ledblink.s)}

The LED blinker driver in ledblink.s is a small, simple assembly language
module that may be useful as an example of how modules are typically
structured.  It provides automatic blinking of the module LEDs, placing
little load on the foreground code.  Several other modules use this driver.

The LED blinker steps through 16 states at a rate of 15.258\,Hz, which
works out to 1.048\,s to complete the full cycle.  For each state, you can
choose each of the two LEDs to be red, green, or off.  That allows for a
wide range of different blink patterns.

\section{API}

The API is summarized in the code comments near the start of the file.  Call
LEDBLINK\_INIT to activate the driver.  Set the blink pattern by writing
chosen values into the global variables LEDBLINK\_TRIS7, LEDBLINK\_TRIS9,
LEDBLINK\_RB7, and LEDBLINK\_RB9.  When LED blinking is no longer desired,
call LEDBLINK\_DONE.

In more detail:  the driver cycles through the 16 bits of all the global
variables, starting from the LSB and moving to the MSB.  In each state, the
TRIS bits (``tri-state'') specify whether the corresponding LED will be on
or off (bit value 1 for off).  The RB bits (``register B'') specify whether
the LED will be green (RB bit value 1) or red (RB bit value 0).  The
registers with ``7'' in their names refer to the left LED and those with
``9'' in their names refer to the right LED.  These variable names are chosen
to line up with the microcontroller register and pin names (TRISB, LATB,
RB7, RB9) used when directly accessing the LED hardware.

For example, to turn off the LEDs entirely, set LEDBLINK\_TRIS7 and
LEDBLINK\_TRIS9 to 0xFFFF, as with the \insn{setm} instruction.  To
blink the LEDs back and forth slowly, in green, set LEDBLINK\_RB7 and
LEDBLINK\_RB9 to 0xFFFF; set LEDBLINK\_TRIS7 to 0x00FF, and set
LEDBLINK\_TRIS9 to 0xFF00.  Setting LEDBLINK\_TRIS7 and LEDBLINK\_TRIS9 both
to 0x00FF would make the LEDs blink together instead of alternately.  To
make just one LED toggle very fast between red and green, not shutting off
at all, clear its TRIS variable and set its RB variable to 0x5555.

\section{How it works}

This driver uses general-purpose Timers~1 and~2 to keep track of the current
state of LED blinking.  That may seem an unecessarily large use of
resources, but the module has few other demends on these general-purpose
timers.

Timer~1 is driven by the 16.000\,MHz instruction clock prescaled by 1:256,
for an input clock frequency of 62.500\,kHz.  Its period is the maximum
(65536 counts), giving a reset rate of 0.954\,Hz:  the overall cycle time of
the LED blinker.  Timer~2 gets the same 62.500\,kHz prescaled clock, but
with a period of 4096 counts, for a reset rate of 15.259\,Hz.  These
configuration settings are put in place by STANDARD\_IO\_CONFIG in
firmware.s.

The idea is that every time Timer~2 resets, we will update the LED state,
looking at the high bits of Timer~1 to find out which state we are in.  We
don't want those bits to be changing while we look at them, so we arrange
for the two timers to reset 2048 counts (half of Timer~2's period) apart
from each other.  That way we can be pretty sure the ISR for Timer~2 will be
running while the high bits of Timer~1 are quiet -- despite the fact that
these timer ISRs run at low priority (priority 2, set in firmware.s).  Even
if there were a synchronization problem, reads from the timers are atomic,
and the extent of getting the wrong value would just be the LEDs showing the
wrong state for about 1/15 of a second.

The LEDBLINK\_INIT subroutine is straightforward:  it just sets all the
control variables to 0xFFFF (LEDs off, and will default to green if turned
on) and turns on the interrupts.  The LEDBLINK\_DONE subroutine is similarly
very simple:  it just disables the interrupts and turns off the LED
hardware.

The ISR for Timer~1 forces Timer~2's count to 2048 (halfway through
Timer~2's period), about once per second just to keep them in the proper
sync.  A slightly interesting point is that it accomplishes that by using
the \insn{clr} and \insn{bset} instructions, to avoid overwriting and thus
needing to save and restore a working register.  The operation is not atomic
but doesn't need to be; the counter only counts at one count per 256
instructions, and one stray count that might get in between the two
instructions will not harm anything.

The ISR for Timer~2 does the real work of LED blinking.  It grabs the top
four bits of Timer~1's state, uses them to index into the four global
variables, and sets the hardware bits that control the LEDs accordingly.  It
also sets the SI\_BLINK1 and SI\_BLINK2 flags, which are used by the
calibration routine for timing pauses; two of these flags because each of
the two threads may want to do it independently.
