% $Id: general.tex 9373 2021-08-30 21:03:39Z mskala $

%
% MSK 012 manual introduction/general stuff
% Copyright (C) 2017, 2021  Matthew Skala
%
% This program is free software: you can redistribute it and/or modify
% it under the terms of the GNU General Public License as published by
% the Free Software Foundation, version 3.
%
% This program is distributed in the hope that it will be useful,
% but WITHOUT ANY WARRANTY; without even the implied warranty of
% MERCHANTABILITY or FITNESS FOR A PARTICULAR PURPOSE.  See the
% GNU General Public License for more details.
%
% You should have received a copy of the GNU General Public License
% along with this program.  If not, see <http://www.gnu.org/licenses/>.
%
% Matthew Skala
% https://northcoastsynthesis.com/
% mskala@northcoastsynthesis.com
%

\chapter{General notes}

This manual documents the MSK~012 Transistor ADSR, which is an envelope
generator for use in a Eurorack modular synthesizer.
It implements a traditional attack-decay-sustain-release envelope without
using any integrated circuits, performing all the logic and other active
functions with discrete transistors and diodes.

\section{Specifications}

The module's peak current requirement in ordinary use is 10mA on the $+$12V
supply and 5mA on the $-$12V supply; most of the time it will draw much
less.  Unusual loads on the output, including so-called ``passive''
modules, may cause the MSK~012 to draw more than this amount of current.  It
does not require $+$5V power.

The input impedance is nominally 100k$\Omega$.  The maximum output impedance
is about 3.5k$\Omega$ (low side) or 1k$\Omega$ (high side).  Shorting the
input or output to any fixed voltage at or between the power rails should be
harmless to the module; patching the MSK~012's output into the output of
some other module on the same power system should be harmless to the
MSK~012, though doing that is not recommended because it is possible the
other module may be harmed.

Attack time is adjustable from about 40$\mu$s to 1.9s in three ranges; decay
time from about 6$\mu$s to 6.8s; and release time from about 8$\mu$s to
9.5s.  All these values were measured on a prototype and may vary a little
with component tolerances.  The attack curve is concave downward and decay
and release are concave upward.

The peak voltage of the envelope is fixed at about 8V; sustain voltage
ranges between 0V and the peak.  The input uses a Schmitt trigger (not
dependent on fast edges), turning on at about 2V and off at about 1V.  The
input is treated as a gate, not a trigger:  when it goes low, the module
immediately goes into the release phase (whether it was previously in
attack, decay, or sustain) and when it goes high, the module
immediately goes into the attack phase (regardless of whether the release
had completed).

This module (assuming a correct build using the recommended components) is
protected against reverse power connection.  It will not function with the
power reversed, but will not suffer or cause any damage.  Some other kinds
of misconnection may possibly be dangerous to the module or the power
supply.

\section{Front panel controls and connections}

Here's a summary of the items on the front panel of the module.  All four
knobs are logarithmic (``audio'') taper.

\begin{description}
  \item[A knob] attack time

  \item[D knob] decay time

  \item[S knob] sustain level

  \item[R knob] release time
  
  \item[range switch] sets the general range of the three timing knobs, up
    for slow and down for fast (labelled as such), or in the unlabelled centre
    position for very fast

  \item[G input] gate control voltage
  
  \item[E output] envelope control voltage
\end{description}

\section{Source package}

A ZIP archive containing source code for this document and for the module
itself, including things like machine-readable CAD files, is available from 
the Web site at 
\url{https://northcoastsynthesis.com/}.  Be aware that actually building
from source requires some manual steps; Makefiles for GNU Make are provided,
but you may need to manually generate PDFs from the CAD files for inclusion
in the document, make Gerbers from the PCB design, manually edit the .csv
bill of materials files if you change the bill of materials, and so on.

Recommended software for use with the source code includes:
\begin{itemize}
  \item GNU Make;
  \item \LaTeX\ for document compilation;
  \item LaTeX.mk (Danjean and Legrand, not to be confused with other
    similarly-named \LaTeX-automation tools);
  \item Circuit\_macros (for in-document schematic diagrams);
  \item Kicad (electronic design automation);
  \item Qcad (2D drafting); and
  \item Perl (for the BOM-generating script).
\end{itemize}

The kicad-symbols/ subdirectory contains my customised schematic symbol and
PCB footprint libraries for Kicad.  Kicad doesn't normally keep dependencies
like symbols inside a project directory, so on my system, these files
actually live in a central directory shared by many projects.  As a result,
upon unpacking the ZIP file you may need to do some reconfiguration of the
library paths stored inside the project files, in order to allow the symbols
and footprints to be found.  Also, this directory will probably contain some
extra bonus symbols and footprints not actually used by this project,
because it's a copy of the directory shared with other projects.

The package is covered by the GNU GPL, version 3, a copy of which is
included in the file COPYING.

\section{PCBs and physical design}

This module is built on a single PCB
4.20$''$\linebreak[0]$\times$\linebreak[0]1.25$''$, or
106.68mm\linebreak[0]$\times$\linebreak[0]31.75mm, which mounts
perpendicular to the front panel.  With about another 1mm
of gap between the PCB and panel, the total depth requirement is 33mm.

\section{Component substitutions}

This circuit should work with most general-purpose bipolar transistors; I
specify 2N5088 and SS8550D (formerly, PN200A).  These are high-gain types. 
I use these same transistors in other North Coast projects.  The output
buffer transistors (Q4 and Q5) are the ones where the gain matters most; be
careful substituting those.

You can substitute other capacitor values to change the overall speed of the
envelopes.  However, I would not recommend using electrolytics because of
their higher leakage, which could be a problem for the long envelope times
where they'd be most likely used; and although substituting a smaller value
for C3 to make the fastest envelopes even faster will work up to a point, I
don't recommend it.  Most people who think they want extremely fast
envelopes are wrong.  The problems they think they can hear that they think
are caused by the envelope not being fast enough are more often caused by
the envelope being \emph{already much too fast}, and will not be solved by
making it even faster.  Also, I encountered some stability problems with
this module's circuit on the breadboard with a value of 0.01$\mu$F for C3,
specifically that the module would go into oscillation when the
\emph{decay} (not attack) was set to its absolute minimum time.  I
couldn't reproduce that with a module built on a PCB, so I think it was
caused by stray impedances on the breadboard, and the final design value for
C3 is 0.033$\mu$F anyway (to make the longer times in the fastest range more
convenient).  But it suggests that there are some limits to how far into the
ultrasonic it's reasonable to push this module's timing.

\section{Modification for $\pm$15V power}

To modify the circuit to run on $\pm$15V power while keeping roughly the
same output voltage range, make sure all components are rated for the
increased voltage, and try changing these resistors.

\begin{itemize}
\item R7 (upper limit of sustain level):  change from 20k$\Omega$ to
  33k$\Omega$.
\item R16 (sensitivity for end-of-attack detector):  change from 27k$\Omega$
  to 18k$\Omega$.
\end{itemize}

For a higher output peak on $\pm$15V power, R7 will need to decrease and R16
increase; for a lower peak, the other way around.  These values were
determined by simulation, not testing a real instance of the circuit, so
some breadboarding and experimentation may be necessary.

\section{Use and contact information}

This module design is released under the GNU GPL, version 3, a copy of which
is in the source code package in the file named \texttt{COPYING}.  One
important consequence of the license is that if you distribute the design to
others---for instance, as a built hardware device---then you are obligated
to make the source code available to them at no additional charge, including
any modifications you may have made to the original design.  Source code for
a hardware device includes without limitation such things as the
machine-readable, human-editable CAD files for the circuit boards and
panels.  You also are not permitted to limit others' freedoms to
redistribute the design and make further modifications of their own.

I sell this and other modules, both as fully assembled products and
do-it-yourself kits, from my Web storefront at
\url{https://northcoastsynthesis.com/}.  Your support of my business is what
makes it possible for me to continue releasing module designs for free. 
Even if you only use the free plans and cannot buy the commercial products I
sell, any assistance you can offer to increasing the profile of North Coast
would be much appreciated.  For instance, you might post photos of your
completed DIY build on your social media.  The latest version of this
document and the associated source files can be found at the North Coast Web
site.

Email should be sent to\\ \url{mskala@northcoastsynthesis.com}.
