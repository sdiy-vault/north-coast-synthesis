% $Id: general.tex 5679 2017-10-13 15:24:08Z mskala $

%
% MSK 008 general notes
% Copyright (C) 2017  Matthew Skala
%
% This program is free software: you can redistribute it and/or modify
% it under the terms of the GNU General Public License as published by
% the Free Software Foundation, version 3.
%
% This program is distributed in the hope that it will be useful,
% but WITHOUT ANY WARRANTY; without even the implied warranty of
% MERCHANTABILITY or FITNESS FOR A PARTICULAR PURPOSE.  See the
% GNU General Public License for more details.
%
% You should have received a copy of the GNU General Public License
% along with this program.  If not, see <http://www.gnu.org/licenses/>.
%
% Matthew Skala
% https://northcoastsynthesis.com/
% mskala@northcoastsynthesis.com
%

\chapter{General notes}

This manual documents the MSK~008 Dual VC Octave Switch, which is a control
voltage processing module for use in a Eurorack modular synthesizer.  Each
of the two channels in the module contains a specialized quantizer, a toggle
switch that adds or subtracts a one-volt offset, and a precision voltage
adder with optional subtraction function.  The module's main function is to
transpose a melody represented by a control voltage, but it can also be used
for several other purposes.

\section{Specifications}

The impedance of the quantizer inputs is high, several hundred kiloohms at
least, depending on the setting of the offset switch.  For the other inputs,
the impedance is 100k$\Omega$ per input; when a signal drives two
inputs through normalization, it will see a 50k$\Omega$ impedance.  The
impedance of the outputs is very low, with current limited by 1k$\Omega$
resistors inside the op amp feedback loops.

Any input voltage between the power supply rails (nominally
$\pm$12V) is safe for the module; output voltages are limited by the
capabilities of the op amps to about $\pm$10V and will clip if the input
voltages sum to a result outside that range.

Shorting any input or output to any fixed voltage at or between the power
rails, or shorting two to each other, should be harmless to the module.
Patching the MSK~008's output into some other module's output should be
harmless to the MSK~008, but doing that is not recommended because it is
possible the non-MSK~008 module may be harmed.

Although primarily intended to operate on DC control voltages, this module
should be usable at all audio frequencies.  Ultrasonic frequencies are
rolled off to ensure op amp stability.

This module (assuming a correct build using the recommended components) is
protected against reverse power connection.  It will not function with the
power reversed, but will not cause or suffer any damage.  Some other kinds
of power misconnection may possibly be dangerous to the module or the power
supply.

In normal operation the peak current demand of this module is 45mA from the
+12V supply and 50mA from the -12V supply.  Placing a heavy
load on the outputs (for instance, with so-called passive modules) will
increase the power supply current.

\section{Voltage modification}

This circuit was designed for $\pm$12V power and will not work properly on
$\pm$15V power unless modified.  To modify it for $\pm$15V, as well as using
the proper power connector and making sure all components are rated for the
increased voltage, make the following resistor changes:
\begin{itemize}
  \item change R6 and R7 (voltage regulator ballast) from 1.8k$\Omega$ to
    2.4k$\Omega$;
  \item change R13 and R14 (quantizer input weights) from 75k$\Omega$ to
    62k$\Omega$;
  \item change R49 and R51 (LED current control) from 1.2k$\Omega$ to
    1.8k$\Omega$; and
  \item change R50, R52, R53, and R54 (LED current control) from
    910$\Omega$ to 1.2k$\Omega$.
\end{itemize}
I have calculated but not tested these resistor changes.

\section{Source package}

A ZIP archive containing source code for this document and for the module
itself, including things like machine-readable CAD files, is available from 
the Web site at 
\url{https://northcoastsynthesis.com/}.  Be aware that actually building
from source requires some manual steps; Makefiles for GNU Make are provided,
but you may need to manually generate PDFs from the CAD files for inclusion
in the document, make Gerbers from the PCB design, manually edit the .csv
bill of materials files if you change the bill of materials, and so on.

Recommended software for use with the source code includes:
\begin{itemize}
  \item GNU Make;
  \item \LaTeX\ for document compilation;
  \item LaTeX.mk (Danjean and Legrand, not to be confused with other
    similarly-named \LaTeX-automation tools);
  \item Circuit\_macros (for in-document schematic diagrams);
  \item Kicad (electronic design automation);
  \item Qcad (2D drafting); and
  \item Perl (for the BOM-generating script).
\end{itemize}

The kicad-symbols/ subdirectory contains my customised schematic symbol and
PCB footprint libraries for Kicad.  Kicad doesn't normally keep dependencies
like symbols inside a project directory, so on my system, these files
actually live in a central directory shared by many projects.  As a result,
upon unpacking the ZIP file you may need to do some reconfiguration of the
library paths stored inside the project files, in order to allow the symbols
and footprints to be found.  Also, this directory will probably contain some
extra bonus symbols and footprints not actually used by this project,
because it's a copy of the directory shared with other projects.

The package is covered by the GNU GPL, version 3, a copy of which is
included in the file COPYING.

\section{PCBs and physical design}

The enclosed PCB design is for two boards, each
3.90$''$\linebreak[0]$\times$\linebreak[0]1.50$''$, or
99.06mm\linebreak[0]$\times$\linebreak[0]38.10mm.  They are intended to
mount in a stack parallel to the Eurorack panel, held together with M3
machine screws and male-female hex standoff hardware.  See
Figure~\ref{fig:panel-stack}.  Including 18mm of clearance for the mated
power connector, the module should fit in 43mm of depth measured from the
back of the front panel.

\begin{figure}
{\centering
\begin{tikzpicture}[scale=0.1]
  \path[draw=black,dashed,thick] (24.2,-46.95) rectangle (42.2,-26.95);
  \path[draw=black,fill=black!30!white] (-2.0,-64.25) rectangle (0,64.25);
  \path[draw=black,fill=white] (0,-47.45) rectangle (10,-41.45);
  \path[draw=black,fill=white] (0,47.45) rectangle (10,41.45);
  \path[draw=black,fill=blue!50!white] (10,-49.53) rectangle (11.6,49.53);
  \path[draw=black,fill=white] (11.6,-47.45) rectangle (22.6,-41.45);
  \path[draw=black,fill=white] (11.6,47.45) rectangle (22.6,41.45);
  \path[draw=black,fill=blue!50!white] (22.6,-49.53) rectangle (24.2,49.53);
  \path[draw=black,fill=white] (24.2,-47.45) rectangle (26.2,-41.45);
  \path[draw=black,fill=white] (24.2,47.45) rectangle (26.2,41.45);
  \path[draw=black,fill=black!10!white] (26.2,-45.95) rectangle (28.2,-42.95);
  \path[draw=black,fill=black!10!white] (26.2,45.95) rectangle (28.2,42.95);
  \node at (5.0,53) {\parbox{10mm}{\centering \small 10mm standoff}};
  \node at (17.1,53) {\parbox{11mm}{\centering \small 11mm standoff}};
  \node at (13,64) {\small 2mm front panel};
  \node at (18.4,60.5) {\small 2$\times$ 1.6mm PCBs};
  \draw[>=latex',->,very thick] (10.8,58.7) -- (10.8,50.53);
  \draw[>=latex',->,very thick] (23.4,58.7) -- (23.4,50.53);
  \node at (33.2,-36.95)
    {\parbox{17mm}{\centering \small 18mm clearance for mated power connector}};
  \draw (42.2,-48) -- (42.2,-64);
  \draw[>=latex',<->,thick] (0,-56) -- (42.2,-56);
  \node[fill=white] at (21.5,-56) {\small $\approx$43mm depth};
\end{tikzpicture}\par}
\caption{Assembled module, side view.}\label{fig:panel-stack}
\end{figure}

\section{Functional description}

Users will probably want to think about this module in terms of the
applications in which it is \emph{used}: octave switching, wavefolding,
mid-side encoding, and so on.  Even the name ``octave switch'' refers to one
application, and some of the others are described in the section on suggested
patches.  But all these applications, and those yet to be invented, can be
understood in terms of what the module \emph{actually does}, as described
here without direct reference to applications.  See the block diagram in
Figure~\ref{fig:block-diagram}.

\begin{figure*}
{\centering
\begin{tikzpicture}[scale=2]
  \draw[thick] (-3.7,0.3) -- (-3,0.3) -- (-3,1);
  \draw[thick] (-4,1) -- (0,1) -- (0,-1) -- (-2,-1);
  \draw[thick] (-2,0) -- (1,0);
  \node[circle,draw,fill=white,minimum height=6.0,inner sep=0]
    at (-3.6,0.1) {};
  \node[circle,draw,fill=white,minimum height=6.0,inner sep=0]
    at (-3.6,0.5) {};
  \node[circle,draw,fill=white,minimum height=6.0,inner sep=0]
    at (-3.2,0.3) {};
  \node at (-3.9,0.5) {$+1$};
  \node at (-3.9,0.3) {$\phantom{+}0$};
  \node at (-3.9,0.1) {$-1$};
  \node[circle,draw,fill=white,minimum height=6.0,inner sep=0]
    at (-4,1) {};
  \node at (-4.35,0.975) {QUA}; 
  \node[circle,draw,fill=white,inner sep=1] at (-3,1) {\LARGE$+$};
  \node[rectangle,draw,fill=white] at (-1.5,1) {\begin{tabular}{c}
      quantize\\$\{-2,-1,0,1,2\}$
    \end{tabular}};
  \node[circle,draw,fill=white,minimum height=6.0,inner sep=0]
    at (-2,0) {};
  \node at (-2.35,0) {CV1}; 
  \node[isosceles triangle,draw,fill=white,minimum height=32,
        isosceles triangle apex angle=60] at (-1,0) {};
  \node at (-1,0) {$1$};
  \node[circle,draw,fill=white,inner sep=1] at (0,0) {\LARGE$+$};
  \node[circle,draw,fill=white,minimum height=6.0,inner sep=0]
    at (1,0) {};
  \node at (1.35,0) {OUT}; 
  \node[circle,draw,fill=white,minimum height=6.0,inner sep=0]
    at (-2,-1) {};
  \node at (-2.35,-1) {CV2}; 
  \node[isosceles triangle,draw,fill=white,minimum height=32,
        isosceles triangle apex angle=60] at (-1,-1) {};
  \node at (-1,-1) {$\pm1$};
\end{tikzpicture}\par}
\caption{Block diagram (one channel).}\label{fig:block-diagram}
\end{figure*}

There are two channels.  Each channel has a QUA input, which goes into a
quantizer that rounds it to the nearest whole volt in the range
-2V\ldots{}+2V.  So any input voltage from -12V to -1.5V will round to -2V;
any input voltage from -1.5V to -0.5V will round to -1V; any input voltage
from -0.5V to +0.5V will round to 0V; and so on.  There is also a
three-position toggle switch, which can add or subtract one volt at the
quantizer \emph{input}; the quantizer output remains in the range
-2V\ldots{}+2V.

Although the quantizer output voltages are intended to be
accurate (typically 1mV error with decent adjustment; this specification is
not guaranteed), the input thresholds are not meant to be very accurate. 
Whole volts will round correctly, but the exact positions of the boundaries
may not be exact half-volt values, and may shift with the position of the
toggle switch.  Allowing some tolerance on these was a necessary trade-off
to fit the module in 8HP with the desired technology.

When the quantizer output (not the final output of the channel) is positive,
whether +1V or +2V, the channel's LED glows red; when it is negative, green;
when zero, the LED remains dark.

The precision adder sums the quantizer output, the CV1 input, and either the
CV2 input or the negative of the CV2 input.  The result of the summation
appears at the channel output.

Whether a channel adds or subtracts the CV2 input is determined by a jumper
built into the circuit board.  The default, standard for our assembled
modules and any built from kits without modifying the boards, is for the
left channel (as viewed from the front when the module is installed) to add
and the right channel to subtract.  This selection for a channel could be
changed during build by cutting the jumper trace and replacing it in the
opposite direction with a blob of solder.  A more ambitious project might
cut the trace and wire an SPDT switch to the three contact points
provided, allowing front-panel selection between the two modes.  But North
Coast Synthesis Ltd.\ has no current plans to market an ``expander'' for
doing that; it is left as an idea for more advanced hobbyists to pursue on
their own.

The CV1 and CV2 inputs of each channel, but not the QUA input, are
normalized to and from the corresponding inputs on the other channel. 
Plugging a cable into CV1 on either channel will drive both CV1 inputs if
there is no cable plugged into the other CV1, and the same for CV2.  Thus,
in a default jumper configuration, plugging cables into CV1 and CV2 on one
channel will give the sum and difference of the two input voltages on the
two channel outputs.

Each QUA input is normalized to 0V.  Plugging a cable into one of the QUA
inputs with nothing on the other end breaks this normalization and leaves
the comparator inputs to float; that may have unexpected effects. 
Typically, the unconnected input will drive the quantizer to $+$2V with the
toggle in the 0 or $+$1 positions and $-$2V with the toggle in the $-$1
position, but that behaviour is not guaranteed and half-patching the input
this way is not recommended.  Some module ``trigger'' outputs, notably those
on the Befaco Rampage, do not produce any specific voltage but expose a
high impedance like unpatched cables when they are not triggering; those
may produce unexpected results if patched into the MSK~008's QUA input.

\section{Use and contact information}

This module design is released under the GNU GPL, version 3, a copy of which
is in the source code package in the file named \texttt{COPYING}.  One
important consequence of the license is that if you distribute the design to
others---for instance, as a built hardware device---then you are obligated
to make the source code available to them at no additional charge, including
any modifications you may have made to the original design.  Source code for
a hardware device includes without limitation such things as the
machine-readable, human-editable CAD files for the circuit boards and
panels.  You also are not permitted to limit others' freedoms to
redistribute the design and make further modifications of their own.

I sell this and other modules, both as fully assembled products and
do-it-yourself kits, from my Web storefront at
\url{http://northcoastsynthesis.com/}.  Your support of my business is what
makes it possible for me to continue releasing module designs for free. 
The latest version of this document and the associated source files can be
found at that Web site.

Email should be sent to\\ \url{mskala@northcoastsynthesis.com}.
